
\newcommand{\sagadocument}{ExTENCI: SAGA Condor Integration}
\newcommand{\sagaversion}{1.0}
\newcommand{\sagabasename}{saga-programming-guide}
\newcommand{\sagaemail}{saga-users@cct.lsu.edu}

\input{saga_include}

\newcommand{\name}{\F{SAGA}\xspace}
\DefineShortVerb{\|}

\begin{document}

 \thispagestyle{empty}

 \sagadocument\hfill  Ole Weidner, CCT/LSU\\
  Version: \sagaversion \hfill {\sagadate}


  \hrulefill\\[2em]

  \B{\large ExTENCI: SAGA Condor Integration}\\[4em]



  %\U{Abstract}

   The aim of this document is to describe the use-cases and requirements 
   for a possible integration of SAGA and Condor. Based on these requirements, 
   an implementation concept for a Condor plug-in for SAGA (\textit{Adatpor}) is 
   derived and described in detail. 
   With this proposed \textit{Condor Adaptor}, it will be possible to use Condor,
   Condor-G as
   well as Condor glide-in programmatically through the standardized SAGA C++ and 
   Python interfaces.
   
   This project is part of the the Extending Science Through Enhanced National 
   Cyberinfrastructure (ExTENCI) project, which is a joint Open Science Grid (OSG)
   and TeraGrid project, funded by the National Science Foundation Office of 
   Cyberinfrastructure (NSF OCI).
   \\[2em]


  \U{Status of This Document}

  This document is still work in progress.\\
  
    \U{Copyright Notice}

  Copyright \copyright~T.B.D (2011).  All Rights
  Reserved.\\

  \newpage

  \tableofcontents

  \newpage

%-----------------------------------------------------------------
% Intro, structure, disclaimer, ...
%-----------------------------------------------------------------
                                        
	\section {SAGA}
	
	\subsection{API Standard}
	
	\subsection{C++/Python Reference Implementation}
	
	\section {Condor}
	
	\subsection{Condor-G}
	
	\subsection{Condor Glide-in}

\end{document}


