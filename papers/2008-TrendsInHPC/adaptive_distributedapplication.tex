\documentclass[10pt,letterpaper]{article}

\usepackage{graphicx}
\usepackage{url}
\usepackage{color}
\usepackage{ifpdf}
\usepackage{wrapfig}
%\usepackage[cm]{fullpage}
\usepackage{textcomp}
\usepackage{srcltx}
\usepackage{fancyhdr}
\usepackage{setspace}
\usepackage{lscape}
\usepackage{longtable}
\usepackage{paralist}

% \usepackage{ams}

% \ifpdf    %   \usepackage[pdftex,
%               colorlinks=true,
%               linkcolor=red,
%               citecolor=red,
%               filecolor=red,
%               ]{hyperref}
% \else
%   \usepackage[hypertex]{hyperref}
% \fi

% Space saving
\usepackage[small,compact]{titlesec}
\usepackage[small,it]{caption}
\renewcommand\floatpagefraction{.9}
\renewcommand\topfraction{.9}
\renewcommand\bottomfraction{.9}
%\renewcommand\textfrare{.1}
\setcounter{totalnumber}{50}
\setcounter{topnumber}{50}
\setcounter{bottomnumber}{50}

\textwidth = 6.5315 in
\textheight = 9.0315 in
\oddsidemargin = 0.0 in
\evensidemargin = 0.0 in
\topmargin = 0.0 in
\headheight = 0.0 in
\headsep = 0.0 in
%\parskip = 0.0in
%\parindent = 0.5cm
\parskip = 0.0716in
\parindent = 0.0cm
\textfloatsep = 0.1in

\ifpdf \pdfinfo{ /Author (Shantenu Jha et al.)  /Title (Programming Abstractions for Large-Scale Distributed Applications) }

  \usepackage[pdftex,colorlinks=true, linkcolor=blue,citecolor=blue,
       urlcolor=blue]{hyperref}
\fi

\graphicspath{{images/}{../images}}

\newcommand{\projectnamefull}{\textit{} }

\newcommand{\upup}{\vspace*{-0.5em}}
\newcommand{\up}{\vspace*{-0.25em}}

\newcommand{\I}[1]{\textit{#1}}
\newcommand{\B}[1]{\textbf{#1}}
\newcommand{\T}[1]{\texttt{#1}}

\newcommand{\BI}[1]{\B{\I{#1}}}

\ifpdf
  \DeclareGraphicsExtensions{.pdf, .png, .jpg}
\else
  \DeclareGraphicsExtensions{.ps, .eps}
\fi

\long\def\comment#1{{\bf \textcolor{magenta}{\bf #1}}}
\long\def\ccomment#1{{\bf \textcolor{blue}{\bf #1}}}
\newcommand{\C}{\comment}
\newcommand{\CC}{\ccomment}

\newcommand{\yes}{$\bullet$}

\newif\ifdraft
\drafttrue

\ifdraft
 \newcommand{\jhanote}[1]{  {\textcolor{red}     { ***Shantenu: #1 }}}
 \newcommand{\luckownote}[1]{ {\textcolor{magenta}    { ***Luckow:      #1 }}}
 \newcommand{\yyenote}[1]{ {\textcolor{blue}    { ***YYE:     #1 }}}
 \newcommand{\amnote}[1]{   {\textcolor{magenta} { ***Merzky:    #1 }}}
\else
 \newcommand{\jhanote}[1]{}
 \newcommand{\luckownote}[1]{}
 \newcommand{\yyenote}[1]{}
 \newcommand{\amnote}[1]{}

\fi


\begin{document}

\title{Adaptive Distributed Applications: Scaling From HTC to HPC} 

\author{Shantenu Jha$^{1,2,3}$, Yaakoub El-Khamra$^{1}$,  Hartmut Kaiser$^{1}$, \\
  Joohyun Kim$^{1}$, Andre Luckow$^{4}$, Andre Merzky$^{1}$,  Ole Weidner$^{1}$ \\[1em]
  \small $^1$Center for Computation and Technology,
  Louisiana State University\\[-0.3em]
  \small $^2$Department of Computer Science,
  Louisiana State University\\[-0.3em]
  \small $^3$e-Science Institute, Edinburgh\\[-0.3em]
  \small $^{4}$Institute of Computer Science, Potsdam University, Germany\\
}

\maketitle
%\tableofcontents

\vspace*{5mm}

\section*{Abstract}

In this chapter, we will discuss adaptive applications -- an important
class of distributed applications. We will discuss how the Simple API
for Distributed Applications (SAGA) has been used to develop a broad
range of adaptive distributed applications. We will discuss how SAGA
can be used as the building block and basis for a large range of
applications types and classes. SAGA presents a comprehensive
``progamming language'' for distributed systems by supporting the most
commonly required functionality.  We will go on to discuss several
specific distributed applications that have been developed using SAGA
and present both details of their execution environments and some
preliminary performance results as the number of distributed resources
utilised increases.


\newpage

\section{Introduction and Outline}
\textcolor{blue}{Shantenu} \hspace{0.1in} \textcolor{blue}{}
\hspace{0.1in} \textcolor{blue}{}


\jhanote{The aim of the book chapter is to discuss the two adaptive
  distributed applications that we have developed using SAGA. This
  chapter will build upon the earlier papers:\\
  i. TeraGrid 08 \\
  ii. Phil Trans Paper and e-Science 2008 for replica exchange\\
  SJ will write the introduction, wherein a general discussion of
  using distributed systems will be presented, the importance of
  adaptive applications in order to utilise these distributed systems
  will be introduced, a brief introduction to developing distributed
  applications will be presented.}

Much development has focused on the support for legacy parallel and
cluster applications codes, without changing application usage and
execution modes. In part, this is also due to a lack of suitable
abstractions to re-architect ``legacy applications'' to have agile
execution and usage modes, as well as to develop first principles
distributed applications. There exist an unexplored range of
applications that can benefit from the Grid paradigm; progress on this
front will come from application development that does not depend on
the homogeneous, isolated and relatively static model of resource
allocation and performance inherited from parallel or cluster
legacies. In many ways, the current situation is analogous to that
which existed when vector processing hard- ware was
introduced. Scientific codes written for previous architectures could
be ported to vector machines, but the true power of the approach was
only available to hand-coded assembly language examples that could
directly exploit the vector hardware and memory model. Until codes
were re-written and new codes developed specifically with vectorization
in mind, the performance benefits were not fully realized.  The lack of
such application-level programming abstractions is compounded by the
fact that there exist often incompatible middleware systems in both
research and production environments.

In general, applications that are able to leverage the heterogeneity
and dynamic performance response inherent in distributed systems to
their advantage, have proved exceedingly difficult to implement.

There are efforts to harmonize different Grids by using a medley of
tools, inter-operational efforts and standardization in the many layers
below the application level. Although driven with the noble aim of
increasing the utilization of Grids, their impact in facilitating
applications remains unclear; these are likely to be necessary steps
towards effective distributed systems, but not sufficient.  To address
these challenges and in particular to find a solution to the universal,
apparently intractable problem of successfully enabling applications
to utilize multiple, distributed resources seamlessly, several
applications groups expressed the desire for a simple programmatic
interface that is widely-adopted and available. The goal of such an
interface would be to provide a ``grid counterpart to MPI'' (at least
in impact if not in details) and that would supply developers with a
simple, uniform, and standard programmatic interface with which to
develop applications.

The need now is to demonstrate the unique benefits of adopting the SAGA
paradigm by delivering concrete examples of advanced Grid
applications.

\section{The Simple API for Grid Applications}\textcolor{blue}{SJ}


\subsection{Our Solutions}

Brief discussion of FAUST

\section{Adaptive Distributed Applications}  \textcolor{blue}{SJ}

Matching requirements to resources \\
Fault tolerance \\
Mechanism to utilise heterogenity \\

Must discuss why this application needs to be i) distributed ii) adaptive/dynamic, iii) use different types of resources

\subsection{Some very simple discussion of i) infrastructure used,
  ii) types of runs iii) results..}

\section{Ensemble Kalman Filter}  \textcolor{blue}{YYE}

\jhanote{Yaakoub is there a better name for this section?}

Must discuss why this application needs to be i) distributed ii) adaptive/dynamic, iii) use different types of resources

\section{Adaptive Distributed Replica-Exchange Simulations}

\section{Challenges}\textcolor{blue}{Ole, SJ, Luckow}

Need to say a few things about schedulers and other factors
that impede adaptivity? \\


\section*{Acknowledgements}


\bibliographystyle{unsrt}
\bibliography{dpa_surveypaper}
\end{document}

