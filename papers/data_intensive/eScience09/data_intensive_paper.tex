%\documentclass[times, 10pt, twocolumn]{article}
%\documentclass[conference,final]{IEEEtran}

\documentclass{rspublic}

%------------------------------------------------------------------------- take
%the % away on next line to produce the final camera-ready version
%\pagestyle{empty}

\usepackage{graphicx}
\usepackage{float}
\usepackage{times}
\usepackage{multirow}
\usepackage{listings}
\usepackage{paralist}
\usepackage{wrapfig}
\usepackage[small,it]{caption}
%%\usepackage{multirow}
\usepackage{ifpdf}
\usepackage{subfigure}
\usepackage{url}

%%\usepackage{subfig}
%\usepackage[pdftex]{graphicx}
%\usepackage{harvard}
%\usepackage{pdfsync}

%Bibliography
\usepackage{natbib}
\usepackage{listings}
\usepackage{keyval}
\usepackage{color}
\definecolor{listinggray}{gray}{0.95}
\definecolor{darkgray}{gray}{0.7}
\definecolor{commentgreen}{rgb}{0, 0.4, 0}
\definecolor{darkblue}{rgb}{0, 0, 0.4}
\definecolor{middleblue}{rgb}{0, 0, 0.7}
\definecolor{darkred}{rgb}{0.4, 0, 0}
\definecolor{brown}{rgb}{0.5, 0.5, 0}
\definecolor{orange}{rgb}{1,0.5,0}

\lstdefinestyle{myListing}{ frame=single, backgroundcolor=\color{listinggray},
  %float=t,
  language=C, basicstyle=\ttfamily \footnotesize, breakautoindent=true,
breaklines=true tabsize=2, captionpos=b, aboveskip=0em,
  %numbers=left, numberstyle=\tiny
}

\lstdefinestyle{myPythonListing}{ frame=single,
backgroundcolor=\color{listinggray},
  %float=t,
  language=Python, basicstyle=\ttfamily \footnotesize,
breakautoindent=true, breaklines=true tabsize=2, captionpos=b,
  %numbers=left, numberstyle=\tiny
}

\title[Understanding Performance Implications of Distributing Data for
Data-Intensive Applications]{Understanding Performance Implications of
Distributing Data for Data-Intensive Applications}


\author[Miceli, Miceli, Rodriguez-Milla, Jha]{ Christopher Miceli$^{1}$,
Michael Miceli$^{1}$, Bety Rodriguez-Milla$^{1}$, Shantenu Jha$^{1,2,*}$ \\
\small{\emph{$^{1}$Center for Computation \& Technology, Louisiana State
University, USA}} \\  \small{\emph{$^{2}$Department of Computer Science,
Louisiana State University, USA}} \\ {\footnotesize {\hspace{0.0 in}
$^*$Corresponding Author sjha@cct.lsu.edu}} }

%\date{}

\def\acknowledgementname{Acknowledgements} \newenvironment{acknowledgement} 

% {\section*{\acknowledgementname}%\parindent=0pt% }

\newif\ifdraft \drafttrue \ifdraft \newcommand{\fixme}[1]{ { \bf{ ***FIXME: #1
}} } \newcommand{\jhanote}[1]{ {\textcolor{red} { ***Jha: #1 }}}
\newcommand{\micnote}[1]{ {\textcolor{blue} { ***Michael: #1 }}} 
\newcommand{\betynote}[1]{ {\textcolor{orange} { ***Bety: #1 }}}
\else
\newcommand{\jhanote}[1]{} \newcommand{\micnote}[1]{}\newcommand{\betynote}[1]{} \newcommand{\fixme}[1]{}
\fi

\begin{document} \maketitle

% \micnote{This can't be more than 200 words. The summary should be
% concise and informative. It should be complete by itself, and must not
% contain references or unexplained abbreviations. It should not only
% indicate the general scope of the article but also state the main
% results and conclusions. Please note that footnotes are not used.}

\begin{abstract}{data-intensive computing, distributed computing,
cloud computing, grid computing} 
Grids, clouds and cloud-like infrastructures are capable of supporting
a broad range of data-intensive applications. There are interesting
and unique performance issues that appear as the volume of data and
degree of distribution increases. New scalable data placement and
management techniques, as well as novel approaches to determine the
relative placement of data and computational workload are required. We
develop and study an All-Pairs based genome sequence matching
application as a representative data-intensive applications.  This
paper aims to understand the factors that influence the performance of
this application, and to understand their interplay.  We are not aware
of similiar approaches for data-intensive applications, even though
analogous benchmarking HPC application on a new platform is
established practise. We also demonstrate how the SAGA approach can
enable data-intensive applications to be extensible and interoperable
over a range of infrastructure.

% Grids, clouds and cloud-like infrastructures are capable of supporting a
% broad range of data-intensive applications. There are interesting and
% unique performance issues that appear as the volume of data increases
% which require scalable data placement and management techniques, as well
% as novel approaches to the relative placement of data and computational
% workload. This paper aims to understand the factors that determine the
% performance of a representative data-intensive application, and to
% understand the performance trade-offs in design decisions. This is
% analogous to benchmarking the performance of an application on a new
% platform. We analyse two techniques to manage data placement. One
% focuses on data placement and the other on worker placement. The goal of
% this paper is to understand techniques for distributing data in a
% distributed environment and understand performance issues associated
% with these techniques.

\end{abstract}

% Grids, clouds and cloud-like
%   infrastructures are capable of supporting problems that are
%   data-intensive. There are interesting and unique performance issues
%   that appear as the volume of data increases which require scalable
%   data placement and management techniques, as well as novel
%   approaches to the relative placement of data and computational
%   workload.  The aim of this paper is to understand the factors that
%   determine the performance of a representative data-intensive
%   application, and to understand the performance trade-off in design
%   decisions. This is analagous to benchmarking the performance of an
%   application on a new platform. We analyse two techniques to manage
%   data placement. One focuses on data placement and the other on
%   worker placement. The goal of this paper is to understand techniques
%   for distributing data in a distributed environment and understand
%   performance issues associated with these techniques.

% As a simple example, local data-intensive applications are nearly
% always I/O bound typically dominated by Bandwidth for Cluster/Parallel
% systems, I/O dominated by latency.

\section{Introduction} 

The role of data-intensive computing is increasing in many aspect of
science and engineering~\cite{fourthparadigm} and other
disciplines. For example, Google, processes around 20 petabytes of
data per day ~\citep{google}, with trends showing continuing
growth. In addition to increasing volumes of data, there are several
reasons driving the distribution of the data.

The challenges in developing effective and efficient distributed
data-intensive applications are a complex interplay of distributed
applications and data-intensive applications, with different design \&
performance metrics. New algorithmic, infrastructure and
data-management techniques are required to handle large data-volumes
effectively. In particular at such scales data-placement and
data-scheduling need increased attention, and thus distributed
applications need to be take precautions when placing, scheduling, and
managing large volumes of data. For distributed systems and
applications the challenge is to find an optimal distribution strategy
that takes into account the ratio of computation workload to data
distribution. Data privacy, security and access policy are crucial
issues but typically are not determinants of performance for
distributed applications.

% However, inefficient data placement can adversely affect system
% performance greatly.  It is decisive then to determine whether to move
% input data to the computational resource, or the computational
% workload to the input data.

In general, there are many degrees-of-freedom that determine the
performance of a given application on distributed infrastructure. Thus
a rigorous benchmarking process, which provides repeatable, extensible
and verifiable performance tests on different distributed platforms is
required. This is analogous to the situation of developing and
testing benchmarks for high performance computing (HPC). The aim of
this paper is to provide some initial approaches to answer the
question of whether ``to distribute or not to distribute data''...
This paper uses a real data-intensive distributed applications to
determine the performance ... Our research focuses on understanding
the performance trade-offs of a DFS compared to ''regular''
distribution and placement techniques, as well as more advanced
intelligent distribution methods and finding how they handle different
data-sets and what the performance patterns are.

\jhanote{Compact. Simplify. Possibly move parts to later section; not
 in introduction} Distributed filesystems (DFS) have come of age,
with multiple open-source, reliant file-systems now available. DFS are
useful and effective tools to consider for data-intensive scientific
applications. A DFS controls the data placement and provides a
uniform interface for accessing files on multiple hosts. Thus it is
worth considering DFS as a viable infrastructure. But as the
underlying algorithms, scheduling strategies and implementations vary
greatly between different infrastructure, it is difficult to estimate
{\it a priori} the application-level performance on a given DFS.
.... Frequently, there is more than one copy of the input data for
fault-tolerance reasons, consequently, the added issue of deciding
between the two or more replicas becomes relevant. While a DFS
removes the responsibility of replica management and data server
placement, the abstraction often increases the difficulty in
determining where in the DFS the data is being stored. This puts
pressure on a DFS's protocols and internal algorithms to perform
well. Despite this, the DFS replication may alleviate this issue by
placing replicas in locations where computational resources reside. A
downfall of DFS is the inability to make the decision of whether to
move the input data, or the computational workload. It can only focus
on minimizing poor data management. The most common parameters in
determining the performance of using a DFS are the performance
overhead compared to a normal local filesystem, number of replicas of
each datum/file, and the number of servers. Our intelligent framework
method differs from a DFS in that it determines where data is and
where the work should go. Determining data location can be as simple
as looking at the IP address of the worker and seeing geographically
where it is located, or as complicated as using network analyses tools
to determine the optimal data transfer minimization time. For this
method, we use gridFTP\jhanote{place proper citation for gridftp}, a
tool that is used to transfer files across machines in a grid. It is
specifically designed for high-bandwidth networks.
 
In this paper, we investigate two ways to handle this
issue... \jhanote{which issue?}  -- with distributed filesystems,
which focus on data placement, or with the use of an intelligent
framework, which focuses on worker placement. This paper is structured
as follows...

% \section{Introduction} Data-intensive computing is a fast growing area
% of computer science. A good example of this is Google, which processes
% around 20 petabytes of data per day ~\citep{google}, and trends show
% continuing growth. It has become very important that a distributed
% application developer takes precautions when placing, scheduling, and
% managing large volumes of data. Careless placement can adversely affect
% system performance greatly. It is decisive then to determine whether to
% move input data to the computational resource, or the computational
% workload to the input data. There are two ways to handle this issue,
% with distributed filesystems, which focus on data placement, or with the
% use of an intelligent framework, which focuses on worker placement. 

% \jhanote{refine} Different metrics of concern for parallel
% vs. distributed data (I/O).  I/O typically dominated by Bandwidth for
% Cluster/Parallel systems, I/O dominated by latency. The challenge in
% load balancing for the former is often disc or bus-limited at the
% hardware level, while at the software/application level the challenge
% is to increase concurrent I/O with computation. For distributed
% systems and applications the challenge is to find an optimal
% distribution strategy that takes into account the ratio of
% computation workload to data distribution.  Data privacy, security and
% access policy is a crucial non-technical issue for distributed
% applications.  


% Distributed filesystems (DFS), motivated in part by developments in
% cloud computing, are useful and effective tools to consider for
% data-intensive scientific applications. A DFS controls the data
% placement and provides a uniform interface for accessing files on
% multiple hosts. Frequently, there is more than one copy of the input
% data for fault-tolerance reasons, consequently, the added issue of
% deciding between the two or more replicas becomes relevant. While a DFS
% removes the responsibility of replica management and data server
% placement, the abstraction often increases the difficulty in determining
% where in the DFS the data is being stored. This puts pressure on a
% DFS's protocols and internal algorithms to perform well. Despite this,
% the DFS replication may alleviate this issue by placing replicas in
% locations where computational resources reside. A downfall of DFS is
% the inability to make the decision of whether to move the input data, or
% the computational workload. It can only focus on minimizing poor data
% management. The most common parameters in determining the performance of
% using a DFS are the performance overhead compared to a normal local
% filesystem, number of replicas of each datum/file, and the number of
% servers. In our experiments, we use the stable open source distributed
% filesystem CloudStore (formerly KFS), which is written in C++ released
% under the Apache License Version 2.0. It is inspired by the highly
% successful Google Filesystem, GFS ~\citep{cloudstore_web}, which is
% closed source and unavailable for research. CloudStore was chosen for
% its high performance focus, C++ implementation, and its source code
% availability. It also provides a means to automatically replicate data
% on different hosts to provide efficient data access and fault tolerance. 

% Our intelligent framework method differs from a DFS in that it
% determines where data is and where the work should go. 
% Determining data location can be as simple as
% looking at the IP address of the worker and seeing geographically where
% it is located, or as complicated as using network analyses tools to
% determine the optimal data transfer minimization time. 
% For this method, we use gridFTP, a tool that is used to transfer
% files across machines in a grid. It is
% specifically designed for high-bandwidth networks. GridFTP is provided by
% the Globus Toolkit, an open source software toolkit
% released under the Apache License version 2.0. Globus provides tools used
% to create and manage grid infrastructures. 

% In this paper, we provide some initial approaches to answer ``To
% distribute or not to distribute data, is the question''. Our research
% focuses on understanding the performance trade-offs of a DFS compared to
% ''regular'' distribution and placement techniques, as well as more
% advanced intelligent distribution methods and finding how they handle
% different data-sets and what the performance patterns are. This paper
% also aims to determine how sensitive the performance is in the context
% of a real data-intensive distributed applications.

\section{SAGA and SAGA-based Frameworks for Large-Scale and
 Distributed Computation}\label{Sec:SAGA}

%\alnote{removed the first paragraph - duplicated content}
% SAGA~\cite{saga_url} provides a simple, POSIX-inspired API to the most
% commonly required distributed functionality at a sufficiently
% high-level of abstraction so as to be independent of the divers\ e and
% dynamic Grid environments.

The Simple API for Grid Applications (SAGA) is an API
% standardization effort within the Open Grid Forum
% (OGF)~\cite{saga_gfd90}, an international standards development body
% concerned primarily with standards for distributed computing. 
that provides a simple, POSIX-style API to the most common distributed
functions at a sufficiently high-level of abstraction so as to be
independent of the diverse and dynamic Grid environments. The SAGA
specification defines interfaces for the most common Grid-programming
functions grouped as a set of functional packages
(Fig.~\ref{Fig:SAGA1}). Some key packages are:

\begin{figure}[!ht]
 \begin{center}
     \includegraphics[width=0.40\textwidth]{stci_saga_figures-1.pdf}
    \includegraphics[width=0.45\textwidth]{distributed_applications_saga_figure.pdf}
\end{center}
\caption{\small [L] Layered schematic of the different components of
  the SAGA landscape. At the topmost level is the simple integrated
  API which provides the basic functionality for distributed
  computing. Our BigJob abstraction is built upon this SAGA layer
  using Python API bindings. [R] Showing the ways in which SAGA can be
  used to develop distributed applications. The different shaded box
  represent the three different types; frameworks in turn can capture
  either common patterns or common application
  requirements/characteristics.} \label{Fig:SAGA1}
\end{figure}

(i) File package - provides methods for accessing local and remote
filesystems, browsing directories, moving, copying, and deleting
files, setting access permissions, as well as zero-copy reading and
writing. (ii) Job package - provides methods for describing,
submitting, monitoring, and controlling local and remote jobs. Many
parts of this package were derived from the largely adopted DRMAA
specification. (iii) Stream package - provides methods for
authenticated local and remote socket connections with hooks to
support authorization and encryption schemes. (iv) Other Packages,
such as the RPC (remote procedure call) and Replica package.

% \begin{itemize}
% \item File package - provides methods for accessing local and remote
%  filesystems, browsing directories, moving, copying, and deleting
%  files, setting access permissions, as well as zero-copy reading and
%  writing.
% \item Job package - provides methods for describing, submitting,
%   monitoring, and controlling local and remote jobs. Many parts of
%   this package were derived from the largely adopted DRMAA
%   specification.
% \item Stream package - provides methods for authenticated local and
%   remote socket connections with hooks to support authorization and
%   encryption schemes.
% \item Other Packages, such as the RPC (remote procedure call) and
%   Replica package.
% \end{itemize}


In the absence of a formal theoretical taxonomy of distributed
applications, Fig.~\ref{Fig:sagaapps} can act a guide. Using this
classification system, there are three types of distributed
applications: (i) Applications where local functionality is swapped
for distributed functionality, or where distributed execution modes
are provided. % A simple but illustrative example is an ensemble of an
% application that uses distributed resources for bulk submission. Here,
% the application remains unchanged and even unaware of its distributed
% execution, and the staging, coordination, and management are done by
% external tools or agents. Most application in this category are
% classified as implicitly distributed.  
(ii) Applications that are naturally decomposable or have multiple
components are then aggregated or coordinated by some unifying or
explicit mechanism. %  DAG-based workflows are probably the most common
% example of applications in this category.
Finally, (iii) applications that are developed using frameworks, where
a framework is a generic name for a development tool that supports
specific application characteristics (e.g., hierarchical job
submission), and/or recurring patterns (e.g., MapReduce, All-Pairs).
SAGA provides the basic API to implement distributed functionality
required by applications (typically used directly by the first
category of applications), and is also used to implement higher-level
APIs, abstractions, and frameworks that, in turn, support the
development, deployment and execution of distributed
applications~\cite{saga_gmac09}. SAGA has been used to develop
system-level tools and applications of each of these types. In
Ref.~\cite{saga_montage_escience09} we discussed how SAGA was used to
implement a higher-level API to support workflows. In
Ref.~\cite{saga_ccgrid09} we discussed how MapReduce could be
developed.

% In this paper, we will discuss how SAGA can be used to implement
% runtime frameworks to support the efficient execution of the
% distributed applications.

% \begin{figure}[!ht]
%   \begin{center}
%     \includegraphics[width=0.45\textwidth]{distributed_applications_saga_figure.pdf}
%   \end{center}
%   \caption{\small Showing the ways in which SAGA can be used to
%     develop distributed applications.  The different shaded box
%     represent the three different types; frameworks in turn can
%     capture either common patterns or common application
%     requirements/characteristics. \label{Fig:sagaapps}}
% \end{figure}



\section{All-Pairs: Design, Development and Infrastructure} We use an
application based upon a grid-enabled All-Pairs abstraction in SAGA
(Simple API for Grid Applications, see Sec. \ref{Sec:SAGA}). The
All-Pairs application was chosen because of its similarity to many
other data-intensive distributed applications. This enables our
results to be abstracted to describe and predict different
applications besides ones based upon the All-Pairs
construct. \jhanote{a bit more describing the fact that this is a
 pattern and how benefits will be general}. This abstraction applies
an operation on two data-sets such that every possible pair containing
one element from the first set and one element from the second set has
some operation applied to it.~\citep{Interop, AllPairs}. Essentially,
All-Pairs is a function of two sets, $A$ and $B$, with number of
elements $m$ and $n$, respectively, which creates a matrix $M$. Each
element $M_{i,j}$ is the result of the operation $f$ applied to the
elements $A_i$ and $B_j$.
\begin{eqnarray}
 AllPairs(A, B, f) & \rightarrow & M_{m \times n}, \\
\mbox{where} \quad M_{i,j} & = & f(A_{i},B_{j})
 \end{eqnarray}

 The result of this application is stored in a matrix similar to Fig.
 \ref{Fig:AllPairsExplanation} . The application spawns distributed
 jobs to run sets of these function operations. Examples of problems
 that fall into this category are image comparison for facial
 recognition, and genome comparison.  The usefullness of All-Pairs
 comes from the ability to easily change the comparison function.  We
 use a SAGA-based All-Pairs framework and simply implemented the
 comparison function.  Our comparison function compares genome to find
 the best matching gene in a genome.  The function finds the number of
 similarities among the genes and returns the percentage of the genes
 that are identical. Our genome comparison application can be
 classified as having a large data throughput, as, even though it has
 large input $O$(GB) and relatively small output $O$(KB), the manner
 of processing causes many data reads.


\begin{figure}[!ht]
 \begin{center}
     \includegraphics[width=0.50\textwidth]{data/allpairs-exp.pdf}
\end{center}
\caption{\small An example result from an All-Pairs enabled application.
Each matrix element describes the similarity between the corresponding
sets. (Larger values indicate more similarity.)}
 \label{Fig:AllPairsExplanation}
\end{figure}

The problem becomes determining which pairs to put into an assignment
set and with which distributed resource to run that set.  \jhanote{The
  following should be shortened} If transferring data to the job takes
too long, we spend more time on data transfer than computation There
may be a resource capable of the work that may be slower than others,
but able to be accessed in a relatively quick manner via the network,
correcting for this lack of computational ability. An intelligent
application attempts to predict and determine a specific data set's
affinity to a specific network resource.

%\subsection{Infrastructure Used}

{\it Infrastructure Used: } In our experiments, we use the stable open source distributed
filesystem CloudStore (formerly KFS), which is written in C++ released
under the Apache License Version 2.0. It is inspired by the highly
successful Google Filesystem, GFS ~\citep{cloudstore_web}, which is
closed source and unavailable for research. CloudStore was chosen for
its high performance focus, C++ implementation, and its source code
availability. It also provides a means to automatically replicate data
on different hosts to provide efficient data access and fault
tolerance.

To test CloudStore we wrote an adaptor for SAGA that implements the
filesystem package. This enables us to compare CloudStore with
gridFTP, while still using the same application. Doing this with SAGA
is really simple. For accurate comparisons,we must consider the time
differences between the gridFTP adaptor and the CloudStore adaptor. We
have measured a slight difference in times, but none that would affect
the outcomes of our experiments.  SAGA allows our application to
handle seamlessly the DFS and gridFTP based data stores on clouds and
grids.  This allows the same exact application to be used for all of
our experiments.



\section{Performance Measurement and Analysis} We developed three types
of experiments in order to compare distributed filesystems with manual
file management. There are many questions we try to answer. For
example, does a distributed filesystem grow more slowly than manual
placement of data? When manually handling data, what are the advantages
of being able to move work to data to the work? For this, we find the
time to completion $t_c$, which is defined as
 \begin{equation}
t_c = t_x + t_{I/O} + t_{compute},
\end{equation}
where $t_x$ is the pre-processing time, whose the dominant component is
the time for transfer, $t_{I/O}$ is the time it takes to read and write
files, and $t_{compute}$ is the time it takes our comparison function to
run. We focus on three variables to measure $t_c$: degree of
distribution, data dependency, and workload. The degree of distribution
($D_d$) is defined as the number of resources that are utilized for a
given computation/problem. For example, if data is distributed over 3
machines $D_d$ is 3; if data is distributed over three machines but the
computational tasks over 4 machines, the $D_d$ is 4.

\subsection{Experimental Configuration}

As explained before, for our experiments, we use an All-Pairs
implementation that utilises SAGA. An XML configuration file defines
various initial parameters which alter the behavior of our All-Pairs
implementation. The configuration file defines the location of data that
comprises the two input sets, the grouping of pairs from these sets to
be provided to the compute resources, and the available machines that
will perform the operation on these sets of pairs. The application takes
these groups of pairs and maps them to a computational resource
dynamically at run-time 

Furthermore, variables external to the All-Pairs implementation also
influence experimental results. The following experiments can be
completely described by a tuple of the following form
 \begin{equation}
(c_s, N_c, M_c, fs, m,r),
\label{Eq:tuple}
\end{equation}
where $c_s$ is the total amount of data in each file of a set (i.e.,
$c_s=\mbox{\textit{chunk} size}$); $N_c$ is the number of work-loads
that the total work is decomposed into, i.e., number of work assignments
generated, and this a measure of granularity of work; $M_c$ is a comma
separated list of the machine configurations of the following form:
$X(c, d)$ where $X$ is a shorthand reference for the computational
resource, $c$ shows if the computational resource $X$ was used in the
computational workloads/calculations, and $d$ if the computational
resource $X$ assisted in data storage, both have a yes/no $(Y/N)$
value.; $fs$ is the type of filesystem used; $m$ is the method used to
access that filesystem; and $r$ is the degree of replication utilized in
the experiment (with a default value of 1). However, for CloudStore, we
investigate the performance with $r = 1, 2 \mbox{ and } 3$.

In our experiments, we have three $(fs, m)$ configurations, and five $X$
configurations. Our $(fs, m)$ configurations are $(L,L)=(\mbox{local,
local})$, $(L,G)=(\mbox{local, gridFTP})$, and $(C,D)=(\mbox{CloudStore,
direct})$. By direct we mean CloudStore controls the data access. Our
$X$ configurations are enumerated in Table \ref{Tab:Configs} below. For
one machine, $C1=X_1(Y,Y)$, where resource $X_1$ has both the data and
the computing; for two machines, we have three configurations, and for
three machines we only work with only one configuration. $M_c$ is a very
important configuration parameter as it determines where data is, the
number of workers, and indirectly the granularity of work. By
granularity, we mean that if we do not have enough distinct resources
for data storage, all data requests will go through a small number of
machines, reducing the effectiveness of adding more compute resources.
Similarly, if we have few compute resources, intelligently splitting and
distributing data becomes futile as the compute resources are being
fully utilised.

\begin{table}
\begin{center}
    \begin{tabular}{ | l | l | l |}
    \hline
    Configurations & $X(c,d)$; $c= \mbox{compute}$, $d=\mbox{data storage}$ & Description  \\ \hline
    $C1$ & $X_1(Y,Y)$  & $X_1$ computes and stores data\\ \hline    
    $C2$ & $X_1(Y,N), X_2(N,Y)$  & $X_1$ computes, $X_2$ stores data \\ \hline
    $C3$ & $X_1(Y,Y), X_2(N,Y)$  & $X_1$ computes, $X_1$, $X_2$ store data \\ \hline
    $C4$ & $X_1(Y,Y), X_2(Y,Y)$  & $X_1$, $X_2$ compute and store data \\ \hline
    $C5$ & $X_1(Y,Y), X_2(Y,Y), X_3(Y,Y)$  & $X_1$, $X_2$, $X_3$ compute and store data \\ 
    \hline
    \end{tabular}
\end{center}
    \caption{\textit{Here we show the machine configurations $M_c$ (tuple
\ref{Eq:tuple}) that we use in our experiments, for one, two, and three
machines. Both $c$ and $d$ can have yes/no (Y/N) values. A $c = Y$ means
the machine $X_i$ does computation, and a $d = Y$ means the machine has
data stored. For C4 and C5, we divide the data equally among the
machines.}}
    \label{Tab:Configs}
\end{table}

An sample description of an experiment will now be explained:
 \begin{equation}
(287 \mbox{MB}, 8, C2, C, D, 1),
\end{equation}
shows that each element of a set is 287 MB in size; we have 8
assignments; the computational resource $X_1$ does calculations, but
does not have data stored, while computational resource $X_2$ does not
calculate, but stores the data; the filesystem used is CloudStore, it
directly access the files, and we have a replication factor of 1 for our
data. The machines $X_i$ we use for our experiments are part of LONI
(Louisiana Optical Network Initiative). For most of our experiments, the
number of assignments is 8, unless otherwise specified. As described
above, the All-Pairs implementation used for our experiments has a fixed
distribution of data, fixed available computational resources, and fixed
sets of pairs to operate with. These variabilities listed above (tuple
\ref{Eq:tuple}) are manipulated to determine causes for different I/O
complexities observed in an attempt to build an understanding of issues
that arise when utilising data-intensive applications. It is also
notable that the following experiments were consistent and reproducible
for a given time, but could vary if run more than a few hours apart.
This variance is attributable to the amount of use that our computing
environment (LONI) was experiencing at the time of the experiment.

\subsection{Experiment I: Baseline Performance}
In the first experiment, we do not have an actual operation being
applied on the pairs, giving us $t_{compute}=0$, this is, we evaluate
data dependencies without the added variable of computation. We use
this to examine the I/O, transfer and coordination costs. We run the
SAGA-based All-Pairs application on one, two, and three unique machines
on a grid (LONI), without any specific data placement strategy; also,
no replication or fault-tolerance takes place. The application
sequentially assigns sets of pairs to the first available computational
resource. All data is accessed via the gridFTP protocol. An important
fact to notice is the essentially random mapping of data sets to
computational resources based on availability. This is to mimic a naive
data-based application. In figure \ref{Fig:ExpIConventionalLocal}, we
find $t_c$ as we vary the number of workers $N_w$. We show our results
for data accessed without any protocol (local case) and data accessed
via gridFTP (Figs. \ref{Fig:ExpIConventionalLocal:a}, and
\ref{Fig:ExpIConventionalLocal:b}, respectively). Using our All-Pairs
framework accessing the data using gridFTP protocol had an overhead
which can be noted by looking at the $y-$scales of both graphs. In figure
\ref{Fig:ExpIConventionalLocal:a}, our local cases, we see that working
with a smaller data set ($c_s \sim 144MB$, $N_c = 8$, 1.15 GB total)
took about half the time than working with a data set double the size
($c_s = 287MB$, $N_c = 8$, 2.3 GB total). We also see that when working
with the same data set size ($2.3 GB$), but partitioned differently,
i.e., S0 ($c_s = 287MB$, $N_c = 8$) vs. S1 ($c_s = 144MB$, $N_c = 16$),
$t_c$ increased for S1, due to added transfer time by doubling the
number of files, although we decreased the file size by half.
\betynote{Only transfer? I/O is the same, right?}In figure
\ref{Fig:ExpIConventionalLocal:b}, where we used gridFTP to access the
files, we see that a single machine took less time compared to the
configurations that involved two machines. Also, having to access data
remotely was a disadvantage, $t_c(\mbox{S1}) > t_c(\mbox{S2})$. For the
one machine configuration, $t_c$ was approximately constant, probably
caused by an I/O bound. For all our cases, it is expected that $t_c$
decreases with increasing number of workers; however, after a critical
$N^c_w$, $t_c$ will increase because it will take more time to
coordinate the workers.

\begin{figure}[!ht]
\begin{center}
\subfigure[ $(M_c, fs, m)=(C1, \mbox{local, local})$]{
\includegraphics[scale=0.48]{data/graphs/LocalFigure}
\label{Fig:ExpIConventionalLocal:a}
}
\subfigure[$(c_s, N_c, fs, m)=(\mbox{287 MB, 8, local, gridFTP})$]{
\includegraphics[scale=0.48]{data/graphs/ConventionalFigure}
\label{Fig:ExpIConventionalLocal:b}
}
\caption{\textit{Performance results for All-Pairs using gridFTP for
file reads being compared to a completely local run. We plot the time to
completion $t_c$ vs. the number of workers $N_w$. Note the scale, the
local case took less $t_c$ than the gridFTP case. In Fig.
\ref{Fig:ExpIConventionalLocal:a}, we note that by doubling the data set
size we doubled $t_c$, and that we created an overhead in S1 (vs. S0) by
increasing $t_x$ as a result of having more files. In Fig.
\ref{Fig:ExpIConventionalLocal:b}, the two-machine configurations took
more time than using a single machine. As expected, computing in one
machine, while having the data stored in another (S1), took longer
compared to having some data stored in the resource performing
computations $(S2)$. For up to 8 workers, $t_c$ decreased as $N_w$
increased, with the exception of a single machine where we reached an
I/O bound.}}
\label{Fig:ExpIConventionalLocal}
\end{center}
\end{figure}

%%\vspace{-0.2in}

%Bety's graph goes here
% \jhanote{We need data for compute (comparison) and I/O (only) for
% different data-set sizes}
\micnote{We need data for three and four
machines (just one graph going from 1 machine to four machines}

%Staging experiment
\subsection{Experiment II: Intelligence Based System}
The second experiment is similar to the first, except the All-Pairs
application is aware of the data location before determining whether or
not to assign a certain set of data-dependent computation to an idle
job. Inspired by earlier work~\citep{netperf}, this version of the
application performs an extra step during application startup that
approximates the performance of the network by pinging the hosts that
may be either a computational resource or a data store. This information
is then assembled into a graph data structure. This graph is utilised at
runtime when the application needs to map an idle worker to an
unprocessed set of pairs defined in the XML file. This changes the
first-available computational resource assignment mechanism described in
the first experiment to an intelligent based system. Though ping is not
very sophisticated in terms of describing a network's behavior, it is a
first-approximation to a model performance aware data-placement
strategy. We also experimented with the netperf application
\citep{netperf_web} as a method to describe a network's behavior, but
since we were working with such a static set of resources (LONI), the
same data graph was generated as with ping. The netperf-based
intelligent system took approximately 8 seconds longer per resource to
run due to the nature of the network evaluation, but yielded no benefit.
Netperf has the advantage of being able to determine throughput and
bandwidth over multiple protocols. These approaches know where the files
are located and their distance to available computational resources,
thus allowing more intelligent decisions when mapping a set of pairs to
a computational resource. An even more involved approach would be to
manage locations of files dynamically at run-time depending on usage
patterns. We leave this approach to future research. 
%Figure 2

\begin{figure}[!ht]
\begin{center}
   \includegraphics[scale=0.5] {data/graphs/IntelligentFigure}
\end{center}
\caption{\textit{$(c_s, N_c, fs, m)=(\mbox{287 MB, 8, local, gridFTP})$.
We compare the gridFTP and the intelligent approaches, for two and three
machines. In all cases, data was spread across all the resources For
the two-machine case, we observed a performance improvement by using our
intelligent approach. However, with more resources involved, our
intelligent system did not improve $t_c$.}}
\label{Fig:IntelligentExp}
\end{figure}


The overhead of intelligence includes the time spent pinging hosts and
building the graph data structure. The total time spent for this
overhead was negligible at approximately two seconds per application
run. In figure \ref{Fig:IntelligentExp}, we achieved a great reduction
in time to completion due to the use of intelligence. However, when the
same tests were performed utilising three resources instead of two, the
intelligence seemed to offer no significant reduction. The explanation
that we propose for this, is that the sets of pairs defined in the
configuration file at application start were geographically equally
dispersed throughout the network. Essentially, each set of pairs had
approximately the same cost calculated by using the graph data
structure. Each set of pairs would take the same time to compute using
any computational resource. % \jhanote{Perhaps define a test to verify
% this, so a note can go in saying we investigated this}

\subsection{Experiment III: CloudStore} 
The third experiment provides information into CloudStore's performance
in handling data locality issues. The same All-Pairs application as in
Experiments 1 and 2 is used, except all data is stored on the
distributed filesystem CloudStore under various configurations. Some
variables of importance include number of data servers that store data,
replication value for data in these data servers, and as above,
placement and number of computational resources. All read and writes
also utilise the distributed filesystem. Again, for our first set of
results, we don't add our comparison function, giving us
$t_{compute}=0$ 
%Figure 3
\begin{figure}
\begin{center}
\subfigure[One and two machines]
{
\includegraphics[scale=0.48]  {data/graphs/CloudStoreFigure}
\label{Fig:experiment3:a}
}
\subfigure[Three machines]
{
\includegraphics[scale=0.48] {data/graphs/CloudStore3Mach}
\label{Fig:experiment3:b}
}
\caption{\textit{$(c_s, N_c, fs, m) = (\mbox{287 MB, 8, CloudStore,
Direct})$. The figure on the left demonstrates All-Pairs' performance
with CloudStore locally and on two different machines. The figure on the
right demonstrates how this scales to three machines, for degree of
replication $r=1,2,\mbox{and } 3$. CloudStore performed better than our
local and intelligent approaches (see Figs.
\ref{Fig:ExpIConventionalLocal} and \ref{Fig:IntelligentExp}). Again,
having data in the resource with the workload decreased $t_c$. When data
was spread across all the computational resources, having a degree of
replication $r = 1, 2, \mbox{or } 3$ did not significantly decrease
$t_c$, except to the case of three machines and 2 workers.}}
%\jhanote{This caption needs attention}}
\label{Fig:experiment3}
\end{center}
\end{figure}


As done above for the first experiment, we attempted to capture how
compute time scales under these configurations, see figure
\ref{Fig:experiment3}. Having data remotely affected the performance. We
can see that computing in one resource while the data was on another
(S1) took the most amount of time. Having data in the resource that was
computing helped performance (S2, S3, S4). The number of machines to
which workload was assigned was also important. Placing workload on two
machines also decreased $t_c$ (S2 vs. S3). We varied the degree of
replication for the $C4$ and $C5$ configurations, i.e., for the cases of
two and three machines, where all the resources had workload assigned
and data stored. For $C4$, having $r = 2$ improved $t_c$, but not
considerably. For $C5$, different degrees of replication only made a
difference in the case of two workers.

We then added the actual genome comparison function and we compared it
to our base case where we did not the function . We defined $\Delta t_c$
which is the time difference between these two cases. In figure
\ref{Fig:experiment4}, we see that the time taken to do the genome
comparison was relatively small compared to the set up time, transfer,
and I/O time added together. The fact that $\Delta t_c$ for a given
$N_w$ was not the same for most of configurations, shows that there was
still an overhead; our $t_x$ and $t_{I/O}$ were not the same at the
times we run our All-pairs framework for both cases. \betynote{Guys,
does this statement seem ``true''?}
%Figure 4

\begin{figure}
\begin{center}
\includegraphics[scale=0.5]{data/graphs/CloudStoreComputeMinusNoCompute144}
\caption{\textit{$(c_s, N_c, fs, m) = (\mbox{144 MB, 8, CloudStore,
Direct})$. Comparison of CloudStore using the All-Pairs application with
and without actual computation. $\Delta t_c$ is defined as the time it
takes to run our framework where we include our genome comparison
function, minus the time it takes to run it when we don't include the
comparison function. For the case of $c_s=144MB$, our genome function
took about two orders of magnitude less than $t_x$ and $t_{I/O}$
combined. $\Delta t_c$ at a given $N_w$ differed for most of our
configurations, showing an overhead, probably caused by different
network conditions at the times our runs were performed.}}
\label{Fig:experiment4}
\end{center}
\end{figure}

We also compared our results with no comparison function
($t_{compute}=0$) for two different data set sizes, one with $c_s =
287MB$, and the other one with half the size, $c_s = 144MB$ (rounded
value). Both used CloudStore, and have eight assignments. We defined two
quantities, $\Delta t_c^d = t_c(c_s = 287MB) - t_c(c_s = 144MB)$, and
$t_{OH} = 2 \times t_c(c_s = 144MB) - t_c(c_s = 287MB)$. Figure
\ref{Fig:CloudStore287minus144} shows us that there are multiple factors
that can alter $t_c$. Some of the factors are network conditions, I/O
time, as well as transferring time that can be size dependent, disk seek
time, etc. In figure \ref{Fig:CloudStore287minus144:a}, we see that the
difference did not scale linearly with the number of workers. It is
worth noticing that $\Delta t_c$ was almost zero (and negative) for
eight workers when all the resources had workload and data stored (S3);
this is, it took about 10 seconds less for a 2.3GB set vs. a 1.15 GB
set. \betynote {Why?}. Figure \ref{Fig:CloudStore287minus144:b} shows
that for most of our cases, there was an overhead which decreased with
the number of workers. It also show us that the remote data
configuration was the one with the most overhead. Moreover, our S3 case
seemed to use a ``non scalable'' infrastructure, as the overhead
increased with eight workers.


\begin{figure}
\begin{center}
\subfigure[$\Delta t_c^d (= t_c(287MB) - t_c(144MB)) \times N_w$]
{
\includegraphics[scale=0.48]{data/graphs/CloudStoreNoCompute_287Minus144TimesNw}
\label{Fig:CloudStore287minus144:a}
}
\subfigure[$t_{OH}=2 \times t_c(144MB)-t_c( 287MB)$]
{
\includegraphics[scale=0.48]{data/graphs/CloudStoreNoCompute_287Minus144CommonTime}
\label{Fig:CloudStore287minus144:b}
}
%%\subfigure[$\Delta t_c$]
%%{
%%\includegraphics[scale=0.5]{data/graphs/CloudStoreNoCompute_287Minus144}
%%\label{Fig:CloudStore287minus144:c}
%%}
\caption{\textit{$(N_c, fs, m) = (\mbox{8, CloudStore, Direct})$. Here
we define two quantities, $\Delta t_c^d$, and $t_{OH}$. $\Delta t_c^d$
is the difference between $t_c$ found for data set sizes 1.15 GB and
2.3 GB, this is, for chunk sizes $c_s = 144\mbox{MB and } 287\mbox{MB}$.
$t_{OH}$ is the overhead time of working with chunks of 287MB in size
vs. 144MB chunks twice. In figure \ref{Fig:CloudStore287minus144:a} we
see that the difference decreased with the number of workers, but did
not scale linearly with $N_w$. In figure
\ref{Fig:CloudStore287minus144:b}, we see that S1 (remote data)
was the one with the most overhead.}}
\label{Fig:CloudStore287minus144}
\end{center}
\end{figure}

We compared the lowest times for both our intelligent approach and
CloudStore as a function of the number of resources $N_r$ in figure
\ref{Fig:CloudStoreVsGridFTP:a}. CloudStore performed better for one and
two machines, but not for three resources, when CloudStore decreased its
performance. The lowest times of our intelligent approach were about the
same for $N_r=1,2,3$. In figure \ref{Fig:CloudStoreVsGridFTP:b} we
plotted our completion time as a function of the number of workers for
three resources. All the resources had workload assigned and data
stored. CloudStore's performance did not vary significantly with the
number of workers, while our intelligent approach performed better as we
increased the number of resources from one to three.

\begin{figure}
\begin{center}
\subfigure[$(c_s, N_c) = (287 MB, 8)$]
{
\includegraphics[scale=0.48]{data/graphs/NumberResourcesFigure}
\label{Fig:CloudStoreVsGridFTP:a}
}
\subfigure[$(c_s, N_c, M_c) = (287 MB, 8, C5)$]
{
\includegraphics[scale=0.48]{data/graphs/CloudStoreVsGridFTPFigure}
\label{Fig:CloudStoreVsGridFTP:b}
}
\caption{\textit{Figure \ref{Fig:CloudStoreVsGridFTP:a}
 shows the lowest times we achieved for a given number of resources
 $(N_{r})$. These lowest times did not vary much for our intelligent
 method, while for CloudStore they improved by adding number of
 resources (up to three). Figure \ref{Fig:CloudStoreVsGridFTP:b}
 demonstrates performance with three resources for both gridFTP and
 CloudStore. The three resources computed and had data store.
 Performance using CloudStore remained about constant for two, four and
 eight workers, while our Intelligent approach improved its performance
 with the number of workers.}}
\label{Fig:CloudStoreVsGridFTP}
\end{center}
\end{figure}

%\section{Analysis}

Overall, the use of CloudStore decreases the time to completion
($t_c$) compared to the intelligent and the local approaches. Our
simple intelligent approach did not performed as well as CloudStore, but
works better than the local tests. For the parameters we used, the
introduction of more workers, up to 8 in our case, decreases the time of
completion; however, for the case of a single machine performing as both
master and workers, we hit an I/O bound, probably caused by the network.
For all of our approaches, GridFTP, Intelligent, and CloudStore, we find
the time to completion for three cases: a single machine is the master
and the workers, one machine does the computing while another has the
data stored, and the case when we spread the data into both machines,
while both of them compute.  In all of our three approaches, the case of
a single machine, shows the least $t_c$. In the case of two machines,
having the data in one, and computing in the other one, increases the
time to completion. By splitting the data in both of the machines, and
computing in both, we decrease $t_c$. We then added another case, all
the data in each of machines, and computing in both. This decreases the
time even further, but not to the point of the local computations;
perhaps, all the approaches are not guaranteed to use the data in the
machine the job is being run, even thought the data is in both machines.

???For our three cases,  $t_c$ may be bounded, and decrease to $t_c$(single machine) as  the number of workers ($N_w$) increases, up to a critical $N_w$, after which $T_c$ will increase due to worker coordination overhead.





\section{Analysis}

Overall, the use of CloudStore decreases the time to completion
($t_c$) compared to the intelligent and the local approaches. Our
simple intelligent approach did not performed as well as CloudStore, but
works better than the local tests. For the parameters we used, the
introduction of more workers, up to 8 in our case, decreases the time of
completion; however, for the case of a single machine performing as both
master and workers, we hit an I/O bound, probably caused by the network.
For all of our approaches, gridFTP, Intelligent, and CloudStore, we find
the time to completion for three cases: a single machine is the master
and the workers, one machine does the computing while another has the
data stored, and the case when we spread the data into both machines,
while both of them compute. In all of our three approaches, the case of
a single machine, shows the least $t_c$. In the case of two machines,
having the data in one, and computing in the other one, increases the
time to completion. By splitting the data in both of the machines, and
computing in both, we decrease $t_c$. We then added another case, all
the data in each of machines, and computing in both. This decreases the
time even further, but not to the point of the local computations;
perhaps, all the approaches are not guaranteed to use the data in the
machine the job is being run, even thought the data is in both machines.

For our three cases, $t_c$ may be bounded, and decrease to $t_c$(single
machine) as the number of workers ($N_w$) increases, up to a critical
$N_w$, after which $T_c$ will increase due to worker coordination
overhead.

\section{Conclusion} Our results show that CloudStore greatly changes the
performance of a distributed application in a positive manner. Our
experiments that utilised this DFS to access and store data outperformed
their gridFTP counterparts by an order of magnitude in most cases. Our
results also indicate that CloudStore scales better as file sizes and
number of files grow, although both seem to scale linearly. Before any
conclusions may be drawn, there are issues that need to be addressed.
Our application utilised SAGA to access CloudStore based and gridFTP based
files. There is an overhead that SAGA introduces. In addition, we were
unable to utilise our entire distributed system, using at most 8 jobs to
handle our work. With a replication level of two in the DFS, data was
almost certainly co-located with the computational resource. In the
second experiment, utilizing information from first staging the network
did improve upon the results of the naive first experiment, but still
did not approach the DFS's performance levels.

Distributed filesystems are important abstractions for a data-intensive
distributed application developers to consider. It also appears that
staging is worth the time required to build a graph representing the
network. Also to note, the second experiment is also naive in the way
that it attempts to optimise data and work assignments. Our staging
only performed pings, not data transfer trials or reliability tests. A
job could have low latency, but poor bandwidth. Perhaps CloudStore's
performance can be attributed to recent work that has shown that data in
large scale distributed applications tends to be accessed together,
despite being seemingly unrelated in the input data-set. Such
correlation in data-access has been observed elsewhere, and specific
abstractions to support the access of ``aggregation of such files'' has
been referred to as a filecule, an application specific group of
files~\cite{filecule}. Attempting to determine if analogous
abstractions could enhance performance for the All-Pair application
could be interesting. In a DFS, however, if the data store is also
capable of data processing, then the DFS is placing commonly used files
together on machines needing them for work; in essence, the DFS is
finding these groups for the developer. The fault tolerance, for which
distributed filesystems are already well renowned for, also has added
benefits to grid application developers in terms of performance. The
distributed application does not have to be aware of where data has been
copied to previously when assigning work; the distributed filesystem
uses the best replica when data is being accessed.

{\bf Acknowledgment:} Important funding for SAGA has been provided by
the UK EPSRC grant number GR/D0766171/1 (via OMII-UK) and HPCOPS
NSF-OCI 0710874. SJ acknowledges the e-Science Institute, Edinburgh
for supporting the research theme, ``Distributed Programming
Abstractions'' and theme members for shaping many important
ideas. This work has also been made possible thanks to the internal
resources of the Center for Computation \& Technology at Louisiana
State University and computer resources provided by LONI. 

%\bibliographystyle{IEEEtran}
\bibliographystyle{kluwer} 
\bibliography{data_intensive_paper}
\end{document}

%Therefore, we introduce the idea of network-closeness. A network-close
%data-set takes a small amount of time to transfer to the work location.
%A network-far data-set is just the opposite. A network-far data-set
%takes a long time to transfer to the work location. If there is an
%unprocessed data-set collocated or network-close with the job, then the
%assignment of that worker to that data-set would have benefits. If there
%is no unprocessed data-set that is network-close to the job, still we
%assign data that may be network-far, in case the network-close job
%failed or there is no available jobs network-close to the data-set.

%There are at least two types of data-intensive applications: the first
%where the actual data generated is large; the second type is where the
%data generated is small, but the volume of data on which computation
%occurs is very large. The application we used, has relatively small
%input and relatively small output, but the manner of processing causes
%many data reads. This type of application can be classified as having
%a large data throughput. \jhanote{Can you elaborate on different
%types of data-intensive applications? What kind is an ImageMagic
%based application?}
