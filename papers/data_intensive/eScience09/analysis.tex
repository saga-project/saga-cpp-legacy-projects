\section{Analysis}

Overall, the use of CloudStore decreases the time to completion
($t_c$) compared to the intelligent and the local approaches. Our
simple intelligent approach did not performed as well as CloudStore, but
works better than the local tests. For the parameters we used, the
introduction of more workers, up to 8 in our case, decreases the time of
completion; however, for the case of a single machine performing as both
master and workers, we hit an I/O bound, probably caused by the network.
For all of our approaches, GridFTP, Intelligent, and CloudStore, we find
the time to completion for three cases: a single machine is the master
and the workers, one machine does the computing while another has the
data stored, and the case when we spread the data into both machines,
while both of them compute.  In all of our three approaches, the case of
a single machine, shows the least $t_c$. In the case of two machines,
having the data in one, and computing in the other one, increases the
time to completion. By splitting the data in both of the machines, and
computing in both, we decrease $t_c$. We then added another case, all
the data in each of machines, and computing in both. This decreases the
time even further, but not to the point of the local computations;
perhaps, all the approaches are not guaranteed to use the data in the
machine the job is being run, even thought the data is in both machines.

???For our three cases,  $t_c$ may be bounded, and decrease to $t_c$(single machine) as  the number of workers ($N_w$) increases, up to a critical $N_w$, after which $T_c$ will increase due to worker coordination overhead.



