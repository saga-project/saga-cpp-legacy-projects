%%%%%%%%%%%%%%%
% RSGUIDE.TEX %
%%%%%%%%%%%%%%%

% Guide to preparing TeX articles for Royal
% Society articles using RSPUBLIC.CLS
% Use this file as a test file

\documentclass{rspublic}

\begin{document}

\title[A guide to editing \TeX\ articles]{A guide to
editing \TeX\ articles for\\ Royal Society journals}

\author[J. Wainwright]{Jonathan Wainwright}

\affiliation{\TandT\ Productions Ltd, London, UK}

\label{firstpage}

\maketitle

\begin{abstract}{Place keywords here}
A brief guide is presented for preparing articles for submission to the Royal Society
Series A journals.
\end{abstract}

\section{Introduction}

This guide is intended to help authors preparing articles for submission to either of
the two A-side journals, \textit{Proceedings} and \textit{Philosophical Transactions}.
The guide outlines the basic structure of the \LaTeX\ file, and also provides some of
the elements of house style and preferred solutions to frequently encountered
typographical problems. It is \textit{not} intended as a introduction to the \TeX\
typesetting system; authors should consult Lamport and Knuth for this.

\section{The preamble}

Every \LaTeX\ file begins with a \verb'\documentclass' command:
\begin{verbatim}
   \documentclass{rspublic}
\end{verbatim}

Local macro definitions may be included next, before the \verb'\begin{document}'
command, although it is advisable to keep all such definitions to a minimum for
simplicity.

\section{The title page}

Every title page displays the article's full title, its authors and their full, postal
addresses, and the abstract. There may also be a present address, if different to the
place where the work was done.

After the preamble the general form is:
\begin{verbatim}
   \begin{document}
   \title[Short title]{Full title}
   \author[Short author names]{Full author names}
   \affiliation{Full postal addresses of all authors}
   \label{firstpage}
   \maketitle
\end{verbatim}

\bi
\item The short title, which is usually less than around 40
characters, is used as the recto running head.

\item The short author names comprise initials and surnames, or, if
these total more than around 40 characters, the first author followed
by \verb'and others'. This argument is used as the verso running head.

\item The full author names, if given, are retained on the title
page.

\item Give the name of any country in English (e.g. Republic of
Ireland not Eire).

\item The only allowed abbreviations are USA and UK (don't use Scotland, Wales,
etc.). Other examples are: People's Republic of China; Taiwan, Republic of China.

\item There is no comma between town and postcode, or between state
and zip code.

\item Abbreviate American states, i.e. CA for California, etc.

\item If authors have different addresses, then superscript arabic
numerals are used as tags.

\item Only present addresses appear as footnotes to authors on
the title page. References to grants received, etc., should
appear in the acknowledgements.
\ei

The abstract follows immediately after the affiliation, and appears
within the \verb'abstract' environment:
\begin{verbatim}
   \label{firstpage}
   \maketitle
   \begin{abstract}{Keywords}
   :
   \end{abstract}
\end{verbatim}
Note that the environment takes as an argument the keywords of the paper.

\section{Structure of the remaining article}

After the title page and abstract there is usually an introduction,
and several more sections before the list of references. There may
also be an acknowledgements and one or more appendixes. The most
general structure is:
\begin{verbatim}
   \section{}
   :
   \subsection{}
   :
   \subsubsection{}
   :
   \begin{acknowledgements}
   :
   \end{acknowledgements}
   :
   \appendix{}
   :
   \subsection{}
   :
   \subsubsection{}
   :
   \begin{thebibliography}
   :
   \end{thebibliography}
   \label{lastpage}
   \end{document}
\end{verbatim}

Some of these elements are optional and some will usually be repeated; for example,
almost every paper comprises several sections, but relatively few contain an appendix.

\section{Floating bodies: figures and tables}

The \verb'figure' and \verb'table' environments are implemented
as described in the \LaTeX\ manual to provide consecutively
numbered floating inserts.

\subsection{Figures}

The usual figure arrangement is:
\begin{verbatim}
   \begin{figure}
   <PostScript inclusion special>
   \caption{}
   \end{figure}
\end{verbatim}

Authors who cannot include PostScript figures should omit the inclusion line.

\subsection{Tables}

The basic structure of all tables is:
\begin{verbatim}
   \begin{table}
   \caption{<title>}
   \longcaption{<long description>}
   \begin{tabular}{<column alignments}
   \hline
   <headings>\\
   \hline
   <body of table>
   \hline
   \end{tabular}
   \end{table}
\end{verbatim}

\bi
\item The title should be short, like that of a section heading;
there is no full stop at the end. Any long description should be
separated from the title and included in the \verb'\longcaption'
command, which sets the argument in parentheses and as a paragraph
below the title.

\item Headings are usually centred, and are not capitalized.
\ei

\section{Theorem and proof environments}

Theorem-like environments---including theorem, lemma, corollary, and proposition
---should not be redefined, but used as follows:
\begin{verbatim}
  \begin{theorem/lemma/corollary/proposition}
  :
  \end{theorem/lemma/corollary/proposition}
\end{verbatim}

\begin{theorem}
Let $E$ be an orbifold  $C^{\ast}$-bundle over the closed ${\rm spin}^c$
orbifold $Q$ and let $D$ be the generalized Dirac operator on $Q$ with
coefficients in $E$.
\end{theorem}

The proof should be included within the following environment:
\begin{verbatim}
  \begin{proof}
  :
  \end{proof}
\end{verbatim}

\begin{proof}
Use $K_\lambda \geq S_\lambda$ to translate combinators into
$\lambda$-terms.  For the converse, translate $\lambda x$ \ldots
by [$x \leq y$] \ldots and use induction and the lemma.
\end{proof}

\section{Elements of mathematical editing}

The following commonly used macros are defined at the end of \verb'rspublic.cls' and
should be used in math mode whenever the entity in parentheses is required:
\bi
\item \verb'\rd' (differential)
\item \verb'\re' (exponential)
\item \verb'\ri' (imaginary number, $\ri=\surd-1$)
\item \verb'\Real' (real part)
\item \verb'\Imag' (imaginary part)
\item \verb'\sgn' (sign)
\item \verb'\const' (constant)
\ei For example, $\re^{-\ri kr\cos s}\,\rd s.$

\subsection{Equations}

All equation environments should be compatible with those defined by the \verb'amsmath'
package, which is a required package for the class file.

\section{Units and symbols for units}

The International System of Units (SI) is used. Six base units are given in table~1;
two supplementary units are given in table~2; and fifteen derived units are given in
table~3.

\bi
\item Symbols for units should be in roman type in all contexts and
separated from each other and from the numeral by a fixed thin space,
given by the macro \verb'\3'.

\item SI prefixes, given in table 4, precede the SI
unit: `mg', `ns', `$\mu$A', `MHz', `kPa'. Note there is no space
separating the prefix from the unit.

\item The sloping letter $\mu$ for the prefix `micro-' will be replaced by the upright letter
during typesetting.

\item Note the preferred spelling of certain units: `metre',
`millimetre', `gram', `kilogram', etc.
\ei

\begin{table}
\label{baseunits}
\caption{Names and symbols for the SI base units}
\begin{tabular}{lll}
\hline
\qquad physical quantity & \multicolumn{1}{c}{name} &
\multicolumn{1}{c}{symbol} \\
\hline
length & metre & m\\
mass & kilogram & kg\\
time & second & s\\
electric current & ampere & A\\
thermodynamic temperature & kelvin & K\\
amount of substance & mole & mol\\
\hline
\end{tabular}
\end{table}

\begin{table}
\label{suppunits}
\caption{Names and symbols for the SI supplementary units}
\begin{tabular}{lll}
\hline
physical quantity & \multicolumn{1}{c}{name} &
\multicolumn{1}{c}{symbol} \\
\hline
plane angle & radian & rad\\
solid angle & steradian & sr\\
\hline
\end{tabular}
\end{table}

\begin{table}
\label{derivedunits}
\caption{Names and symbols for SI derived units}
\begin{tabular}{llll}
\hline
\qquad physical quantity & \multicolumn{1}{c}{name} &
\multicolumn{1}{c}{symbol} & \multicolumn{1}{c}{definition}\\
\hline
energy & joule & J & m$^2$\3kg\3s$^{-2}$\\
force & newton & N & m\3kg\3s$^{-2}$\\
pressure & pascal & Pa & m$^{-1}$\3kg\3s$^{-2}$\\
power & watt & W & m$^2$\3kg\3s$^{-3}$\\
electric charge & coulomb & C & A\3s\\
electric potential &&&\\
\quad difference & volt & V & m$^2$\3kg\3s$^{-3}$\3A$^{-1}$\\
electric resistance & ohm & $\Omega$ & m$^2$\3kg\3s$^{-3}$\3A$^{-2}$\\ electric
conductance & siemens & S & m$^{-2}$\3kg$^{-1}$\3s$^3$\3A$^2$\\ electric capacitance &
farad & F & m$^{-2}$\3kg$^{-1}$\3s$^4$\3A$^2$\\ magnetic flux & weber & Wb &
m$^2$\3kg\3s$^{-2}$\3A$^{-1}$\\ inductance & henry & H &
m$^2$\3kg\3s$^{-2}$\3A$^{-2}$\\ magnetic flux density & tesla & T &
kg\3s$^{-2}$\3A$^{-1}$\\ frequency & hertz & Hz & s$^{-1}$\\ activity & bequerel & Bq &
s$^{-1}$\\ absorbed dose & gray & Gy & J\3kg$^{-1}$\\
\hline
\end{tabular}
\end{table}
\begin{table}
\label{SIprefixes}
\caption{SI prefixes}
\begin{tabular}{llllll}
\hline
multiple & prefix & symbol & multiple & prefix & symbol\\
\hline
$10^{-1}$ & deci & d & 10 & deca & da\\
$10^{-2}$ & centi & c & $10^2$ & hecto & h\\
$10^{-3}$ & milli & m & $10^3$ & kilo & k\\
$10^{-6}$ & micro & $\mu$ & $10^6$ & mega & M\\
$10^{-9}$ & nano & n & $10^9$ & giga & G\\
$10^{-12}$ & pico & p & $10^{12}$ & tera & T\\
$10^{-15}$ & femto & f & $10^{15}$ & peta & P\\
$10^{-18}$ & atto & a & $10^{18}$ & exa & E\\
\hline
\end{tabular}
\end{table}

\section{Elements of copy-editing}

\subsection{Spelling, capitalization and abbreviations}

\bi
\item Use English spelling throughout (follow the \textit{OED} or \textit{The Collins English Dictionary}): e.g.
\begin{center}
\begin{tabular}{llll}
ageing & analogue & analyse & behaviour
\\
centre & centred & characterize & crystallize
\\
favour & focused & focusing & formulas
\\
generalize & modelled & modelling & neighbour
\\
parametrize & polarize & realize & recognize
\\
vapour & vaporize
\end{tabular}
\end{center}
Note the common endings: \textit{-ize} and \textit{-yse} rather than \textit{-ise} and
\textit{-yze}.

\item Remove spurious capitalization; for example, in the
text use `figure~1', `table~2', `equation~(3.4)', `theorem~5.6',
`preposition~7.8', `the appendix', i.e. no capitals (except, of
course, when they start a sentence). But `Appendix~A', i.e. when a
specific section is referenced. Also note that these words should
always be given in full.

\item Replace `section' by the symbol (\S), except when it begins a
sentence. The fourth subsubsection in the third subsection in the second section is
referenced by \verb'\S2$\,c\,$(iv)' produces \S2$\,c\,$(iv).

\item Use capitals for words derived from names---Cartesian, Gaussian,
Hamiltonian, Abelian, etc.---except for the following: boson,
fermion, ohmic, voltaic, coulombic.

\item For trade names (e.g.\ Teflon) follow Chambers.

\item
Full caps should be used for all acronyms, which should be defined on first use (e.g.
face-centred cubic (FCC)). Since the abstracts are published separately, this rule
applies to both the abstracts and the main body of the text.

\item Common abbreviations are: `cf.', `e.g.', `etc.', `i.e.' Note
that `etc.' is always preceded and followed by a comma except where
it ends a sentence.
\ei

\subsection{Em rules and en rules}

\bi
\item A pair of em rules (or dashes) is used to indicate asides and
parantheses, in a way similar to commas, but forming a more distinct
break. (Commas are preferred for short parenthetical remarks.) No
spaces should be put between the em rules and their associated
asides.

\item An en rule has the following uses:

\quad to indicate a range of numbers, e.g. `20--100\3keV' (but note
the expression, `from 20 to 100\3keV'; avoid `from 20--100\3keV');

\quad between interactions, e.g. photon--photon;

\quad with log--log (often seen when describing the axes of a
figure);

\quad between linked names, e.g. Hartree--Fock.

\item An en rule should not be used in `between 20--100\3keV';
`between 20 and 100\3keV' is the preferred form.
\ei

\subsection{Miscellaneous details}

\bi
\item Dates are expressed in the form, `day month year': e.g. 9 April
1995.

\item The plural \textit{data} is far more common than its singular
form, and should always be treated as a plural, i.e. `data \textit{are}\ldots' is
correct while `data is\ldots' is incorrect. Some words, on the other hand, appear to be
plural but are treated as singular: dynamics, statistics, mechanics.

\item Single and not double quotes should be used.

\item Retain, or put in if missing, the accents for proper names such
as Schr\"{o}dinger, Poincar\'e, etc.
\ei

\section{References}

\subsection{In the text}

\bi
\item Two authors are always connected by an ampersand:

\quad The work of Smith \& Jones (1995) remains important in the
field\ldots

\item And three or more authors are abbreviated as follows:

\quad The work of Smith \textit{et al.} (1995) remains important in the
field\ldots

The year always goes in parentheses, and immediately follows
the authors. If the whole citation is parenthetical, then both
authors and year are contained within parentheses, with no comma
before the year:

\quad Detailed studies (Smith \& Jones 1995) show\ldots

\item Additional labelling is used to ensure unambiguous citation:

\quad The work of Smith (1995$a,b$) \ldots\

\quad Important results (Smith 1995$c$)\ldots

\item Multiple citations within parentheses are separated by
semicolons, except where the citation is to a work by the same
author(s), when the years are separated by commas:

\quad Detailed studies (Smith 1989; Smith \& Jones 1994, 1995)
show\ldots
\ei

\subsection{In the list}

\bi
\item The list of references appears at the end of the article and is
contained within the environment:
\begin{verbatim}
       \begin{thebibliography}{}
       :
       \end{thebibliography}
\end{verbatim}
Each reference is begun with an \verb'\item', and all contain at
least one name and a year:

\begin{thedemobiblio}{}
\item
Adams, H. D. 1979 \ldots
\item
Adams, H. D. 1980 \ldots
\item
Adams, H. D. \& Brown, I. 1981 \ldots
\item
Adams, H. D. \& Charles Jr, A. L. M. 1978 \ldots
\item
Adams, H. D., Charles Jr, A. L. M. \& Brown, I. 1989 \ldots\hfill[1]
\item
Adams, H. D., Brown, I. \& Charles Jr, A. L. M. 1990 \ldots\hfill[2]
\item
Brown, I. (ed.) 1989 \ldots\hfill[3]
\item
Brown, I. \& Charles Jr, A. L. M. (eds) 1990 \ldots\hfill[4]
\item
Brown, I. \& Charles Jr, A. L. M. 1991 \ldots\hfill[5]
\end{thedemobiblio}\smallskip

\item The list is alphabetical, except for works of three or more
authors sharing the same initial author, which are arranged chronologically. The logic
behind using chronological instead of alphabetical ordering in lines [1] and [2] is
that the citations in the text only refer to the first author and the year (`Adams
\textit{et al.} 1989, 1990'), and not to the second and third authors.

\item Each author's name comprises the surname followed by initials
(spaced), with commas separating surnames and initials.

\item The abbreviations for `editor' and `editors' are `ed.' and
`eds', respectively, in lines [3] and [4].

\item The abbreviation for `Junior' is `Jr', and is placed as in line
[5].

\item Note the lack of punctuation or parentheses around the year.

\item If there are ten or more authors, then only the first author
need be included:

\begin{thedemobiblio}\smallskip{}
\item
Adams, H. D., \textit{et al.} 1989 \ldots
\end{thedemobiblio}\smallskip

\item Labels `$a$', `$b$', etc., are used to ensure unambiguous
citation:

\begin{thedemobiblio}\smallskip{}
\item
Adams, H. D. 1989$a$ \ldots
\item
Adams, H. D. 1989$b$ \ldots
\end{thedemobiblio}\smallskip

The simplest coding is \verb'1989$a$', etc.
\ei

This structure is common to all entries in the list. The
remaining bibliographic details follow the year and can take a
variety of forms.

\subsubsection{Journals}

\bi
\item The title of the article follows the year, is set in roman
and lower case, and is ended with a full stop.

\item The journal is abbreviated (see Appendix~B) and set in
italics.

\item The volume is set in bold roman, followed by a comma and the
page range, with the pages joined with an en-rule.

\begin{thedemobiblio}\smallskip{}
\item
Pironneau, O. 1973 On optimum profiles in Stokes flow. \textit{J. Fluid Mech.}
\textbf{59}, 117--128.
\end{thedemobiblio}\smallskip

\item Some journals have an additional letter in their title; most of
these precede the volume number, but the letters of two journals \textit{follow} the
volume number:

\begin{thedemobiblio}\smallskip{}
\item
\ldots \textit{Phys. Rev.} B\,\textbf{59}, 117--128.
\item
\ldots \textit{Proc. R. Soc. Lond.} A\,\textbf{59}, 117--128.
\item
\ldots \textit{Physica} \textbf{59A}, 117--128.
\item
\ldots \textit{Phys. Lett.} \textbf{59A}, 11\,789--12\,108.
\end{thedemobiblio}\smallskip

Note that a thinspace \verb'\,' separates the letter and volume
number when the letter comes first, and is also used in five- or
higher-digit numbers, e.g. 11\,789.
\ei

\subsubsection{Books}

\bi
\item The title of the book follows the year, is set in italics
and lower case, and is ended with a full stop, except where there is
a page, chapter, volume, or edition number.

\item The place of publication and the publisher completes the entry.

\begin{thedemobiblio}\smallskip{}
\item
Gakhov, F. D. 1966 \textit{Boundary value problems}, ch.~3, pp.~45--47. Oxford:
Pergamon Press.
\item
Titchmarsh, E. C. 1986 \textit{Introduction to the theory of Fourier integrals}, 3rd
edn. New York: Chelsea.
\end{thedemobiblio}\smallskip

\item The title of a contribution to a multi-author work is set in
roman and lower case, and ended with a full stop. This is followed by
`In' and the title of the work in italics and lower case, followed by
 the editor in parentheses and the page range.

\begin{thedemobiblio}\smallskip{}
\item
Gakhov, F. D. 1966 Boundary value problems. In \textit{Introduction to the theory of
Fourier integrals} (ed. E. C. Titchmarsh), pp.~30--56, 3rd edn. New York: Chelsea.
\end{thedemobiblio}\smallskip

\item References to contributions to conference proceedings are dealt
with in a similar way.

\begin{thedemobiblio}\smallskip{}
\item
Kaplun, S. 1993 Low Reynolds number flow past a circular cylinder. In \textit{Proc.
Int. Conf. on Fluid Mechanics and Singular Perturbations, Tokyo, Japan, 12 October
1992}, pp.~67--85.
\end{thedemobiblio}\smallskip

\item Note that the conference title is in italics and retains the
capitals, and includes the place and date.

\item Monographs and named series:

\begin{thedemobiblio}\smallskip{}
\item
Berry, M. V. 1986 Riemann's zeta function: a model for quantum chaos? In
\textit{Quantum chaos and statistical nuclear physics} (ed. T. H. Seligman \& H.
Nishioka). Springer Lecture Notes in Physics, no. 263, pp. 1--17.
\item
Craik, A. D. D. 1985 \textit{Wave interactions and fluid flows.} Monographs on
mechanics and applied mathematics. Cambridge University Press.
\end{thedemobiblio}\smallskip
\ei

\subsubsection{Reports, memoranda, preprints, theses, etc.}

\bi
\item These are set as follows, using roman and lower case throughout
for titles:

\begin{thedemobiblio}\smallskip{}
\item
Kaplun, S. 1993 Low Reynolds number flow past a circular cylinder.
Report no. 345-12, Department of Engineering, University of London.
\item
Li, T. Y.  \& Yorke, J. A.  1978 Ergodic transformations from an
interval into itself. Preprint, New York University.
\item
Ogorzalek, M. J.  1993 Taming chaos. Ph.D. thesis, Imperial College,
London.
\item
Ott, E.  1993 Chaos in dynamical systems. D.Phil. thesis, Oxford University.
\item
Saito, T.  1990 An approach toward higher-dimensional hysteresis
chaos generators. Memorandum no. UCB/ERL M94/42, University of
California at Berkeley, USA.
\end{thedemobiblio}\smallskip
\ei

\subsection{Miscellaneous}

\bi
\item Translations are included in parentheses:

\begin{thedemobiblio}\smallskip{}
\item
Zabolotskaya, E. A. 1970 \textit{Akust. Zh.} (Transl. \textit{Soviet Phys. Acoust.}
\textbf{16}, 39--43.)
\end{thedemobiblio}\smallskip

\item Personal communications are \textit{not} included in the list; the
citation in the text is the only reference: (L. M. Pecora 1989,
personal communication).

\item Unpublished work is also \textit{not} included in the list; the
citation in the text is the only reference: (L. M. Pecora 1989,
unpublished work).
\ei

\end{document}
