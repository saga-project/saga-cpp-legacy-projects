\documentclass[]{paper}
\usepackage{graphicx}
\usepackage{color}

% type user-defined commands here
\newif\ifdraft
\drafttrue
\ifdraft
\newcommand{\onote}[1]{ {\textcolor{cyan} { (***Ole: #1) }}}
\newcommand{\terminology}[1]{ {\textcolor{red} {(Terminology used: \textbf{#1}) }}}
\newcommand{\owave}[1]{ {\cyanuwave{#1}}}
\newcommand{\jwave}[1]{ {\reduwave{#1}}}
\newcommand{\alwave}[1]{ {\blueuwave{#1}}}
\newcommand{\jhanote}[1]{ {\textcolor{red} { ***shantenu: #1 }}}
\newcommand{\alnote}[1]{ {\textcolor{green} { ***andreL: #1 }}}
\newcommand{\amnote}[1]{ {\textcolor{blue} { ***andreM: #1 }}}
\newcommand{\smnote}[1]{ {\textcolor{brown} { ***sharath: #1 }}}
\newcommand{\pmnote}[1]{ {\textcolor{brown} { ***Pradeep: #1 }}}
\newcommand{\msnote}[1]{ {\textcolor{cyan} { ***mark: #1 }}}
\newcommand{\note}[1]{ {\textcolor{magenta} { ***Note: #1 }}}
\else
\newcommand{\onote}[1]{}
\newcommand{\terminology}[1]{}
\newcommand{\owave}[1]{#1}
\newcommand{\jwave}[1]{#1}
\newcommand{\alnote}[1]{}
\newcommand{\amnote}[1]{}
\newcommand{\athotanote}[1]{}
\newcommand{\smnote}[1]{}
\newcommand{\pmnote}[1]{}
\newcommand{\jhanote}[1]{}
\newcommand{\msnote}[1]{}
\newcommand{\note}[1]{}
\fi

\begin{document}

\title{FutureGrid 2012 Project Challenge: Interoperable and
  Standards-based Distributed Cyberinfrastructure and Applications}
 
\author{Andre Luckow 
  \and Andre Merzky
  \and Mark Santcroos
  \and Ole Weidner 
  \and Shantenu Jha
}
\date{May 15th, 2012}
\maketitle

\begin{abstract}
\end{abstract}

\section{Introduction}

FutureGrid provides researchers with new possibilities to engage in
science relating to the state-of-the-art in cloud and grid
computing. As members of the RADICAL group, we have taken full
advantage of the opportunities that FutureGrid provides. Here are some
of the ways we are using FutureGrid resources to push the envelope and
pursue exciting new discoveries.

\jhanote{Topics to be covered: (i) P* (AL) (ii) CSA ? (AM) (iii)
  OGF-Standards based development and testing (AM) (iv) Class Project
  (SJ) (v) Others?}

\section{P* - Pilot-Job Interoperability}

Pilot-Jobs (PJ) have become one of the most successful abstractions in
distributed computing. In spite of extensive uptake, there does not exist a
well defined, unifying conceptual model of Pilot-Jobs, which can be used to
define, compare and contrast PJ implementations. This presents a barrier to
extensibility and interoperability. The P*
model~\cite{pstar-2012,pstar-sc-2012} provide a minimal but complete model of
Pilot-Jobs, which has been successfully applied to different Pilot-Job
frameworks, e.\,g.\ Condor and DIANE. The Pilot-API~\cite{pilot_api} provides 
an abstract, unified interface to PJ frameworks that adhere to the P* Model.

To demonstrated the interoperable and concurrent use of multiple Pilot-Job
frameworks via the Pilot-API on different production and research
infrastructures. Figure~\ref{fig:perf_perf-bfast-bj} shows how the Pilot-API
enables the user to run applications interoperably on different production and
research infrastructures. For this purpose we investigate the performance of
BFAST, a genome sequencing application. BFAST is very I/O sensitive -- we
observed for example, an I/O bottleneck if many BFAST CUs are run on the same
shared file system. The Pilot-API enables applications to scale to different
infrastructures in such cases.

\begin{figure}[t]
\centering
\includegraphics[width=0.7\textwidth]{figures/128-bfast-egi-fg-xsede-osg.pdf}
\caption{\textbf{PJ Framework Performance on XSEDE, FutureGrid, EGI and 
  OSG:} Running 128 BFAST match tasks on 128 cores. The longer runtimes on EGI 
  and OSG are mainly caused by  longer queuing times and the necessity to      
  stage all input files.}
  \label{fig:perf_perf-bfast-bj}
\end{figure}

\section{BigJob and BigData}

FutureGrid has been an important testbed for the development of
BigJob~\cite{saga_bigjob_condor_cloud} and
BigData~\cite{pstar-sc-2012,pmr-2012}. BigJob (BJ) is a SAGA-based Pilot-Job
(PJ) framework that implements the Pilot-API. BJ has been designed to be
general-purpose and extensible. While BJ has been originally built for HPC
infrastructures, such as FutureGrid and XSEDE, it is generally also usable in
other environments, such as OSG. This extensibility mainly arises from the
usage of SAGA as a common API for accessing distributed resources.

\begin{figure}[t]
	\centering
	\includegraphics[width=0.7\textwidth]{figures/re_bigjob_interactions.pdf}
	\caption{\textbf{BigJob Architecture:} The
	          BJ architecture resembles many elements of the P* Model. The
	          BigJob-Manager is the central Pilot-Manager, which
	          orchestrates a set of Pilots. Each Pilot is represented by a
	          decentral component referred to as the BigJob-Agent.}
			\label{fig:figures_re_bigjob_interactions}
\end{figure}

Figure~\ref{fig:figures_re_bigjob_interactions} illustrates the BJ
architecture: The BJ-Manager is the Pilot-Manager responsible for coordinating
the different components of the frameworks. The BigJob-Agent is the actual
Pilot that is submitted to a resource. BigData extends the Pilot-Job concept
to data. BigData provides late-binding capabilities for data by separating the
storage allocation and application-level~\cite{pstar-2012}. Similar to BigJob,
it is comprised of two components: the BD-Manager and the BD-Agents, which are
deployed on the physical resources.

\begin{figure}[t] 
\centering 
\includegraphics[width=0.8\textwidth]{figures/bigjob-varying-cores-alamo-noadvert.pdf}
\caption{\textbf{Pilot-Job Coordination Mechanism:}  The runtime of a
workload of 4 tasks per core, i.\,e.\ 32 - 512\,tasks, using different Pilots
and configuration. For BJ-Redis the runtime  increases only moderately, the
client-server-based implementations BJ-ZMQ and CORBA-based DIANE show
particularly a steep increase when going from 64 to 128 cores.}
\label{fig:perf_bigjob-varying-cores} 
\end{figure}

We used FutureGrid to evaluate the overheads typical associated with PJ
frameworks. For the evaluation of the communication \& coordination (c\&c)
subsystem of BigJob and DIANE we run a different number of very short running
(i.\,e. zero workload) tasks on Alamo/FG concurrently. In general, the c\&c
systems used are mostly insensitive to the number of coordinated tasks.

Figure~\ref{fig:perf_bigjob-varying-cores} illustrates the scalability of BJ
and DIANE with respect to the number of cores and tasks managed by Pilot. For
this purpose, we execute 4\,tasks per core, i.\,e.\ between 32 and 512\,tasks.
BigJob with Redis (local) shows almost linear scaling up to 128 cores. BigJob
with Redis (remote) imposes an increase of about 14\,\%. BigJob with ZeroMQ
performs very well with lower core counts; with larger core counts, the
runtimes increase, indicating a potential scalability bottleneck. Due to
higher startup overhead, at lower core counts DIANE shows a longer runtime
than ZeroMQ or Redis. At higher core counts DIANE behaves similar to
BigJob/ZeroMQ, but shows a greater increase in the overall runtime. This
increase is likely attributable to the single central manager in DIANE's
CORBA-based client-server architecture. Using Redis as central data space for
BigJob decouples Pilot-Manager and Agent, yielding better performance in
particular with many replicas.



\section{Replica Exchange}

Various applications have been developed using Pilot Abstractions and BigJob.
An important application class are those based on the Replica-Exchange
algorithm. Replica-Exchange (RE) are used to understand physical phenomena –
ranging from protein folding dynamics to binding affinity calculations. RE
methods represent a class of algorithms that involve a large number of
loosely-coupled ensembles. We develop a framework for RE that supports
different replica-pairing and coordination mechanisms, that can utilize a
range of production cyberinfrastructure concurrently. We use this framework to
implement three different formulations of the RE algorithm: the synchronous, 
asynchronous (centralized) and asynchronous (decentralized) formulation.

\alnote{Main issue: In Async-RE paper we only have TG/LONI numbers}
\jhanote{I'm glad for RE to be removed from this paper and left as
  Sai's contribution in the students paper. He has some graphs for
  that too}

\section{Conclusion}


\bibliographystyle{plain}
\bibliography{pilotjob,saga,saga-related}


\end{document}
