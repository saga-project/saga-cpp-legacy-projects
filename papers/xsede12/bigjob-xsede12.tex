\documentclass{sig-alternate}
\usepackage[numbers, sort, compress]{natbib}
\usepackage{graphics}
\usepackage{graphicx}
\usepackage{epstopdf}
\usepackage{color}
\usepackage{hyperref}
\usepackage{pdfsync}
\usepackage{mdwlist}
\usepackage{booktabs}


\begin{document}

\conferenceinfo{XSEDE '12, July 16-20, 2012, Chicago, USA.} {}
\CopyrightYear{2012}
\crdata{}
%\clubpenalty=10000
%\widowpenalty = 10000

% \title{BigJob: Lessons of Supporting High Throughput High Performance
%   Ensembles on XSEDE}

%\title{The Anatomy of a successful ECSS Project: Leveraging Expertise
%  of Users, Software-providers and XSEDE Personnel}

\title{The Anatomy of Successful ECSS Projects: Lessons of Supporting
  High-Throughput High-Performance Ensembles on XSEDE}

\numberofauthors{6}
\author{
\alignauthor Melissa Romanus\\
       \affaddr{CAC/ECE}\\
	\affaddr{Rutgers University} \\
       \affaddr{94 Brett Road} \\
	\affaddr{Piscataway, NJ} \\
       \email{melissa@cac.rutgers.edu}
\alignauthor Pradeep Kumar Mantha\\
       \affaddr{Center for Computation and Technology}\\
       \affaddr{Louisiana State University}\\
       \affaddr{216 Johnston}\\
       \affaddr{Baton Rouge, LA}
       \email{pmanth2@cct.lsu.edu}
\and
\alignauthor Matt McKenzie\\
       \affaddr{National Institute for Computational Sciences}\\
       \affaddr{Oak Ridge, TN} \\
       \email{mmckenz9@utk.edu}
\alignauthor Tom Bishop\\
       \affaddr{Louisiana Tech University}\\
       \affaddr{Ruston, LA} \\
       \email{bishop@latech.edu }
%\alignauthor Emilio Gallichio\\
%       \affaddr{ BioMaPS Institute - Rutgers University}\\
%       \affaddr{Piscataway, N} \\
%       \email{emilio@biomaps.rutgers.edu }
\and
\alignauthor Andre Merzky\\
       \affaddr{Center for Computation and Technology}\\
       \affaddr{Louisiana State University}\\
       \affaddr{216 Johnston}\\
       \affaddr{Baton Rouge, LA}\\
       \email{andre@merzky.net}
\and       
\alignauthor Yaakoub El Khamra\\
       \affaddr{Texas Advanced Computing Center}\\
       \affaddr{The University of Texas at Austin} \\
       \affaddr{Austin TX} \\
       \email{yaakoub@tacc.utexas.edu}       
\alignauthor Shantenu Jha\\
     \affaddr{CAC/ECE}\\
     \affaddr{Rutgers University}\\
      \affaddr{94 Brett Road}\\
      \affaddr{Piscataway, NJ}
     \email{shantenu.jha@rutgers.edu}
}

\maketitle


\begin{abstract}
  The Extended Collaborative Support Service (ECSS) of XSEDE is a
  program to provide support for advanced user requirements that
  cannot and should not be supported via a regular ticketing
  system. Recently, two ECSS projects have been awarded by XSEDE
  management to support the high-throughput of high-performance (HTHP)
  molecular dynamics (MD) simulations; both of these ECSS projects use
  a SAGA-based Pilot-Jobs approach as the technology required to
  support the HTHP scenarios.  Representative of the underlying ECSS
  philosophy, these projects were envisioned as three-way
  collaborations between the application stake-holders,
  advanced/research software development team, and the resource
  providers. In this paper, we describe the aims and objectives of
  these ECSS projects, how the deliverables have been met, and some
  preliminary results obtained. We also describe how SAGA has been
  deployed on XSEDE in Community Software Area as a necessary
  precursor for these projects.
\end{abstract}


\newif\ifdraft \drafttrue \ifdraft
\newcommand{\mrnote}[1]{{\textcolor{green} { ***MR: #1 }}}
\newcommand{\jhanote}[1]{ {\textcolor{red} { ***SJ: #1 }}}
\newcommand{\yyenote}[1]{ {\textcolor{cyan} { ***YYE: #1 }}}
\newcommand{\pmnote}[1]{ {\textcolor{blue} { ***PM: #1 }}}
\newcommand{\todo}[1]{ {\textcolor{red} { ***TODO: #1 }}}
\newcommand{\fix}[1]{ {\textcolor{red} { ***FIX: #1 }}}
\newcommand{\reviewer}[1]{} \else \newcommand{\yyenote}[1]{}
\newcommand{\mrmnote}[1]{} \newcommand{\pmnote}[1]{}
\newcommand{\jhanote}[1]{} \newcommand{\todo}[1]{ {\textcolor{red} {
      ***TODO: #1 }}} \newcommand{\fix}[1]{} \fi



\category{D.1.3}{Software}{Concurrent Programming}{ Distributed programming/parallel programming} 
\category{J.3}{Computer Applications}{Bioinformatics}

\section*{General Terms}{Design,Measurement,Theory}

 \keywords{}

\section{Introduction}



The Extended Collaborative Support Service (ECSS) pairs members of the XSEDE
user community with expert staff for an extended period to work together to
solve challenging science and engineering problems through the application of
cyberinfrastructure~(\cite{ECSS_webpage}). In depth staff support, lasting a few
weeks up to a year in length, can be requested at any time through the XSEDE
allocations process. Expertise is available in a wide range of areas, from
performance analysis and petascale optimization to the development of community
gateways and work and data flow systems.

For example, although individual Molecular Dynamics (MD) simulations
are very common on XSEDE, it is not very common to execute multiple MD
simulations concurrently, and even far less multiple MD simulations on
multiple MD resources concurrently. Although the abilty to execute an
ensemble of MD simulations is sought by many MD users, the abilty to
do so in a simple, scalable and general-purpose fashion does not
exists.  The ECSS provides an excellent program to build upon base
existing capabilties and transform them into a value-added service.

In this paper, we discuss two ECSS projects that overlap in both
scientific endeavor and computational challenges. The success of these
two projects is attributed to the strong interaction and familiarity
between ECSS staff and infrastructure developers. The collaboration
between system administrators and middleware developers has led to the
deployment of production-grade infrastructure that is currently being
used by scientists originally intended as beneficiaries of ECSS as well
as other scientists.

The computational challenges were circumvented through innovative solutions in
distributed and high throughput computing, thus satisfying the computational
scientists' requirements. The development and deployment of these tools is left
to computer scientists, developers, and programmers of infrastructure and
middleware. The ECSS staff supports both scientists and tool developers in
deployment, bug reporting, testing, fixes and system issues. This three-way
split insulates the scientists from the ``plumbing'' and the ECSS consultants
from middleware development. The successful distribution of labor translates
to reduced effort across the board, higher scientific output, and less issues.
Furthermore, the fact that middleware developers are in constant communication
with ECSS consultants improves the middleware itself. The ECSS consultants
report to the developers' issues ranging from middleware misbehavior,
to choke points, scaling issues, unnecessary increases in file system load, and so
on. The developers resolve these issues, test, and re-deploy while maintaining
backwards compatibility, leaving the scientists unaffected.

The overlap of two ECSS projects with the same computational challenges lead to
increased efficiency. The same tools, infrastructure, and documentation prepared
for one ECSS project are being used by another, similar project. There was,
however, a slight increase in the number of issues reported: twice the number
of users meant more bugs reported. This is only expected, and to some extent
quite welcome. At the end of the day, however, two ECSS projects are nearing
completion ahead of schedule for the cost of just one project.

In this paper, we discuss the ECSS projects, the computational challenges
involved and the middleware solution. We also discuss in detail the lessons
learned from testing and deploying on different machines and collaborating
across the three disciplines of scientists, middleware developers and ECSS
consultants.



\section{Scientific Background and Motivation}

In this section, we provide the scientific background of both ECSS
projects and the computational challenges inherent in the two
projects. We establish that there are common computational challenges
and thus {\it coupling} ECSS projects with similar computational
challenges with the same ECSS consultant and middleware developer team
is an efficient use of XSEDE resources.

\subsection{Distributed and Loosely Coupled Parallel Molecular Simulations}
The first ECSS project~(\cite{RonLevy}) is part of an intense effort
to understand important aspects of the physics of protein-ligand
recognition by multidimensional replica exchange (RE) computer
simulations. These are compute intensive calculations which require
large numbers ($10^3$-$10^4$) of loosely coupled replicas and long
simulation times (days to weeks). In conventional synchronous
implementations of RE, simulations progress in unison and exchanges
occur in a synchronous manner whereby all replicas must reach a
predetermined state (typically the completion of a certain number of
MD steps), before exchanges are performed. This synchronous approach
has several severe limitations in terms of scalability and
control. The first such limitation is that sufficient dedicated computational resources must be secured for all of the replicas before the simulation can
begin execution. Secondly, the computational resources must be
statically maintained until the simulation is completed. Thirdly,
failure of any replica simulation typically causes the whole
calculation to abort. These traditional limitations are overcome
in the asynchronous RE formulation. The scientific aim is to study a
range of physical systems using these advanced formulations and and to
utilize the distributed high-performance capacity of XSEDE resources.

\subsection{High Throughput, High Performance MD Studies of the
  Nucleosome}

The second ECSS project focuses on high throughput, high performance
molecular dynamics studies of the
nucleosome~(\cite{TomBishop}). Genomes in higher organisms exist for
most of the cell's cycle as a protein-DNA complex called chromatin.
Nucleosomes are the building blocks of chromatin, thus, nucleosome
stability and their positioning within chromatin impacts virtually all
genetic processes including: transcription, replication, regulation,
repair.

The primary goal of the proposed simulation studies in this project is to
investigate by means of all atom molecular dynamics simulations variations in
nucleosome structure and dynamics arising from DNA chemical modifications and
from receptor binding. This is being accomplished by means of a hierarchical
modeling and simulation strategy that utilizes high throughput, high
performance all-atom molecular dynamics simulation techniques. The project is
on track to generate approximately 140,000 ns of molecular dynamics simulation
data for at least 150 different realizations of the nucleosome.


\subsection{Computational Challenges}

Most runtime environments and tools for high-performance and
distributed computing assume either that there is a single instance of
a task/simulation that must be executed, or there is a complex
dependency (and thus workflow) associated with distinct ``tasks''.
However, in an increasingly large number of scientific problems, there
is a need for a large number of similar tasks to be run concurrently.

There is a fundamental need for supporting ensemble-based simulations
(ES) across a range of disciplines, including but not limited to
biological, chemical, environmental and material sciences; existing
capabilities are insufficient to support ES at scale. 

Depending upon the specific problem, the identical tasks may vary in
the number of cores required, or in the degree of coupling between the
tasks.  Either way, there are missing capabilities to support the
multiple similar tasks that need to run concurrently, and the management
of such concurrent tasks is a challenge for ES. 

Typically, for a given science problem, the number of tasks does not vary during
the lifetime of the execution. The objective of ES is similar to 
traditional workflows in that it aims for a reduction of the makespan.
However, in contrast to such workflows, there is not a multi-level
dependency to solve and thus the execution of ES 
is not fundamentally a challenging scheduling problem.

The concurrent support of multiple similar simulations requires the
efficient and transparent management of a large number of ensembles,
possibly on many heterogeneous resources. The same runtime solution
should be applicable to a range of physical model sizes -- from
thousands of atoms to hundreds of thousands of atoms.
The above should be addressed without being tied to a specific
underlying application kernel or physical infrastructure. 
 
Although multi-site resources have the potential for substantial
improvements in time-to-completion of ES, the capability to
effectively utilize NSF's National Production Cyberinfrastructure
XSEDE is missing, as developing, deploying and executing these
capabilities to use the collective power of XSEDE is a challenging
undertaking, which cannot be carried forth by a single group or
sustained without the involvement of all stakeholders: scientists,
resource providers and software developers.

\section{ECSS Projects}

The two ECSS projects were aligned in infrastructure, tools and
workflow.  For example, the projects were synergistic in that
the requirements of one were often also requirements of the
other. This meant that the deployment of the infrastructure (SAGA and BigJob), 
the development of the actual workflows, and data management tools were
requirements for both ECSS projects. This section outlines the
level of effort involved in both projects.

\subsection{Project 1}
\subsubsection*{Distributed and Loosely Coupled Parallel Molecular Simulations}
The reliance on a static pool of computational resources and lack of
fault tolerance prevents the synchronous RE approach from being a
solution on XSEDE resources. We therefore implemented asynchronous
parallel replica exchange conformational sampling algorithms together
using the SAGA distributed computing framework to enable dynamic
scheduling of resources and adaptive control of replica
parameters. The approach allows pairs of replicas to contact
each other to initiate and perform exchanges independently from the
other replicas. Because replicas do not rely on centralized
synchronization steps, asynchronous exchange algorithms are scalable to an
arbitrary number of processors. These algorithms also circumvent
the need to maintain a static pool of processors and therefore
can be distributed logically and physically across XSEDE resources.
In that case, the number of concurrent replicas changes dynamically
depending on available resources. This mode of execution is particularly
suitable for implementation using SAGA and tools based upon SAGA (i.e. BigJob). The technical details of both SAGA and BigJob are discussed in Sections \ref{saga} and \ref{bigjob}.

In terms of scope, assistance from XSEDE personnel was needed to: (i) set up and
harden the necessary SAGA, BigJob and Advert Service software infrastructures on
Ranger, Lonestar, Kraken, and Nautilus and deploy the scientific software
IMPACT~(\cite{IMPACT}) and AMBER~(\cite{AMBER}) (ii) work with the
users to enable launching NAMD~(\cite{NAMD}), 
IMPACT, and AMBER distributed jobs using the
SAGA-BigJob framework, (iii) work with users to develop customized RE scripts
and with the SAGA team to test and validate relevant adaptors aimed at
conducting synchronous and asynchronous file-based replica exchange simulations
in the context of the SAGA-based Pilot-Job framework, and (iv)  document and
validate the usage of the Replica-Exchange framework on XSEDE.

Hardening the software infrastructures requires profiling and benchmarking the
workflow management tools (SAGA, BigJob) to measure overhead and identify any
load spikes on the filesystems. Once the bottlenecks and problem areas were
identified, they were resolved by the SAGA-BigJob
development teams with help from
the ECSS consultants. The SAGA-BigJob development team handled all major feature
implementations and bug fixes required for this effort including custom
adaptors for Lonestar and Ranger. 

\subsection{Project 2}
\subsubsection*{High Throughput, High Performance MD Studies of the
  Nucleosome}

The workflow for the second ECSS project requires simulating as many as 85
independent systems simultaneously. For each system, every nanosecond of
simulation is an HPC event that requires approximately 50 MB of input, generates
4 GB of output, and scales efficiently from as few as 32 to as many as 512 cores.
On 64 processors, the runtime for a single 1 ns simulation task is approximately
6 hrs. The goal is to accumulate 50 ns of simulation for the entire set of 85
systems (4,250 1 ns simulation tasks in total) as quickly as possible, then
select a subset of systems from the ensemble for which 500 ns of additional
dynamics will be accumulated. If we chose only 10 systems for continued studies,
there would still be 5,000 1 ns simulation tasks to be completed. To reduce the
time-to-completion of the entire workload and to manage simulation
inputs/outputs on scratch file systems, we intend to use multiple XSEDE
resources simultaneously. This approach requires data staging.

In contrast to Project 1, this project requires: 
(i) a much larger number of ensembles
running concurrently and for longer durations, (ii) the chaining of
ensembles, (iii) much larger data-volumes, (iv) BigJob infrastructure that currently does not support the
co-movement and coordinated movement of data (files) in conjunction
with ensemble placement. BigJob file movement capabilities are being
enhanced to provide such support.

Support from ECSS consultants is necessary for disk
and storage bottlenecks. Data management (especially varying transfer rates,
less-than-portable transfer mechanisms, etc.) is a growing issue that must be 
addressed in order for this project to be successful.

\subsection{Common Effort across Both Projects}

Both projects share the same basic infrastructure in terms of software and
hardware. Both projects share consultant teams, software developers and some of
the scientific tools. The workflows are not dissimilar either: both ECSS
projects intend to launch pilot jobs in a dynamic workflow on distributed
resources, stage and collect data, etc.

With a view towards making these capabilities more publicly accessible and
widely used, this ECSS effort also includes the development of a prototype
Replica-Exchange Gateway. This gateway was developed using the
DARE~(\cite{DARE})
scientific gateway framework. This scientific gateway prototype is hosted
on IU's Data Quarry machine~(\cite{DataQuarry}). Furthermore, all tools,
utilities, and components resulting from this ECSS will be placed in the Community 
Software Area (CSA) space on Ranger,
Lonestar, and Kraken. All code and corresponding documentation will also be
maintained in a local revision control software repository which is publicly
available.


\section{Methodical Solution and Technology}

Although the objectives of the two ECSS projects are distinct, there
is certain amount of commonality in the technology employed. Both
projects use SAGA and SAGA-based Pilot-Jobs to support the concurrent
execution of multiple simulations (ensembles or replicas as the case
maybe). In this section we briefly introduce the technology employed.

\subsection{SAGA: Standards-based Access to the Resources Layer}
\label{saga}
XSEDE is inherently a very complex infrastructure.  Given the wide
range of user groups and application use cases it aims to address,
and the large number and the diversity of participating resource
providers, this is to be expected.  Complex systems are though
usually very difficult to use, as that complexity and the underlying
resource diversity often translates into complicated user tools and
interfaces.  The relatively clean architecture of XSEDE is, to some
extent, addressing this problem, but is, in itself and at this point
in time, a moving target.

The Simple API for Grid Applications (SAGA)~(\cite{saga_url}) aims to address a
part of that problem, by providing a well defined and stable API, which
exposes those operations which are required on application level, but
encapsulates the complexity of translating them into the respective
operations on the XSEDE infrastructure.  In other words: SAGA tries
to move the complexity of dealing with distributed
cyberinfrastructures like XSEDE out of the application, and into the
SAGA implementation layer, while providing the semantics necessary to
efficiently implement distributed applications which can utilize
XSEDE.

SAGA is an implementation of an Open Grid Forum (OGF) Technical Specification
that provides a common and consistent high-level API for the most
commonly required functionality to construct distributed applications.
It also provides a high-level API to construct tools and frameworks to
support distributed applications. The functional areas that are
supported by SAGA include job-submission, file transfer and access, as
well as support for data streaming and distributed coordination. SAGA
provides both a syntactic and semantic unification via a single
interface to access multiple, semantically distinct middleware
distributions.

%SAGA is an API that provides the basic functionality for developing
%distributed applications, tools and frameworks. 

The key advantages of the development using SAGA include, but are not
limited to: i) to provide a general-purpose, commonly used yet
standardized functionality, while hiding complexity of heterogeneity
of back-end resources, ii) to provide building blocks for constructing
higher-level functionality and abstractions, iii) to provide the means
for developing broad range of distributed applications such as
gateways, workflows, application management systems, and runtime
environments. Interestingly, SAGA provides an integrated, light-weight
approach to support scripting for building distributed applications.

Different aspects of SAGA appeal to different groups. The
standardization of SAGA as an OGF Standard is important because it
makes it more likely that production infrastructures, like NSF XSEDE,
EU PRACE and Open Science Grid, will support SAGA. Having SAGA
deployed and tested on these systems makes it readily available for
users and developers of national Cyberinfrastructure projects. The
fact that SAGA is an OGF technical specification also makes SAGA
highly appealing to application frameworks, services and tool
developers, which is quite understandable, as it not only simplifies
their development but also makes for scalable, extensible and
sustainable development. Users find the simple and extensible
interface providing the basic abstractions required for distributed
computing is very appealing to add their own ``functionality'' to a
core base of functionality. Furthermore, SAGA is now part of the
``official'' access layer for the \$121M NSF TG/XSEDE
project~(\cite{xsede_url}), as well as for the world largest distributed
infrastructure EGI~(\cite{egi_url}).

The SAGA API provides the base abstractions upon which tools and
frameworks that provide higher-level functionality can be
implemented. Ref.~(\cite{saga_url}) discusses distributed application
frameworks and run-time systems that SAGA has been used to develop
successfully. 

\begin{figure}[t]
\centering
\includegraphics[width=0.52\textwidth]{./figs/saga-architecture-1}
\caption{\textbf{SAGA Overview: } SAGA is an OGF technical
  specification that provides a common interface to heterogeneous DCI
  -- hitherto typically Grid systems.  The implementation of the
  SAGA~(\cite{saga_url}) specification consists of a high-level {\it
    API}, the SAGA {\it Engine} providing that API, and backend,
  system-specific {\it Adaptors}.  The engine is a lightweight,
  highly-configurable dynamic library that manages call dispatching
  and the dynamic runtime loading of the middleware adaptors.  Each of
  these adaptors implements the functionality of a specific functional
  package (e.g., job adaptors, file adaptors) for a specific
  middleware system. Adaptors are also realized as dynamic libraries.}
 \label{fig:saga-overview}
\end{figure}

A SAGA implementation consists of a high-level API, the SAGA
Engine providing that API, and backend, middleware/systems specific
adaptors. Each of these adaptors implements the functionality of
a functional package (e.g., job adaptors, file adaptors) for a
specific middleware system. The engine is a dynamic library that
manages call dispatching and the runtime loading of the middleware
adaptors. Adaptors are also realized as dynamic libraries. The SAGA
API has been used (in C++, Python and Java versions) to provide almost
complete coverage over nearly all grid and distributed computing
middleware/systems, including but not limited to Condor, Genesis,
Globus, UNICORE, LSF/PBS/Torque and Amazon EC2.

SAGA is currently used on production Cyberinfrastructure in several
ways. Admittedly the number of first-principle distributed
applications developed is currently low, but SAGA has been used to develop
``glue-code'' and tools that are used to submit and marshal jobs and
data across and between heterogeneous resources. Specifically, it has
been used to support multiple computational biology applications that
use high-performance and high-throughput molecular dynamics (MD)
simulations.

SAGA has been used to develop a standards-based library for Science
Gateways to easily utilize different distributed resources; some
science domains that are using SAGA-based Science Gateways include
gateways to support next-generation sequencing, docking and
high-throughput of ensembles.


\subsection{SAGA-based Pilot-Job: BigJob}
\label{bigjob}
A common approach to circumvent the computational challenges faced by our ECSS
projects is the use of container jobs with sophisticated workflow management to
coordinate the launch and interaction of actual computational tasks within the
container. This is a particular instance of a more general
concept: \emph{pilot-jobs (PJ)}. A pilot-job is a mechanism by which a proxy
for the actual simulations is submitted on the resource to be
utilized; this proxy agent in turn, is given responsibility to convey
to the application the availabilty of resources and also influence
which tasks are executed. The abstraction of a Pilot-Job generalizes
the reoccurring concept of utilizing a placeholder job as a container
for a set of compute tasks; instances of that placeholder job are
commonly referred to as Pilot-Jobs or pilots.The SAGA-based pilot-job
is a container job which can include within its fold multiple smaller
tasks.

The pilot-job (PJ) abstraction is also a promising route to address
additional requirements of distributed scientific
applications~(\cite{ko-efficient,bigjob_cloudcom10}), such as
application-level scheduling. The abstraction of a Pilot-Job
generalizes the reoccurring concept of utilizing a placeholder job
as a container for a set of compute tasks; instances of that
placeholder job are commonly referred to as Pilot-Jobs or pilots.

A SAGA-based PilotJob, BigJob (BJ)~(\cite{bigjob_web,saga_bigjob_condor_cloud}),
is a general-purpose pilot-job framework. BigJob has been used to support
various execution patterns and execution
workflows~(\cite{async_repex11,repex_ptrsa}). For example, SAGA-BigJob was used
to execute scientific applications categorized as embarrassingly parallel
applications and loosely coupled applications on scalable distributed
resources~(\cite{ecmls_ccpe10, dare-ecmls11})

Figure~\ref{fig:figures_re_bigjob_interactions} illustrates the
architecture of BJ. BJ utilizes a Master-Worker coordination
model. The BigJob-Manager is responsible for the orchestration of
pilots, for the binding of sub-tasks. For submission of the pilots,
SAGA relies on the SAGA Job API, and thus can be used in conjunction
with different SAGA adaptors, e.\,g.\ the Globus, the PBS, the Condor
and the Amazon Web Service adaptor. Each pilot initializes a so called
BJ-agent. The agent is responsible for gathering local information and
for executing tasks on its local resource. The SAGA Advert Service API
is used for communication between manager and agent. The Advert
Service (AS) exposes a shared data space that can be accessed by
manager and agent, which use the AS to realize a push/pull
communication pattern.  The manager pushes a sub-job to the AS while
the agents periodically pull for new sub-jobs. Results and state
updates are similarly pushed back from the agent to the
manager. Furthermore, BJ provides a pluggable communication \&
coordination layer and also supports alternative c\&c systems,
e.\,g.\ Redis~(\cite{redis}) and ZeroMQ~(\cite{zmq}).

In many scenarios it is beneficial to utilize multiple resources,
e.\,g.\ to accelerate the time-to-completion or to provide resilience
to resource failures and/or unexpected delays.  BJ supports a wide
range of application types, and is usable over a broad range of
infrastructures, i.\,e.\ it is general-purpose and extensible
(Figure~\ref{fig:figures_re_bigjob_interactions}). 


\begin{figure}[t]
  \centering
  \includegraphics[width=0.51\textwidth]{./figs/re_bigjob_interactions}
   \caption{\textbf{BigJob Architecture:} The core of the framework,
     the BigJob-Manager orchestrates a set of pilots. Pilots are
     started using the SAGA Job API. The application submits WUs, the
     so-called sub-jobs via the BigJob-Manager. Communication between
     the BJ-Manager and BJ-Agent is done via a shared data space, the
     Advert Service. The BJ-Agent is responsible for managing and
     monitoring sub-jobs. From Ref.~(\cite{saga_bigjob_condor_cloud})}
        \label{fig:figures_re_bigjob_interactions}
\end{figure}

HPDC infrastructure is by definition comprised of a set of resources
that is fluctuating -- growing, shrinking, changing in load and
capability. This is in contrast to a static resource utilization model
traditionally a characteristic of parallel and cluster computing. The
ability to utilize a dynamic resource pool is thus an important
attribute of any application that needs to utilize HPDC infrastructure
efficiently. Applications that support dynamic execution have the
ability to respond to a fluctuating resource pool, i.\,e., the set of
resources utilized at time (T), $T=0$ is not the same as $T>0$.  Thus,
the need to support dynamic execution is widespread for computational
science applications; here we accomplish this by using the SAGA-based
Pilot-Job.

\section{Deep Integration of SAGA/BigJob on XSEDE}

\subsection{CSA Deployment of SAGA and BigJob}
 \label{ssec:csa}
 
SAGA and BigJob are deployed on all major XSEDE
machines~(Table \ref{table:CSA-Deployments}) (Ranger,
Kraken, Lonestar, Trestles, Blacklight, and Steele) as well as FutureGrid machines
(india, sierra, hotel, alamo). While it is possible to install and
use SAGA and BigJob each user's directory, it would not be without a non-trivial
amount of effort in configuration and maintenance. Furthermore, having a
central location on each machine where SAGA is deployed as a community code
saves ECSS consultants time and effort in maintaining the installation and
supporting its users. XSEDE continues the TeraGrid model of providing Community
Software Areas (CSA) to communities, which basically is a system
level area where software packages can be shared among a community of users.
Therefore it was only logical to deploy SAGA and BigJob in a CSA space on XSEDE
and FutureGrid machines.

The installation and update are semi-automatic: a set of deployment
scripts are used to manually trigger the update. The update can encompass an
individual machine or all machines, an individual SAGA component or the entire
SAGA/BigJob installation. The set of supported components includes the SAGA core
libraries, different API packages, the supported
middleware adaptors for each machine, the python bindings, and the
BigJob package and its dependencies. Each installation automatically creates a
README file that describes the installed components, and documents settings
required to use the installation, settings such as the
\texttt{LD\_LIBRARY\_PATH} and so on. A module file is also generated and on
some hosts is symbolically linked to the system default module path. 

After each update, a set of unit tests is run \textit{on the
target machines} to ensure that not only the updated version is deployed, but
more importantly, that it is correctly installed and
configured for that target machine. The unit tests range from basic
(low-level) environment testing, to application level test runs of a
BigJob application. The test README, module file and test results
are all committed to the central SAGA code repository, and
(partially) used to document the current state of deployment on the
SAGA deployment wiki.

\begin{table}
\begin{center}
\begin{tabular}{ll}
\toprule
\textbf{Machine}  & 
\textbf{Adaptors Supported} 
\\ \midrule
Kraken   & 
X509, Globus, SSH, Torque/PBS
\\ \midrule
Ranger   & 
X509, SSH
\\ \midrule
Lonestar & 
X509, Globus, SSH
\\ \midrule
Blacklight & 
X509, Globus, SSH, Torque/PBS
\\ \midrule
Steele & 
X509, Globus, SSH, Torque/PBS
\\ \midrule
Trestles & 
X509, Globus, SSH, Torque/PBS
\\ \bottomrule
\end{tabular}
\caption{CSA Deployments on XSEDE Resources}
\label{table:CSA-Deployments}
\end{center}
\end{table}

\subsection{Developments and Challenges}

The use of multiple resources brings about certain challenges which
tend to reappear in every distributed ECSS project. For example, each
of the machines (Lonestar, Ranger, Kraken, and Trestles) may have
different user environments, different shells, and different
invocation order of startup profiles on both login and compute
nodes. A major complication is the differences in system versions of
Python. Most machines have an older version of Python supplemented by
a Python module. The Python module did not typically include the tools
required to easily install user-side python modules. Therefore, a
fresh installation of Python is always present in the shared CSA
space. A similar issue is encountered with GCC compiler versions as
well.

A slightly more intricate issue is the use of custom ``mpirun'' wrappers. On
Ranger/Lonestar, it is ``ibrun'', on Kraken, it is ``aprun''. These wrappers
massage the nodelist files, aggregate important environment variables to launch
with the application and so on. Modifications have to be made the launch
mechanism in BigJob to account for the use of these scripts.

Job submission is another interesting issue. Lonestar and Ranger use SGE, while
Kraken and Trestles use PBS. While SAGA retains the ability to submit jobs
through the Globus (\cite{Globus}) job adaptor, it is an unnecessary burden on
users. Furthermore, when Globus submitted jobs fail, they generate a very
lengthy error report without much useful information. Both projects needed an
immediate, clear, and fail safe mechanism to submit jobs and this lead to the
development of the \textit{pbs-ssh} and \textit{sge-ssh} plugins to support both
PBS and SGE. The plugins enable local/remote launch of BigJob agents using
traditional PBS/SGE script over SAGA ssh job adaptors.

The last issue is a user-side issue. The more diverse the machines and their
environments are, the more diverse the documentation and the higher the entry
barrier. For example, Kraken requires the initialization of a ``myproxy'' for
successful job submission, whereas Ranger and Lonestar do not. These small  but
critical differences can mean the difference between successful several thousand
core jobs or a week of waiting in the queue to exit on an error.


\subsection{Testing and Documentation Process}

The source code for SAGA and BigJob is stored in a
git repository (\cite{bigjob_web}). A github wiki is used to store user guides
for each of the individual XSEDE machines. The BigJob wiki stores all
information about the installation and configuration of BigJob. Only users of
the BigJob development group can edit this wiki. This wiki is public and can be
shared amongst all collaborators. Public wikis also serve as a way to promote
other people to use and try BigJob for their scientific needs.

A BigJob CSA release is the result of a production pipeline. Any newly developed
features, code modifications and bug fixes are created in branches. After
review, branches are merged onto the master branch of the git repository. The
main developer determines when a new version of BigJob will be released. 

Members of the BigJob development team then checkout the release candidate
version from git and test on XSEDE resources. Tests include checking output and error logs, 
ensuring that the installation is seamless, ensuring backwards compatibility with existing 
scripts and workflows, monitoring the submission environment and so on. This rigorous quality 
assurance process is repeated for every machine.

For every machine there are least two machine-specific examples in the
git repository and CSA space. These are example BigJob scripts that
run a job in both single and multiple communication (i.e. MPI)
mode. The first script simply runs a shell ``/bin/date''
command. Since the ECSS project supports molecular dynamics
simulations, the second script runs a real AMBER MD simulation. These
scripts test the basic functionality of BigJob using the CSA
installations and serve as standard tests.

With each testing phase, the testing team follows the instructions in the user
guide, step-by-step, in a sterile user environment to make sure the
documentation is correct and the updates are backwards compatible.
Naturally, each machine has scripts that are tailored to the batch queue submission
system on that machine -- i.e. Ranger uses an SGE submission while Kraken uses
PBS. By testing these across multiple job submission systems, we are also
exercising the backend job submission BigJob code to make sure that changes have
not negatively affected the submission. After the jobs finish successfully, the
output is analyzed to validate the results. The AMBER scripts also serve as a
test for the AMBER installations on the appropriate machines, and allow the
testing team to check that AMBER starts and runs normally.

After testing is complete, the python code is pushed to the Python Package Index
(\textit{pypi}). This code is then deployed into the CSA space using pypi.
Careful consideration is taken to ensure that the updates to CSA space
will not impact any current users of BigJob on the XSEDE resources. Another round of testing 
is then completed to verify that the CSA installations are working and no changes to 
the users' environments are required. Any corresponding documentation on the 
github wiki is updated to reflect the changes.

In addition to this release process, the BigJob team is engaged with members
of the ECSS team in order to resolve any system-level issues that may arise.
These issues may include but are not limited to differences in schedulers or MPI
implementations. Additionally, the scientific collaborators use the wiki as a
starting point to run their applications on XSEDE machines. The scripts and
associated documentation explain how to use BigJob with their own applications
simply by specifying the job description. 

If the end users encounter any issues, the BigJob team works with the ECSS
consultants to resolve the problems. A mailing list including ECSS consultants,
BigJob development and deployment teams and the end users ensures the fast
possible response time. If any questions arise during the end user's use of
BigJob, the BigJob team is available to fully assist them. This assistance can
range from simple diagnostics of unexpected output to  the creation of
custom scripts for the researchers to use. For instance, some MD simulations
run until a certain time step and then need to be restarted, thus requiring a
custom BigJob script. The BigJob team then leveraged the existing BigJob
functionality to create a custom script that can submit a number of jobs after
the first set of jobs finishes. In the future, we plan to use a shared space
where scientists can share their custom scripts and workflows with others.

\section{Computational Progress}

In this section, we aim to demonstrate how the end-users are utilizing
the infrastructure in novel and interesting ways, that were not
possible previously. In order to do so, we provide a couple of
representative configurations of the size of the BigJob submitted, the
number of sub-jobs executed as part of the BigJob.

% \mrnote{Show work from chemists (if available), Applicability of
%   BigJob to other applications beyond RE / chemists}

\begin{table}[h]
\begin{center}
\begin{tabular}{p{1.1cm}p{1.2cm}p{1.2cm}p{1.2cm}p{1.1cm}p{0.8cm}}
\toprule
\textbf{Machine}  & 
\textbf{\# of}    &
\textbf{BJ Size} & 
\textbf{Duration} & 
\textbf{\# of}    &
\textbf{Size}     \\
                 &
\textbf{BigJobs} &
\textbf{(cores)} &
\textbf{(hours)} &
\textbf{SubJobs} &
\textbf{(cores)} 
                  \\ \midrule
Lonestar & 2 & 1200 & 24 & 100 & 12 \\ \midrule
Lonestar & 5 & 2400 & 24 &  50 & 48 \\ \midrule
Ranger   & 1 & 2400 & 24 &  50 & 48 \\ \midrule
Ranger   & 1 & 1800 & 24 &  50 & 48 \\ \midrule
Kraken   & 1 & 1800 & 24 &  50 & 36 \\ \bottomrule
\end{tabular}
\caption{Configuration of Jobs submitted as part of ECSS Project
  2.}
\label{table:results1}
\end{center}
\end{table}


\begin{table}[h]
\begin{center}
\begin{tabular}{p{1.1cm}p{1.2cm}p{1.2cm}p{1.2cm}p{1.1cm}p{0.8cm}}
\toprule
\textbf{Machine}  & 
\textbf{\# of}    &
\textbf{BJ Size}     & 
\textbf{Duration} & 
\textbf{\# of}    &
\textbf{Size}     \\
                  &
\textbf{BigJobs}  &
\textbf{(cores)} &
\textbf{(hours)}  &
\textbf{SubJobs}  &
\textbf{(cores)} 
                  \\ \midrule
Lonestar & 1 &  200 & 24 &  50 &  4 \\ \midrule
Ranger   & 1 &  800 &  6 &  50 & 16 \\ \midrule
Ranger   & 1 &  800 &  3 &  50 & 48 \\ \bottomrule
\end{tabular}
\caption{Configuration of jobs submitted as part of ECSS Project
  1. Interestingly, these tasks were data analysis tasks (Normal Mode
  Analysis).}
\label{table:results}
\end{center}
 \end{table}

From the data in Table~\ref{table:results1} and
Table~\ref{table:results}, it is evident that the end-users are able
to marshall a very large number of simulations concurrently, which
they were not able to do earlier. This is thanks to the fact that
BigJob provides first-class support for ensembles of simulations.

BigJob has been used extensively for ensemble-based simulations as well as data
analysis. Table~\ref{table:results} shows several hundred individual simulations
launched within BigJob Pilot or container jobs. The first use case for example
included several hundred independent analysis simulations that are part of the
efforts to understand HIV drug resistance.

Simulations captured by the data in Table~\ref{table:results} were
used to investigate mutational impact on drug binding in the HIV-1
reverse transcriptase enzyme. In this system mutations distal to the
drug binding site induce resistance and the scientists are seeking
plausible explanation for how this occurs.

Although the software operating environments of Kraken and Ranger are
very different thanks to the deep integration of SAGA and BigJob on
XSEDE, for the end-uer the operating environment (ie BigJob) remains
invariant.  These leads to seamless uptake of different machines for
advanced capabilities, i.e., these capabilities are not tied to a
specific machine as was historically, and the end-user scientist can
utilize these capabilities not only over individual different
resources but multiple different resources concurrently. This can be
seen from the plot shown in Figure~\ref{fig:multi_bigjob}, whereby,
several BigJobs were executed on multiple XSEDE resources
concurrently.

\begin{figure}
  \centering
  \includegraphics[scale=0.6]{figs/multi_bj_gnu}
  \caption{\footnotesize This figure compares the time to completion
    (including waiting time) when the BigJobs are submitted to only
    one machine and when BigJobs are submitted to multiple
    machines. Here we submit jobs to Kraken, Ranger and Lonestar and
    assign the workload to the BigJob that becomes active first. This
    way we are able to select the BigJob with the least waiting
    time. The time to completion includes queue waiting time. In each
    case we submitted 3 BigJobs per machine of size 2,016 processors
    and assigned 96 cores per sub-job. }
  \label{fig:multi_bigjob}
\end{figure}



%\section{Conclusions and Future Work}
\section{Discussions and Lessons Learnt}

The strong collaboration between XSEDE system side consultants
and middleware developers ensures rapid functionality
development and bottlenecks/show-stopper-bugs are identified very
early in the development cycle and squashed immediately. A specific aspect
that can withstand improvement is the collaboration between XSEDE consultants
and the end-user scientists. Having an ECSS consultant whose research overlaps
with the scientific aims of both projects would have been welcome. The most
effective aspect of this project is undoubtedly the overlap of the two ECSS
projects in terms of tools and infrastructure (software and hardware). While the
ECSS staff's research interests are not aligned with MD simulations, they
are however familiar with the particular MD simulation tools and
packages used.

The triple-pronged approach to the ECSS consultants insulated scientists from
the ``plumbing'' of infrastructure allowing them to focus on the science. The
middleware/tool developers were free to focus on the computational challenges
presented by the scientific problem and ensured strong collaboration with the
ECSS consultants. The end result is an infrastructure solution that satisfies
the computational needs and is system/resource friendly.

The technical difficulties encountered in porting from the first
machine to the second machine prepared the ECSS team to the possible
pitfalls in porting to the third and fourth machines. Consequently,
the testing procedure incrementally became more rigorous as the number
of machines increased. Perhaps the most important technical lesson
learnt is the need to keep track of common problems and expect them to
repeat on new systems.

The importance of a rapid ``hot-patching'' deployment scheme cannot be
underestimated. Whenever a bug is discovered it is immediately
reported by the end users or testing team to the consultants. Issues
relating to the system are quickly identified and workarounds
suggested to the development team. The development team creates a
hot-fix, the deployment team pushes it to the machines and it
undergoes testing to ensure the user issue is resolved. Having a
tightly knit group of developers and ECSS consultants makes the entire
difference between resolving bugs in deployment or having end users
abandon the infrastructure due to unresolved bugs. This is where
involving the ECSS consultants in the infrastructure development cycle
pays off.

In summary, we encourage other ECSS projects and ECSS project
management to consider adopting the following best practices:
\begin{itemize}
 \item Align multiple projects that use the same infrastructure
 with the appropriate consultants. This is the most ECSS consultant
 ``bang'' for scientific discovery ``buck''
 \item Establish strong collaboration between middleware and infrastructure
 developers and the ECSS consultants as early as possible. Familiarity with the software makes for early
 identification of problematic issues, quick bug fixes and easier testing
 \item Establish strong collaboration between ECSS consultants and the science
 end-users whenever possible. In our project we have not been as lucky as to
 have an ECSS consultant whose scientific research is aligned with the end-users
 but experience from the ECSS consultants on other projects show that this is highly
 desirable.
\end{itemize}



\section{Acknowledgments}
In addition we are grateful to Abhinav Thota who provided much of the
initial support (Bishop) and to Ole Weidner for support with SAGA. We
are grateful to the BigJob Development team, includingAndre Luckow for
his original development of BigJob. Part of this work also was
performed as part of an ECSS in collaboration with Ron Levy and Emilio
Gallichio. We also thank Dave Wright (UCL) and Charlie Laughton
(Nottingham) for providing useful input and feedback. Computing
resources used for this work were made possible via NSF TRAC award
TG-MCB090174 and LONI resources.  \bibliographystyle{abbrv}
\bibliography{bigjob-xsede12,saga}
\end{document}

