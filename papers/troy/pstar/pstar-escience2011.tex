\documentclass[conference,final]{IEEEtran}

\usepackage{latex8}
\usepackage{times}

\usepackage[utf8]{inputenc}
\usepackage{url}
\usepackage{float}
\usepackage{times}    
\usepackage{multirow}    
\usepackage{listings}   
\usepackage{times}     
\usepackage{paralist}    
\usepackage{wrapfig}    
\usepackage[small,it]{caption}
\usepackage{multirow}
\usepackage{ifpdf}
%\usepackage{srcltx}
\usepackage{subfigure}
\usepackage{paralist}

\usepackage{listings}
\usepackage{keyval}  
\usepackage{color}
\definecolor{listinggray}{gray}{0.95}
\definecolor{darkgray}{gray}{0.7}
\definecolor{commentgreen}{rgb}{0, 0.4, 0}
\definecolor{darkblue}{rgb}{0, 0, 0.4}
\definecolor{middleblue}{rgb}{0, 0, 0.7}
\definecolor{darkred}{rgb}{0.4, 0, 0}
\definecolor{brown}{rgb}{0.5, 0.5, 0}

\newif\ifdraft
\drafttrue
\ifdraft
\newcommand{\jhanote}[1]{ {\textcolor{red} { ***shantenu: #1 }}}
\newcommand{\alnote}[1]{ {\textcolor{blue} { ***andre: #1 }}}
\newcommand{\smnote}[1]{ {\textcolor{green} { ***sharath: #1 }}}
\newcommand{\msnote}[1]{ {\textcolor{cyan} { ***mark: #1 }}}
\newcommand{\note}[1]{ {\textcolor{magenta} { ***Note: #1 }}}
\else
\newcommand{\alnote}[1]{}
\newcommand{\athotanote}[1]{}
\newcommand{\smnote}[1]{}
\newcommand{\jhanote}[1]{}
\newcommand{\msnote}[1]{}
\newcommand{\note}[1]{}
\fi

\lstdefinestyle{myListing}{
  frame=single,   
  backgroundcolor=\color{listinggray},  
  %float=t,
  language=C,       
  basicstyle=\ttfamily \footnotesize,
  breakautoindent=true,
  breaklines=true
  tabsize=2,
  captionpos=b,  
  aboveskip=0em,
  belowskip=-2em,
  %numbers=left, 
  %numberstyle=\tiny
}      

\lstdefinestyle{myPythonListing}{
  frame=single,   
  backgroundcolor=\color{listinggray},  
  %float=t,
  language=Python,       
  basicstyle=\ttfamily \footnotesize,
  breakautoindent=true,
  breaklines=true
  tabsize=2,
  captionpos=b,  
  %numbers=left, 
  %numberstyle=\tiny
}

\newcommand{\up}{\vspace*{-1em}}
\newcommand{\upp}{\vspace*{-0.5em}}
\newcommand{\numrep}{8 }
\newcommand{\samplenum}{4 }
\newcommand{\tmax}{$T_{max}$ }
\newcommand{\tc}{$T_{C}$ }
\newcommand{\tcnsp}{$T_{C}$}
\newcommand{\bj}{BigJob}

% This is now the recommended way for checking for PDFLaTeX:
\usepackage{ifpdf}

%\newif\ifpdf
%\ifx\pdfoutput\undefined
%\pdffalse % we are not running PDFLaTeX
%\else
%\pdfoutput=1 % we are running PDFLaTeX
%\pdftrue
%\fi

\ifpdf
\usepackage[pdftex]{graphicx}
\else
\usepackage{graphicx}
\fi

% \title{Towards A Framework for Pilot-Abstractions for Production
%   Cyberinfrastructure}

\title{P*: An Extensible Model of Pilot-Abstractions for Dynamic Execution}

% \jhanote{Alternate title: The Tiered Resource OverlaY framework
%   (TROY): An Empirical Framework for Pilot-* Abstractions}
% 
% \jhanote{old title: TROY -- Tiered Resource Overlay Framework: Towards
%   a Framework for Pilot-Abstractions for Distributed
%   Cyberinfrastructure}


\date{}

\begin{document}

\ifpdf
\DeclareGraphicsExtensions{.pdf, .jpg, .tif}
\else
\DeclareGraphicsExtensions{.eps, .jpg}
\fi

\author{
  Andre Luckow$^{1}$, Mark Santcroos$^{2,1}$, Sharath Maddineni$^{1}$, Andre Merzky$^{1}$, Shantenu Jha$^{3,1*}$\\
  \small{\emph{$^{1}$Center for Computation \& Technology, Louisiana State University, USA}}\\
 \small{\emph{$^{2}$Bioinformatics Laboratory, Academic Medical Center, University of Amsterdam, The Netherlands}}\\
 \small{\emph{$^{3}$ Rutgers University, Piscataway, NJ 08854, USA}}\\
  \small{\emph{$^{*}$Contact Author: \texttt{shantenu.jha@rutgers.edu}}}\\
  \up\up\up\up }

\maketitle

\begin{abstract}
  Distributed cyberinfrastructures (DCI) and applications require the ability
  to utilize resources dynamically, and not just statically.  Pilot-Jobs
  have been notable in their ability to support dynamic resource
  utilization -- both in the number of applications that use them, as
  well as in the scope of usage and number of CI that support them.\alnote{do we 
  use CI or DCI as abbrev?}
  In spite of broad uptake, however, there does not exist a well
  defined, unifying theoretical framework for Pilot-Jobs which can be
  used to compare, contrast and define different implementations. This
  paper is an attempt to (i) provide a minimal but complete model (P*)
  of Pilot-Jobs, (ii) extend the basic model from compute jobs to
  data, (iii) introduce TROY as an implementation of this model using
  SAGA, i.\,e.\ consistent with its API, job-model etc., (iv) we
  establish the generality of the P* model by mapping various existing
  and well known Pilot-Jobs such as DIANE to P*, (v) establish and
  validate the implementation of the TROY API by concurrently using
  multiple distinct Pilot-Jobs.\upp\upp\upp\upp
\end{abstract}

\note{
\section*{Outline}
The primary objectives of this work are:
\begin{enumerate}
\item Establish the need for dynamic execution of applications -
  distributed as well as high-end performance.
\item We define the basic characteristics of the dynamic apps and we
  understand the requirements of dynamic apps need to do in a
  distributed environment.
\item We understand the capability that must be provided by the
  infrastructure to support these application
\item We describe the pilot-job as a good prototype of an abstraction
  that supports dynamic execution
\item We define the characteristics that need to be supported by a
  pilot-job \jhanote{Infrastructure or Application characteristics?}
  \alnote{I think we meant application characteristics}
\item There exist multiple PJ implementations out there but no way to
  compare and contrast. Provide a framework to aid an understanding of
  pilot-jobs and the ability to compare, contrast and understand
  different pilot-jobs.  Provide both a theoretical and empirically
  useful approach to determining which PJ to use
\item Empirical implementation of TROY and demonstration of
  concurrent/interoperation between equivalent but distinct Pilot-Job
  implementation. Highlight unique feature of Troy: User extensible
  and customizable. \jhanote{We should talk about this in light of the
    reviews of the paper with Bishop}
\end{enumerate}
Points 1-3 can go into the beginning of \S2.  Points 4 \& 5 should be
addressed in both introduction (see one of the \jhanote{} above), as
well as in the beginning of \S2. Points 6, 7 are addressed in the
Introduction.
}

\section{Introduction and Overview \upp\upp}

% The ability to use resources dynamically
% and how they are provisioned and federated, it is {\it ipso facto}
% determined that
 
Distributed infrastructure almost by definition is comprised of a set
of resources that is fluctuating -- growing, shrinking, changing in
load and capability.  The ability to utilize such a dynamic resource
pool is thus an important attribute of any application that needs to
utilize distributed infrastructure efficiently; this is in contrast to
a static resource utilization model characteristic of parallel and
cluster computing.  In addition, the evolution or internal dynamics of
an application may vary, thereby changing the resource requirement.
%or utilization ability.  this has the resultant effect of an
%application being dynamic, though dynamic resource execution maybe
%just one of multiple responses.
For example, different solvers, or granularity of existing solvers,
adaptive algorithms and/or implementations, can also provide
applications with an agile execution model, effectively equivalent to
dynamic resource utilization.

The two approaches can be viewed as different approaches to the
dynamic resource problem.
%in one, the supply is kept fixed and the demand is changed to meet the
%supply; in the other the supply changes in response to the demand.
For the purposes of this paper, we will focus on the latter, wherein
the availability and utilization of resources changes.  As a
consequence, we will not consider dynamic execution arising from
internal application changes.  In other words, we will concern and
confine ourselves to the case where applications have the ability to
respond to a fluctuating resource pool, i.\,e., the set of resources
utilized at $T=0$ is not the same as $T>0$.
% Typically, this will involve a single ``large'' distributed
% application, that is comprised of many smaller tasks.\alnote{it
%   would be good if we could relate to our DARE use case here}

\alnote{not sure why do we need week and strong model? Determinism and
  distributed systems don't go together well. Also, I wouldn't claim
  that the prob. of adv. rerv. is 1 (it might be 0.99...). PJ are more
  in line with the weak model. Do we say that Advance Reserv. are
  better?}  \jhanote{Addressed}

% There exist two models of dynamic resource utilization, which we refer
% to as the weak (probabilistic) model of dynamic execution (DE) and the
% strong (deterministic) model of DE. In the former, there exists a
% certain probability to acquire a resource at a given instant of time;
% in the latter the probability of acquiring a resource is by definition
% is 1, but there exists a broad range of times over which this
% probability will first be 1. 

Multiple approaches exist to support dynamic resource utilization, for
example, advanced scheduling (without pre-emption) provides
essentially a guarantee of resource % (probability of 1)
at a sufficiently far out time window.  Another common approach for
decoupling workload management and resource assignment/scheduling are
\emph{pilot-jobs (PJ)}. Along with empirical evidence, our experience
suggests that distributed applications that are able to use tools,
abstractions and services that break the coupling between workload
management and resource assignment/scheduling have been more
successful at efficiently utilizing distributed resources. In
addition, the ability to provide these capabilities in user-space,
leads to the basic concept of a pilot-job.

% \jhanote{Candidate for removal: On reflection, distributed
%   applications that strongly and statically (early) bind to resources,
%   have been less successful at efficiently utilizing distributed
%   resources, as well are unlikely to be able to scale -- either due to
%   hindrances arising from failures, unpredictable system loads and or
%   changing application requirements and resource availability, amongst
%   other factors.}

Interestingly there exist many implementations of the PJ
abstraction, wherein different projects and users have rolled-out
their own. The fact that users have voted with their feet for
PJs reinforces that the Pilot-Job is both an useful
and correct abstraction for distributed cyberinfrastructure; the fact
that it has become an ``unregulated cottage industry'' reaffirms the
lack of common nomenclature, integration, interoperability and
extension.

Our work is motivated by the current status of the usage and
availability of the PJ abstraction vis-a-vis the current
landscape of distributed applications and cyberinfrastructure.
To achieve our objectives, we begin this work with an attempt to
provide a minimal, but complete model -- for the PJ
abstraction, to provide a common and consistent framework to compare
and contrast different PJ implementations/frameworks. This is,
to the best of our knowledge, the first such attempt.

A natural and logical extension of the PJ model, arising from the need to treat
data as a first-class schedulable entity, is a concept analogous to PJ: the
\emph{pilot-data (PD)} abstraction. Given the consistent treatment of data and
compute, as potentially equal components in a framework to support dynamic
resource and execution, we refer this model as the P* model ("P-star").

In \S3 we introduce TROY -- A Tiered Resource OverlaY framework -- as an
implementation of the P* model using the SAGA API. The specific
pilot-job and pilot-data implementations in TROY are referred to as
BigJob and BigData. As we will discuss, consistent with the goals and
aims of SAGA, there can be multiple {\it atomic} instances of BigJob
for different backends.  Thus we posit that the TROY API is a general
purpose API that can be used for all pilot-jobs. Before validating
this claim, in \S4 of this paper, we discuss the mapping of DIANE to
the P* Model and outline briefly how other well known pilot-jobs can
be understood using the {\it vectors}~\cite{dpa_surveypaper} of the P*
Model.

In \S5 we validate the TROY API by demonstrating how DIANE -- an existing and
widely used pilot-job implementation, can be given a well defined API via the
TROY API. To further substantiate the impact of TROY API, we will demonstrate
interoperability between different PJs -- a native SAGA based PJ, referred to as
BigJob-SAGA, and DIANE. We believe this is also the first demonstration of
interoperation of different pilot-job implementations.

% \jhanote{Should we to introduce Dynamic Applications explicitly in the
%   title? Just a question, not a suggestion...}

%\subsection{Introduction \jhanote{SJ, AL}}


%\section{Pilot-* : An abstraction for Dynamic Execution}
%\section{TROY: A Model of Pilot-Abstractions for Dynamic Execution}

\section{P* Model: A Conceptual Framework for Pilot-Abstractions for
  Dynamic Execution \upp\upp}
\label{sec:pilot-model}


% In general, there exist multiple reasons why distributed applications
% have not been able to utilize distributed cyberinfrastructure
% effectively and without immense effort~\cite{dpa_surveypaper}.  At the
% root of the problem is the fact that developing large-scale
% distributed applications is fundamentally a difficult process.

% The range of proposed tools, programming systems and environments is
% bewildering large, making integration, extensibility and
% interoperability difficult. Additionally, 

The uptake of distributed infrastructures by scientific
applications has been limited by the availability of extensible,
pervasive and simple-to-use abstractions which are required at
multiple levels – development, deployment and execution stages of
scientific applications.

% Specifically, Pilot-Jobs support the decoupling of workload
% submission from resource assignment; this results in a flexible
% execution model, which in turn enables the distributed scale-out of
% applications on multiple and possibly heterogeneous resources.

The PJ abstraction has been shown to be an effective
abstraction to address specific requirements of distributed scientific
applications. Although they are all functionally equivalent -- they
support the decoupling of workload submission from resource
assignment, it is often impossible to use them interoperably --- most
PJ implementations however, are tied to a specific
infrastructure, or even to compare and contrast them.

In an attempt to provide a common analytical framework using which
most, if not all commonly used pilot-jobs can be understood, we
present the P* model of pilot-abstractions. The P* model is derived
from an analysis of many pilot-job implementations; based upon this analysis, we
first present the common {\it elements} of the P* model, followed by
a description of the properties that must be assigned and that
determine the interaction of these elements and the overall
functioning of any pilot-job based upon the P* model. Further, we will show
that these elements and interactions can be used to describe pilot-data
model.

\subsection{Elements of the P* Model \upp\upp}

In this section we define the elements of a pilot-job framework. The PJ
framework provides the abstraction of a container for compute ``tasks''
that may be dynamically added to it independently from the underlying resource
pool. It provides a mechanism to decouple compute ``tasks'' from being
hard-coded to a specific ``resource'' or at least to delay that binding. Commonly, PJ 
frameworks consist of the following elements:

\begin{compactitem}
\item \textbf{Pilot-Job (PJ):} The PJ (also referred to as pilot) is
  the entity that actually gets submitted and scheduled on a resource.
  Commonly, the PJ utilizes an agent to manage the set resources 
  allocated to it. Depending on the context, PJ also refers to the concept
  of utilizing a placeholder job to which at a later step the user's
  workload is submitted to as well as to the PJ abstraction.

\item \textbf{Unit of Work (UW):} A unit of work is the workload that
  encapsulates a self-contained piece of work and will be assigned to
  a resource indirectly via a PJ.  The PJ in turn has
  flexibility in determining when and how-many resources the UW will
  receive.

\item \textbf{Unit of Scheduling (US):} US are the entities 
  that get scheduled to the resource, and to which UWs are assigned
  (inside the PJ framework). 
  % A sub-job runs an instance of the application
  %   kernel~\ref{application-program}.
  
% are the unit of works that are sent down to \ref{pilot-agent} to be
%   executed.  Sub-jobs/jobs are used to reserve the resources required
%   by a task.
%\jhanote{I changed the US description significantly} 


% \item \textbf{Resource:} A storage/compute resource that has a common
%   entry point (like a queue). Commonly, a resource is accessed by a 
%   LRM. \alnote{move to impl. section} \label{resource}

\item \textbf{Pilot-Manager:} The PJ manager is responsible for coordinating
	  the different components of the framework, i.\,e.\ the pilots, UWs and 
	  USs.
\end{compactitem}
The application itself is not strictly part of the core P* elements. The term
application generally refers to the upper layer of the stack. The application
utilizes the PJ framework to execute multiple instances of an application kernel
(an ensemble) or alternatively instances of multiple different application
kernels (an workflow). An application kernel is the actual binary that gets run.
To execute an application kernel, an application must define a UW specifying the
application kernel as well as other parameters. This UW can then be assigned to
the PJ framework.
	
As we will see in \S4, the above elements can be mapped to specific
entities in many pilot-jobs in existence and use.

 

% \textbf{Diane Definition of Terms: } The computation consists of many worker
% processes which communicate with one master process (the worker processes do not
% need to share the filesystem nor memory). The ensemble of computation is called
% a run and it consists of many tasks which may be executed in parallel. A task is
% defined as a set of parameters which are produced by the RunMaster (running on a
% master node) and consumed by the WorkerAgent (running on a worker node).
% 
% from 
% 
% DIANE assumes the master-worker computing model (Fig.
% 1). Client sends job parameters to the Planner which partitions
% the job into smaller tasks executed by the Workers. Integrator
% is responsible for merging the results of task execution and
% sending the final job outcome to the Client.


\subsection{Characteristics of P*-Model:\upp\upp}
\label{sec:p_star_elements}

To understand the degrees of freedom that any specific pilot-job
implementation must constraint as well as the functioning of
pilot-jobs implementation, we propose a set of fundamental
properties/characteristics. These characteristics are integral
components of the P* Model, in that they describe the interactions
between the elements, and thus aid in the description of P*
model. 

% Further, these properties are important for the implementation of
% P*.  list several characteristics.

\textbf{Coordination:} describes how the various elements of a PJ framework are
internally coordinated. % The way the coordination between the different elements
% is handled is required to understand a PJ implementation.
A distributed coordination mechanisms, such as master/worker or a set
of distributed software agents, can be utilized. A property for
describing coordination is the point and process of decision making
(e.\,g.\ concerning the binding and scheduling of compute tasks): In
the \textbf{central} model information about available resources (aka
pilots) are collected by a central manager. Decisions are centrally
made by a manager process, which decides which UW is executed on what
resource. In the \textbf{hierarchical} model the decision making
process is divided up into a hierarchy of distributed agents. Each of
them coordinates a defined aspect (e.\,g.\ a certain set of
resources). In the \textbf{decentral} model, control is distributed
among the different components.  PJs with decentralized decision
making often utilize autonomic agents that accepts respectively pull
UWs according to a set of defined policies.



% \begin{compactitem}
% \item \textbf{Decision Making:} 

% \item \textbf{Push versus Pull Model:} We define the terms push and
%   pull based upon the determinism of the binding. If the task is
%   centrally explicitly addressed to a specific pilot, and there is
%   thus no freedom for a pilot to select the task, we speak of a push
%   model. In all other situations, when there is a degree of freedom
%   for the pilots to select a task, we speak of pull. This model is
%   applicable to different aspects of the pilot-job framework (e.g. to
%   task binding, resource binding, resource additions/removals).
%   \alnote{should we move push/pull to communication? It's is also
%     currently references there already}
% 
% \end{compactitem}



    % \item \textbf{Coordination} describes how the various components of 
    % the Pilot-Job framework are coordinated. Possible values for this vector 
    % include data flow, control flow, SPMD (where the control flow is implicit in 
    % all copies of the single program), master-worker (the work done by the 
    % workers is controlled by the master). 
    % \begin{itemize}
    %   \item The data flow describes the flow of messages between the 
    %   components of the framework, e.\,g.\ point-to-point messaging or 
    %   publish-subscribe (see Communication).
    %   \item The control flow describes how the various components of the
    %    framework are managed. In the context of pilot-jobs this particularly 
    %    means how the resource on which a sub-job is executed is determined. 
	% \item Sub-Job pull vs. push applies to control and data flow (different 
	% objects that are pushed/pulled)

\textbf{Communication:} describes the mechanisms for data exchange
between the elements of the framework: e.\,g.\ messaging 
(point-to-point, all-to-all, one-to-all, all-to-one, or
group-to-group), stream (potentially unicast or multicast),
publish/subscribe or shared data spaces. 
		
% \textbf{Dynamic Resources:} Dynamic resources describes the ability 
% 	to acquire and release resources at runtime of the pilot-job.
% 	    \alnote{By definition a BJ should be dynamic. After acquiring additional 
% 	    resources, the external coordination is the same for both BJ.}

\textbf{Binding} defines how and at what time the assignment of a UW
to a PJ is done.  For example, a UW can be bound to a PJ either before
the PJ has in turn been scheduled (early binding), whereas late
binding can occur if the UW is bound after the PJ has been assigned to
a resource.  Depending upon the type of binding supported, the
workload is either specified completely or just partially independent of a (specific)
resource.  \alnote{Question: Is the time of binding
  independent of wether the PJ has resources or not? Late binding
  can also occur if a PJ has resources.}  Furthermore, the time of
binding describes the point at which the binding decision is made,
e.\,g.\ in the case of early binding, the decision could be made by
the application, while in late binding mode the decision is often made by
the middleware framework.

    %   \begin{itemize}
    %       \item In the traditional early binding approach a job carries
    %        exactly one task which is specified at a time of job submission.
    %       \item Partial binding?
    %       \item Late binding is a scheduling and coordination
    %        method, where work is assigned to a job at runtime rather than at
    %        submission time.
    %       \begin{itemize}
    %           \item At sub-job submission 
    %           \item After sub-job submission
    %       \end{itemize}
    % \end{itemize}

% \jhanote{Difference between mapping and assignment}\alnote{in this
%   case both terms are used very synonymous.}

\textbf{Scheduling:} Scheduling describes the process of assigning a US to
physical resources via a pilot. Scheduling has a spatial component (which US is
assigned to which pilot?) and a temporal component (at what time?). To optimize
execution, the manager can group UWs into a US. The different scheduling
decisions that need to be made are representative of multi-level scheduling
decisions that are often required in distributed environments. Commonly a set of
(user-defined) policies (e.g. resource capabilities, data/compute affinities,
etc.) or the implementation of a custom scheduler is supported.

%\begin{itemize}
% \item Multi-level scheduling: In a distributed environment often
%   multiple levels of autonomous schedulers are involved.
% \begin{itemize}
% \item The pilot job is schedules using the local RMS or some other
%   Grid (meta)-scheduler.
% \item UW are scheduled to US on application-level within the
%   pilot-job.
% \end{itemize}
%\end{itemize}

	
	
    % \item \textbf{Agent:} Does every pilot-job implementation have an agent? 
    % What are the tasks of an agent? \alnote{Decision for now: agent is not a 
    % fundamental component. Can we cover this in other components, e.g. discuss 
    % multi-level scheduling in task scheduling.}
    % \begin{itemize}
    %     \item Wikipedia Definition: may refer to one who acts for, or in the 
    %     place of, another, by authority from him; one entrusted with the 
    %     business of another.
    %     \item Difference between implementation and role of agent
    %     \item Agent explicitly in the model? Or requirements that must be met by 
    %     some component in the model?
    %     \item \textbf{Roles of an agent:}
    %     \begin{itemize}
    %         \item manages control flow
    %         \item mapping tasks to resources
    %         \item resource management
    %         \item coordination between different components
    %         \item state management of the complete pilot job state
    %         \item discovering local information
    %         \end{itemize}
    %     
	% \item Infrastructure Independence (Supported middleware types and infrastructures)

% \jhanote{I think upto this point, there should not be mention of
%   BigJob -- we should be talking implementation independent, e.g., must
%   not have sub-job and bigjob agent}


%\subsection*{Summary}

% \jhanote{To generalise the previous point, we need to have a few
%   sentences that aims to put all the elements, characteristics
%   together and presents a {\it unified} view of the working of the P*
%   Model.}
 

\begin{figure}[htbp]
    \centering    
    \includegraphics[width=0.5\textwidth]{figures/pstar_model.pdf}
    \caption{\textbf{P*-Model: Elements and Interactions:} The core of the model 
is the manager who is responsible for managing pilot-jobs and UWs. After a UW is 
assigned to the manager it binds it to a US and schedule the US to an available 
resource. \upp\upp}
    \label{fig:figures_pstar}
\end{figure}

{\it Putting it all together:} Figure~\ref{fig:figures_pstar}
illustrates the interactions between the elements of the P* model. A
typical usage mode consists of the following steps: The application
specifies the number of resources required using a UW description
(step 1). During the instantiation of a PJ and the assignment of
resources to the PJ, pilots are queued and started via the local
resource manager (step 1-4). Having initialized the PJ, UWs can be
assigned to it (step 5).

% The pilot is then executed on a
% certain resource (step 5).  

Most PJs utilize a master/worker coordination model, i.\,e.\ the
manager tightly controls the actions of the worker (the
agent). However, it is feasible to include more logic and decision
capabilities into the agent. The application assigns a UW
to the PJ framework in most cases via the Manager component, which is
then responsible for binding the UW to a US and then scheduling the US 
to a running pilot.  In
the simplest case one UW corresponds to one US. The Manager then binds
a US to specific PJ according to a specified policy or a user provided
scheduler function (step 6). The US gets executed to a physical
resource on which the pilot is operating (step 7).


\subsection{Pilot-Data: Extension of P* to Dynamic Data\upp\upp}
\label{sec:pilot-data}

% \jhanote{We should purely by analogy go with BigData, no? Need a
%   diagram that talks about TROY = BigJob + BigData + ``??''. Need to
%   define ``??''}

% \jhanote{note: DARE == Dynamic Application Runtime Environment: TROY +
%   MapReduce + Other capabilities}

% \jhanote{Ideally the background/underlying theory of Pilot Data/Store
%   should be presented outside of TROY -- not sure this will be
%   possible, or where, possibly in the \S II? Maybe as an explicit \S
%   II-D, where we say, ``having defined a P* model, we extend it to
%   Dynamic Data..''}


%\subsubsection{Overview}
% The concept of correlated access originates in
% Filecules~\cite{Doraimani:2008:FGS:1383422.1383429}.

% \begin{figure}[t]
%     \centering
%         \includegraphics[width=0.4\textwidth]{figures/pilotstore.pdf}
%     \caption{Pilot Data and Store Overview}
%     \label{fig:figures_pilotstore}
% \end{figure}

\jhanote{Motivation for PD: We have discussed the importance of DE for
  compute; there exist similar and important considerations for DE for
  data. Analogous challenges of DE for data as DE for Compute, e.g.,
  importance of issues of (late) binding and decision making} 

Dynamic Execution is at least equal important for data-intensive applications:
applications must cope with various challenging issues e.\,g.\ varying data
sources (such as sensors and/or other application components), fluctuating data
rates, optimizations for different queries, data-/compute co-location etc. Thus,
having defined the P* Model, we propose to extend it to data so as to facilitate
an understanding of DE. This will motivate an analogous abstraction that we call
\emph{Pilot-Data (PD)}. PD provides delayed-binding capabilities for data by
separating the allocation of physical storage and application-level units of
data. Further, it provides an abstraction for expressing and managing
relationships between DU as well as DUs and UWs. These relationships are
referred to as
\emph{affinities}.
\jhanote{difficult to use affinity
  without defining it..}
\alnote{tried to describe affinities a little bit better}

\subsubsection*{Extension of P*-Model Elements}

% The P* model can in many parts be applied to pilot-data.
The elements defined by P* (in section~\ref{sec:p_star_elements}) can be
extended by the following elements, which are by symmetry, the
elements which can be used to define the Pilot-Data abstraction:
\begin{compactenum}[A.]
\item \textbf{Pilot-Data Framework} facilitates the late-binding between data 
 units that can dynamically associated with a Pilot-Data object, which is 
 associated with a physical resources. 
\item \textbf{Data Unit (DU):} The base unit of data used by the framework,
  e.\,g.\ a data file or chunk. 
\item \textbf{Pilot-Data (PD):} Allows the logical grouping of files
  and the expression of data-data affinities. This collection of files
  can be associated with an extensible set of properties. One of this
  property is affinity. A PD containing a set of DUs forms the PJ
  equivalent of a UW.
\item \textbf{Pilot-Store (PS):} Binds a pilot-data object to an actual
  physical resource. A pilot-store object can function as a
  placeholder object that reserves the space for a pilot-data object. A PS  
  facilitates the late-binding of data and resource and is equivalent to the 
  pilot-job.
\item \textbf{Pilot-Data Manager} is analogous to the PJ-Manager responsible for 
  managing DU, PD and PS elements. 
\end{compactenum}

% \jhanote{Are these the elements of Pilot-Data? If so, please make
% connection of above elements to earlier defined elements of P* Model
% explicit, e.g. is Pilot-Data = UW, PS = US etc. Either way, needs
% integration with the following elements} 
\alnote{I would equate a PS with a pilot-job (the placeholder). US is
  IMHO more an internal entity not really exposed to the end-user. The
  PD-Manager framework could e.g. use multiple PS to replicate the
  data}

In summary, a PD is a logical container that describes the properties of a group
of files. A PS is a placeholder reserving a certain amount of storage. By
associating a PD to a PS the data is actually moved to the physically location
associated with the PS.  

%\alnote{insert Millau figure}

\subsubsection*{Extension of P*-Model Characteristics}

\jhanote{Compress into a paragraph which extends the Characteristics
  of P*-Model? i.e., Can Data Characteristics, Access Patterns and
  affinity be demoted from elements, and be put under
  characteristics?}

While the PD model introduces new elements, the characteristics remain to a
great extent the same. The coordination characteristic describes how the
elements of PD interact; the communication characteristic maps the identified
communication patterns to the PD framework. The remaining two characteristics
can be extended with respect to data: Binding describes the intelligent grouping
and assignment of PDs and UWs to facilitate an optimal performance. The
scheduling particularly needs to consider affinities, i.\,e.\ user-defined
relationships between PJs and/or PDs, e.\,g.\ data-data affinities exist if
different DU must be present at the same compute element, data-compute
affinities arise if data and compute must be co-located for a computation, but
their current location is different. The decision of where to place the data and
compute is made by the scheduler based on the defined policies, affinities,
available sowetatic \alnote{what does this word mean?} and dynamic resource information.


PJ and PD encapsulate cross-cutting properties across data and
computation. Both PJ and PD can be used to express different kind of
affinities.  The P* implementation will optimize data- and computing
according to the defined affinities and policies. In the following, we
discuss further implementation considerations for P*.

% \subsection*{Not sure what to with this yet}
% 
% In summary, Pilot-Data is a set of abstractions for expressing data localities 
% and affinities. Pilot-Data can be used to create groups of file clustered
% together using a quantifiable property, such as affinity ($\alpha$)
% e.g., $\alpha = 1.0$ would imply that files are always stored
% together.
% 
% 
% Pilot-Data provides a set of basic operations on top of these file
% groups, whilst Pilot Store is a container that represents a logical
% group of physical files that share the same affinity. Pilot-Store
% containers can be used to express data-data affinities. The
% abstraction supports basic management tasks (create, delete, update,
% move, list).
% 
% The Pilot-Data abstraction serves the following needs:
% \begin{itemize}
% \item Reservation of physical disk space: acquisition of data storage
%   (advanced reservation, place holder)
% \item Virtual destination: dynamically mapping of data to pilot
%   stores.
% \item Runtime environment for $\alpha$ based data
% \item Automatic data partitioning and distribution
% \end{itemize}
% 
% \jhanote{This could be eliminated?} While the application-level
% abstraction enables application developers to model data affinities,
% dependencies etc., the runtime framework will be responsible for
% managing data-compute co-allocations, data-transfers, the dynamic
% expansion of storage pools etc.
% 
% 
% \subsubsection*{PD Impl. Cons.}
% 
% 
% \textbf{Data Characteristics:}
%     \begin{itemize}
%     \item Static data refers to data that is infrequently changed and
% does not need to be moved.
%     \item Dynamic data refers to different spatial and temporal
% properties of data:
%     \begin{itemize}
%     	\item Data that is generated or changing at runtime
% (temporal).
%     	\item Data that is in place or needs to be moved (spatial).
% 	\end{itemize}
% 	\item Streaming data
% \end{itemize}
% \textbf{Data Access Patterns: } While the P* is primarily
% concerned with capturing aspects of distributed coordination, this
% elements is extended to include patterns of data access, e.\,g. co-access or gather/scatter that is e.\,g.\ used by MapReduce.

\subsection{Implementation Considerations\upp\upp}


%In the following we will discuss how we implement
%use the % developed framework
%to develop the
%TROY -- implementation of the P* model.

% \jhanote{I propose we remove ``framework'' ; if anything it should be
%   Model?}

To implement the P* model, there are additional consideration that must be
taken, e.\,g.\ the exposed end-user abstraction and usage model etc. The API
usage model defines how resources are allocated as well as how UWs are
described, assigned and monitored. Further, non-functional properties such as
fault tolerance and security must be considered. PJ implementations differ in
their support for authorization, authentication and accounting. Commonly grid
technologies e.\,g.\ GSI, VOMS, MyProxy, etc. are often used. Further, it can be
differentiated between single- and multi-user: The former PJ runs under the
identity of a single user and is only able to accept jobs from this user, while
the latter is able to accept jobs from different users.

% Pilot-Jobs can be run on different types of homogeneous and
% heterogeneous resources. Generally, the Local Resource Manager (LRM)
% is the gateway to local resources. This can be e.\,g.\ a PBS/Torque or
% WMS service. HPC resources are specifically designed for high-end
% parallel jobs, while HTC resources are particularly suited for
% independent ensemble of tasks. Different applications require a mix of
% HTC and HPC, e.\,g.\ when running an ensemble of MPI jobs.  Different
% workload characteristics (UW heterogeneities, parallelism, UW
% dependencies, etc.) can be supported by a Pilot-Job, e.\,g.\ via
% special information collector and scheduler capabilities.


\section{TROY: A SAGA-based Implementation of the P* Model\upp\upp}

%Pilot-Abstraction for Dynamic Execution 

% \jhanote{Mark to put in a para about how the API implements different
%  Pilot-Jobs} \msnote{In hindsight not sure why it was at this place,
%  and what you expected but here is an attempt:}

% \jhanote{Need to define and motivate TROY better, i.e. TROY: BigJob +
%   BigData etc.}

\begin{figure}[t]
	\centering
		\includegraphics[width=0.35\textwidth]{figures/pstar_troy.pdf}
	\caption{\textbf{TROY -- An Implementation of the P* Model:}  TROY provides 
	a SAGA-based API for the management of PJs and PD. BigJob and BigData are 
	the realization of the actual PJ and PD functionality. BJ-SAGA and BigData 
	heavily rely on SAGA for implementation of the PJ.\upp
	}
	\label{fig:figures_pstar_troy}
\end{figure}

TROY is an implementation of the P* model (section~\ref{sec:pilot-model}) using
SAGA (see Figure~\ref{fig:figures_pstar_troy}). SAGA~\cite{saga_url,saga_gfd90}
provides a simple, POSIX-style API to the most common grid functions (e.\,g.\
file, job management and distributed coordination) at a sufficiently high-level
of abstraction so as to be independent of the diverse and dynamic grid
environments. As shown in Figure~\ref{fig:figures_pstar_troy} TROY is located on
top of the SAGA API and runtime environment, which is utilized for job
management, data transfer and distributed coordination. The TROY API maintains
the SAGA Look \& Feel and was designed to be similar to SAGA in appearance and
philosophy: it re-uses many of the well defined (and standardized) semantics and
syntax of the File, Job and Advert API. Further, we aimed for a simple-to-use
API exposing the least amount of detail necessary.

TROY supports different usage modes i) it provides stand-alone
PJ functionalities, ii) it provides a unified API to various
PJ implementations (e.\,g.\ Condor Glide-In and Diane) and iii)
enables the concurrent usage of multiple PJs
implementations. Further, we show in section~\ref{sec:bigdata} how
TROY is extended to support dynamic data and affinity-based
scheduling.

The TROY framework consists of the API package, the runtime environment and
different PJ implementations referred to as BigJob. The TROY API is based on the
artifacts defined by the P* model \jhanote{Is framework here P* ?}\alnote{yes} defined in
section~\ref{sec:pilot-model}. It defines two description classes that extend
the known SAGA Job Description, the PJ description and the UW description. A
TROY manager represents a pool of resources. Resources can be added by
submitting a pilot-job description to the TROY manager using the
\texttt{add\_resource()} method. Subsequently, UW can be assigned to the TROY
manager. At runtime resources can be added and removed at any time. 
The TROY API is available at~\cite{troy_api}.


% \jhanote{In this paper, we describe the design and implementation of a
%   SAGA-based Pilot-Job, which supports a wide range of application
%   types, and is usable over a broad range of infrastructures, i.e., it
%   is general-purpose and extensible, and as we will argue is also
%   interoperable with Clouds.}
 

% \begin{figure}[htbp]
% 	\centering
% 		\includegraphics[width=0.45\textwidth]{figures/troy.pdf}
% 	\caption{TROY Overview}
% 	\label{fig:figures_troy}
% \end{figure}

\note{
\subsection{What is Unique about a SAGA-based Implementation of the
  P* Model?}
\alnote{these points should be addressed in the introduction above}
\begin{itemize}
\item Consistent with based API and job/file/data model
\item thus programmable (eg. affinity-based PJ) and extensible 
\item ....
\item different usage modes: a stand-alone PJ, API access to other PJ,
  concurrent usage with other PJ (as will be demonstrated)
\end{itemize}
\jhanote{(i) Need for interoperable PJ (ii) focus on TROY API as
  common access layer for different PJ, i.e portability (iii) ability
  to use specific/unique functionality is required. Underlying
  argument is that if you need these properties, then you need to do
  it the SAGA way..}
}
\subsection{BigJob: A Pilot-Job for TROY\upp\upp}

\jhanote{MUST provide SAGA URL for updated BigJob API and
  documentation}

% \jhanote{Alternative title: ``BigJob: TROY Pilot-Job'' ?}

% \jhanote{It is CRITICAL to explain why we need to expose the details
%   of multiple atomic BigJobs to the end-user? Remember part of the
%   whole idea of the exercise is, (i) theory: to provide a framework
%   for understanding any differences, (ii) practice: make all these
%   differences go away from the end user!}  \alnote{Since we were not
%   sure about the term ``atomic'', we could also use base bigjob, or
%   core bigjob}

BigJob (BJ) is a pilot-job implementation within the TROY
framework~\cite{bigjob_web}. It supports a wide range of application
types, and is usable over a broad range of infrastructures, i.\,e.\ it
is general-purpose and extensible.

\begin{figure*}[t]
	\begin{minipage}[b]{0.475\linewidth}
	\centering
	\includegraphics[width=\textwidth]{figures/distributed_pilot_job.pdf}
	\caption{\textbf{BigJob -- SAGA-based TROY Implementation:} BigJob is the implementation of the actual PJ functionality for TROY. Various BJ implementation for different grid and cloud backends exist.}
	\label{fig:figures_distributed_pilot_job}
	\end{minipage}
	\hspace{0.035\linewidth}
	\begin{minipage}[b]{0.475\linewidth}
	\centering
   	\includegraphics[width=\textwidth]{figures/pilot-data-manager.pdf}
    \caption{\textbf{BigData Architecture:} The PD-Manager is exposing the 
	TROY Pilot-Data API. Application can create group of files and assign file
	to storage. The PD manager tracks files and physical locations in the data 
	catalog. The data scheduler optimizes the co-locations of files and compute 
	(in conjunction with BJ). The transfer manager initiates and monitors the 
	actual data movements. \upp\upp}
	\label{fig:pilot-data-architecture}
	\end{minipage}
\end{figure*}

    % \includegraphics[width=0.3\textwidth]{figures/distributed_pilot_job.pdf}
    % 	\caption{SAGA-based TROY Implementation - BigJob}
    % 	\label{fig:figures_distributed_pilot_job}
    % 	\end{figure}
	


% General overview of BJ implementations & P* model
The BigJob implementations are plugged into the TROY runtime environment via a
thin adaptor layer (see Figure~\ref{fig:figures_distributed_pilot_job}).
BigJob-SAGA~\cite{saga_bigjob_condor_cloud} is a implementation of TROY built on
top of the SAGA API. BJ-SAGA utilizes a master/worker coordination model:
The BigJob Manager is responsible for the orchestration and scheduling of the
PJs represented by the BigJob Agent. For submission of the pilots, BigJob-SAGA
relies on the SAGA Job API and thus, can be used in conjunction with different
SAGA adaptors, e.\,g.\ the Globus, PBS, Amazon Web Service and local adaptor.
The SAGA Advert Service and API are used for communication between manager and
agent. The Advert Service exposes a shared data spaces that can be accessed by
the manager and agent, which follow a push/pull communication pattern, i.\,e.\
the manager pushes a US to the Adverts Service while the agents periodically
pull for new data. Results and state updates are pushed back from the agent to
the manager in the same way. The binding between UW and US \& PJ takes places at
submission (early binding). Currently, only a simple binding scheme exists,
i.\,e.\ each UW (a task) is mapped to a single US (a so called sub-job). For
scheduling a simple FIFO scheduler is used.

In addition to the vanilla BigJob SAGA various other BigJob
implementations exist, e.\,g. there are specific BJ flavors for cloud
resources as Amazon EC2 and Microsoft Azure that are capable of
managing set of VMs as well as as a BJ with a Condor Glide-In based
backend. The mentioned BJ implementations are referred to as core
BigJobs. Core BJs are confined to a single resource and don't allow
neither elasticity nor late binding.

%paragraph on dynamic capabilities

In many scenarios it is beneficial to utilize multiple resources, e.\,g.\ to
accelerate the time-to-completion or to provide resilience to resource failures
and/or unexpected delays. The TROY API allows for dynamic resource
additions/removals as well as late binding. The support of this feature
depends on the backend used. To support this feature on top of various core
BigJobs, the concept of a BigJob pool is introduced. A BigJob pool consists of
multiple core BJs. Each BigJob manages a particular resources. An extensible
scheduler is used for dispatching UWs to one of the BJs of the pool (late
binding). For this purpose an extensible scheduler is used. By default a FIFO
and an affinity-aware scheduler are provided. Various other backends support
resource elasticity natively, e.\,g.\ DIANE and Condor. 

% For this purpose, the dynamic BigJob provides an introspection API
% that applications can utilize to query the manager for a list of
% current resources.

%  Thus, BigJob-DIANE provides dynamic
% capabilities out-of-the-box, the traditional BigJob
% implementation~\cite{saga_bigjob_condor_cloud} utilizes two layers: the
% \emph{Core BigJob} and the so called \emph{BigJob Pool} layer. Core BigJobs are
% confined to a single resource and represented by a single master process
% (i.\,e.\ a single BigJob Manager). Generally, Core BigJobs are used to
% encapsulate resource specifics, e.\,g.\ for a certain type of infrastructure
% (Condor, Cloud, etc.). A BigJob Pool manages multiple Core BigJobs.

%\jhanote{Should Table I be zapped?} \alnote{gone}  

% \begin{table}[t]
% \centering
% \begin{tabular}{|p{1.8cm}|p{1.7cm}|p{1.7cm}|p{1.7cm}|}
% 	\hline
% 	&\textbf{UW Binding} &\textbf{Coordina\-tion} & \textbf{Communica\-tion} \\
% 	%\hline
% 	%\textbf{BigJob} & &&&&\\
% 	\hline
% 	BigJob-SAGA &&&\\
% 	\hline
% 	\hspace{2mm} Globus/PBS   &at UW submission
% 									  &Master/Worker &SAGA Advert \\  
% 	\hline
% 	\hspace{2mm} Cloud (EC2)  &at UW submission 
% 									  &Master/Worker &SAGA Advert \\ 
%     \hline
%    BigJob-Condor &after UW submission &Master/Worker &Condor-internal \\
% 	\hline
%  	BigJob-Cloud &at UW submission   &Master/Worker 
% 				 &local Python queue / SAGA Job (SSH adaptor) \\ 
% 	\hline
% 	BigJob-Azure &at UW submission
% 	             &Master/Worker &Azure Storage \\ 
% 	\hline
%     BigJob-Diane &after UW submission  &Master/Worker &CORBA  \\ 
% 	\hline	
% 	% \textbf{Dynamic BigJob} & &&&&\\
% 	% 	\hline
% 	%     Dynamic BigJob &late binding (after job submission) &same as BJ &central decision making &same as BJ &(yes in future)\\
% 	%     \hline
%     %   ManyJob-Cloud &late binding (after job submission) &no &central decision making &SAGA Job (SSH) (push) &no\\
%     % \hline 
%     % ManyJob-Affinity &late binding (after job submission)
%     % &yes &central decision making &SAGA Advert (push/pull) &no\\
%     % \hline
% \end{tabular}
% \caption{SAGA-based Troy Implementations: Characteristics According to
%   Defined Vectors} \label{tab:pilotjob_overview}
% \end{table}		


% Aspects that need to be addressed:
% \begin{itemize}
%     \item Big-Job Agent: capacity (physical size) is a property of an agent. 
% 	cardinality: how many sub-jobs can be managed by an agent?
% 	\item Sub-Job Agent: Agent assignment should be separated from resource 
% 	assignment. Agent has the freedom to assign tasks to sub-job in any way 
% 	it want. Agent can do local decisions.    
%   \item Would it make sense to use the ``internal'' versus
%     ``external'' coordination concept to distinguish sub-job
%     versus big-job agent
% \end{itemize}





% Dynamic BigJob provides the ability to dynamically add and remove resources to a 
% big-job. The API consists of two parts, the resource management and the resource 
% introspection part:
% \begin{itemize}
%     \item \texttt{add\_resource()}: New resources are added by starting a new
%     big-job.There are various flavors of this method:
%     \begin{itemize}
%         \item \texttt{add\_resource(re\-sour\-ce\_dic\-tionary)}: Start another big-job on the resource defined in the \texttt{resource\_dictionary}.
%         \item \texttt{add\_resource(affinity, number\_cores)}: Add another big-job to the specified affinity group.
%     \end{itemize}
%     \item \texttt{remove\_resource(bigjob)}: Removes the big-job from the
%     resources.
% \end{itemize}
% 
% Higher-level wrappers that encapsulate e.\,g.\ the specific resource
% descriptions can be implemented. Further, to implement this dynamic resource
% capabilities it is necessary to provide different dynamic resource introspection
% in the dynamic big-job layer:
% \begin{itemize}
%     \item \texttt{get\_resources()}: returns a list of managed big-job objects.
%      Each big-job object can be queried for it's allocated resources (number 
%      nodes, number cores).
% \end{itemize}


% It uses SAGA BigJob approach to start multiple BigJobs agents 
% whether on a single resource or on multiple resources. And 
% these agents are responsible for pulling the tasks from advert 
% service and run the possible subjobs concurrently or in generations.

% \jhanote{The rationale behind dynamic BJ is that for the same
%   application scenario different BJ with different
%   properties/characteristics may be required. Thus dynamic BJ maybe
%   comprised of either homogeneous or heterogeneous atomic BJs}
% \alnote{did one pass... still not perfect}

% \jhanote{I propose the next 3 subsections -- bigjob-cloud, bigjob-aws
%   and bigjob-azure, be removed}\alnote{gone, there was no news anyhow}

% \subsubsection{BigJob for Cloud Computing}
% 
% \jhanote{Once again -- why do differences in execution details between
%   grids and clouds result in the need for different atomic BJs needs
%   to be explained. Must emphasis that the API remains the same.}
% 
% At the execution level, clouds differ from Clusters/Grids in at least a couple
% of different ways. In cloud environments, user-level jobs are not typically
% exposed to a scheduling system; a user-level job consists of requesting the
% instantiation of a virtual machine (VM). Virtual machines are either assigned to
% the user or not (this is an important attribute that provides the illusion of
% infinite resources). The assignment of job to a VM must be done by the user (or
% a middleware layer as BigJob). In contrast, user-level jobs on grids and 
% clusters are exposed to a scheduling system and are
% assigned to execute at a later stage. Also a description of a grid job typically
% contains an explicit description of the workload; in contrast for Clouds a user
% level job usually contains the container (description of the resource
% requested), but does not necessarily include the workload. In other words, the
% physical resources are not provisioned to the workload but are provisioned to
% the container.  Interestingly, at this level of formulation, pilot-jobs attempt 
% to provide a similar model of resource provisioning as clouds natively offer. 
% 
% \subsubsection{BigJob and SAGA AWS Adaptor}
% 
% BigJob provides support for various cloud computing environment. The SAGA BigJob
% implementation can be used in conjunction with the AWS adaptor for SAGA to run
% on EC2-based cloud infrastructures, such as FutureGrid. However, there are some 
% limitations mainly caused by the restrictions of SAGA/AWS adaptor for the SAGA 
% Job package. The SAGA job service object e.\,g.\ does not provide a mean to 
% specify a set of resources. Using the AWS adaptor it is only possible to utilize 
% a single VM instance, which must be configured prior to the run in a 
% configuration file. If multiple VMs are required, the dynamic BigJob 
% implementation must be used. In this case however, it is still not possible to 
% run MPI jobs across multiple VMs. 
% \smnote {why is it not possible to run MPI jobs across Multiple VM's?} \alnote{MPI jobs are (unless you do something outside of BJ) constraint to run
% on resources managed by a single BJ agent. The agent must generate a nodefile
% from this list of resource it is managing. The agent is not aware of resources
% managed by another BJ)}
% 
% \subsubsection{BigJob Cloud \& BigJob Azure}
% 
% To address this limitation, BigJob-Cloud~\cite{saga_bigjob_condor_cloud} was
% developed. BigJob-Cloud provides an implementation of the BigJob API, which is
% completely independent from the SAGA (and thus, the SAGA AWS adaptor). It
% directly utilizes the Amazon tools to access cloud resources. It can manage
% cluster of VM; for this purpose BigJob provides a rich interface for describing
% cloud resources. For this purpose a Python dictionary is used (see
% section~\ref{sec:api}). The VMs can be managed centrally by the BigJob manager:
% All VMs have a public IP and there is no need to interface with a local resource
% manager (SAGA BigJob e.\,g.\ evaluates the \texttt{\$PBS\_NODEFILE} to obtain a
% list of resources). Thus, it is not necessary to deploy an agent on the VM - all
% necessary metadata can be obtained from the AWS backend. Job are spawned via
% SSH.
% 
% % \begin{itemize}
% %   \item ManyJob is required to manage the set of VMs. The BigJob-Cloud can manage a set of VMs without the need of ManyJob.
% %   \item Bigjob uses advert server for communication between BigJob-agent and BigJob whereas BigJob-Cloud does not use an advert server.
% %   \item Bigjob-cloud does not require SAGA-AWS adaptors as opposed to requirement in original Bigjob. 
% % \end{itemize} 
% 
% 
% BigJob-Azure~\cite{10.1109/CloudCom.2010.85} utilizes a similar approach as
% BigJob-Cloud. It utilizes the Azure REST interface to startup VM Worker Roles.
% However, since Azure does not support SSH access it is necessary to utilize an
% agent-based approach. For communication between the agent and the manager the
% Azure Storage is used.
% 
% \msnote{If BigJob is the atomic unit, it should not differ per
%   backend}\alnote{That's mainly a restriction of the job package which
%   does not really map to AWS. There is no common way in the job
%   package to specify the \# of resources that suppose to be
%   used. Thus, this limitation}


\subsection{BigData: Pilot-Data for TROY\upp\upp}
\label{sec:bigdata}
% \jhanote{theory goes upfront; implementation and architecture stays
%   here} \alnote{ok}

BigData is the SAGA-based implementation of the Pilot-Data abstraction.
Figure~\ref{fig:pilot-data-architecture} gives an overview of the
architecture. The system consists of two components: the PD manager and
the agents deployed on the physical resources. The coordination 
scheme used is master/worker with some intelligent that is lying de-centrally at 
the BD agent. As communication mechanism the SAGA Advert Service is used.


The PD manager is responsible for 1) meta-data management, i.\,e.\ it keeps
track of the pilot stores that a pilot data object is associated with, 2) it
schedules data movements and data replications taking into account the
application requirements defined via affinities and 3) for managing data
movements. The PD scheduler particularly supports affinity-aware scheduling.
Similar to BigJob, an agent on each resource is used to manage the physical
storage on a resource. Both PJ and PD are tightly integrated to efficiently support the compute- and data-related aspects of dynamic execution. 


% For this purpose, the PD and PJ manager work closely together to manage compute-/data-affinities for applications. 
% 
% For this purpose the PJ scheduler was
% extended to support affinities, i.\,e.\ when scheduling UW it considers
% dependencies toward PDs.


% \jhanote{Need to explain/describe architecture of BigData? using the
%   terminology of Section II and P*-Model}

% \begin{figure}[htbp]
%     \centering
%         \includegraphics[width=0.49\textwidth]{figures/pilot-data-manager.pdf}
%     \caption{\textbf{BigData Architecture:} The PD-Manager is exposing the 
% 	TROY Pilot-Data API. Application can create group of files and assign file
% 	to storage. The PD manager tracks files and physical locations in the data 
% 	catalog. The data scheduler optimizes the co-locations of files and compute 
% 	(in conjunction with BJ). The transfer manager initiates and monitors the 
% 	actual data movements. \upp\upp}
%     \label{fig:pilot-data-architecture}
% \end{figure}



% A possible implementation option would be the 
% integration of the PD and BigJob agent, which is particularly useful for 
% managing data-/compute-affinities.

% A core part of the data manager is the data scheduler. The scheduler aims for a
% optimum of data and compute locality for an applications: it selects the storage
% and compute elements for a UW submitted. For this purpose the PJ scheduler was
% extended to support affinities, i.\,e.\ when scheduling UW it considers
% dependencies toward PDs. The manager is then responsible for managing necessary
% file transfer, the pre-fetching of files if applicable, as well as the executing
% the actual UW. For file-transfer management BigData utilizes SAGA and thus, is
% infrastructure independent. It supports all underlying SAGA adaptors (SSHFS,
% GridFTP) and future adaptors such as Globus Online.



% \subsubsection{BigJob and BigData Integration}
% 
% In particular for data-intensive applications data locality is an important
% concern. Different types of affinity, e.\,g.\ data-data or data-compute, exists.
% Dynamic BigJob provides support for data-compute affinities. Each resource
% (i.\,e.\ each big-job) can be assigned to a certain affinity. The affinity-aware
% scheduler then ensures that sub-jobs that demand a certain affinity are only
% executed on resources that fulfill this constraint.


% \subsubsection{Related Work}
% 
% Work on optimizing file transfers: Kosar[2011]
% Work on reliable file transfer: RFT, Globus Online
% 
% \emph{iRods}
% \emph{Stork}
% 
% 
% \emph{BitDew}
% 
% Random Notes
% \begin{itemize}
% 	\item Focus on Desktop Grid
% 	\item Java-based implementation (ie difficult to interface with Python-based PS/SAGA)
% 	\item highly distributed: stable and volatile nodes
% 	\item pull model, i.e. a node pulls for new data
% \end{itemize}
% 
% 
% Mapping to BitDew:
% \begin{itemize}
% 	\item Pilot Store in its current implementation covers Bitdew Data Catalog and Repository
% 	\item For data management and placement the Active Data API and the Bitdew data scheduler could be used
% 	\item Transfer Management is done via SAGA File API	
% \end{itemize}

% How to evolve pilot data/store?
% \begin{itemize}
% 	\item Active management of data (e.g. replication, automatic affinity management) requires an active component:
% 	\begin{itemize}
% 		\item Manager/Agent model as in BigJob?
% 		\item Who runs active components? Started as part of batch job or separate install/start?
% 	\end{itemize}
% \end{itemize}
% 
% Questions:
% \begin{itemize}
%     \item How should
%     we store data in order to effectively cope with non-uniform demand for
%     data? 
%     \item How many copies of popular data objects do we need? 
%     \item Where should we store them for effective load balancing?
% \end{itemize}

% \textbf{TODO/Future Work}
% 
% 
% Limitations:
% \begin{itemize}
%     \item No active agent that monitors state of files
%     \item No placement policy support or autonomic behavior
%     \item Infrastructures generally expose insufficient locality/topology information
%     \item Compute – Data Affinity: Dynamic BigJob with affinity only provides a very coarse-grained affinity
%     \item No policy for what’s happening if data is not available in right location:
%     \begin{itemize}
%         \item Run anyways – affinity is just an hint
%     \end{itemize}
%     \item When to move pilot stores? Move or copy?
%     \item Move data to compute or visa versa?
%     \item Data Replication: Identification of the same file: logical filename -> physical files. Manage replication process (consistency!)
% \end{itemize}

%%%%%%%%%%%%%%%%%%%%%%%%%%%%%%%%%%%%%%%%%%%%%%%%%%%%%%%%%%%%%%%%%%%%%%

\section{Understanding Other Pilot-Jobs\upp\upp}


\jhanote{Depending upon where the TROY API is discussed we can have
  two ways forward. If TROY API is discussed in \S3, then we go for
  Mode I, where Mode I: The aim of this section is to show: (i) that
  our P* Model can be used to explain/understand DIANE, (ii) Show that
  the TROY API can be used to marshall Diane stand-alone, (iii) Using
  TROY API, both BigJob and Diane can be used standalone}

\jhanote{If TROY API is discussed in \S5, then we go for Mode-II,
  where Mode II: The aim of \S4 is to show: (i) that our P* Model can
  be used to explain/understand DIANE.  Then in \S5, after having
  discussed TROY API we, (ii) Show that the TROY API can be used to
  marshall Diane stand-alone, (iii) Using TROY API, both BigJob and
  Diane can be used stand alone}

\jhanote{I think there was agreement to go with Mode II}

As more applications take advantage of dynamic execution, the Pilot-Job concept
has grown in popularity and has been extensively researched and implemented for
different usage scenarios and infrastructure. There is a variety of PJ
implementations: Condor Glide-In~\cite{condor-g}, SWIFT~\cite{Wilde2011},
DIANE~\cite{Moscicki:908910}, DIRAC~\cite{1742-6596-219-6-062049},
PanDA~\cite{1742-6596-219-6-062041}, ToPoS~\cite{topos},
Nimrod/G~\cite{10.1109/HPC.2000.846563}, Falkon~\cite{1362680} and
MyCluster~\cite{1652061} to name a few. The aim of this section is to show that
our P* Model can be used to explain/understand some of these PJ implementations,
in particular DIANE as well as Condor Glide-In and Swift.
Table~\ref{table:bigjob-saga-diane} shows how the elements P* model can be
mapped to these frameworks. Table~\ref{table:pilot-job-comparison} compares the
characteristics of the four PJ implementations.


% In addition to the three Pilot-Job framework discussed in this section, various
% other frameworks exist.
% \begin{itemize}
%     \item MyCluster~\cite{1652061} enables the 
% 	creation of a Condor, PBS or SGE clusters on-demand.
%     \item Falkon~\cite{1362680} is a Pilot-Job framework that emphasizes the 
% 	performance of its task dispatcher.
%     \item Nimrod/G~\cite{10.1109/HPC.2000.846563}
%     \item DIRAC~\cite{1742-6596-219-6-062049} is another pilot-job framework 
% 	used by the LHCb community.
%     \item ToPoS~\cite{topos} is a REST-based web service primarily designed with 
% 	respect to parameter sweep applications. Internally, ToPoS utilizes PJ 
% 	capabilities to efficiently manage resources.
%     \item The Production and Distributed Analysis System 
% 	(PanDA)~\cite{1742-6596-219-6-062041} is the workload management system of 
% 	the ATLAS experiment. PanDA utilizes multi-user PJs for resource management. 
% 	The PJ component is built on top of Condor-G and referred to as AutoPilot. 
% 	It can also be used independently of the ATLAS environment. 	
% \end{itemize}



\begin{table*}[t]
\centering
\begin{tabular}{|p{2.5cm}|p{3cm}|p{3cm}|p{3cm}|p{3cm}|}
\hline
\textbf{P* Element} &\textbf{BigJob-SAGA} &\textbf{DIANE} &\textbf{Condor} 
&\textbf{Swift Coaster}  \\
\hline
Manager &BigJob Manager & RunMaster & condor\_master, condor\_collector, condor\_negotiator, condor\_schedd &Coaster Service\\ 
\hline
Pilot-Job &BigJob Agent  & Worker Agent &condor\_master, condor\_startd &Coaster Worker\\
\hline
Unit of Work &Task &Task &Job &Application Interface Function (Swift Script)\\
\hline
Unit of Scheduling &Sub-Job &Task &Job &Job\\
% \hline
% Dynamic Resources &no/yes &yes (AgentFactories)\\
\hline
\end{tabular}
\caption{P* Elements and Pilot-Job Frameworks} \label{table:bigjob-saga-diane}
\end{table*}

\subsection{DIANE\upp\upp}

% Coordination and Communication
DIANE~\cite{Moscicki:908910} is a task coordination framework, which was
originally designed for implementing master/worker applications, but also
provides PJ functionality for job-style executions. DIANE utilizes a single
hierarchy of worker agents as well as a PJ manager referred to as
\texttt{RunMaster}. Further, there is ongoing work on a multi-master extensions.
For the spawning of PJs a separate script, the so-called submitter script, is
required. For the access to the physical resources the GANGA 
framework~\cite{DBLP:journals/corr/abs-0902-2685} is used. GANGA
provides a unified interface for job submissions to various resource types,
e.\,g.\ EGI resources or TG resources via a SAGA backend. Once the worker agents
are started they register themselves at the RunMaster. In contrast to BJ-SAGA, a
worker agent generally manages only a single core and thus, by default is not
able to run parallel applications (e.\,g.\ based on MPI). In contrast to
BJ-SAGA, DIANE worker agents possess less local autonomy and are directly
controlled by the master. BJ-SAGA delegates the management of the local
resources from the BJ manager to the BJ agent. The agent collects information
about local resources and manages the allocation of these resources locally. For
communication between the RunMaster and worker agents point-to-point messaging
based on CORBA (omniORB) is used. CORBA is also used for file staging, a feature
currently not fully supported by BJ-SAGA.

% Binding 
DIANE is primarily designed with respect to HTC environments (such as
EGI~\cite{egi}), i.\,e.\ one PJ consists of a single worker agent with the size
of 1 core. BJ-SAGA in contrast is designed for HPC systems such as TG, where a
job usually allocates multiple nodes and cores. To address this issue a
so-called multinode submitter script can be used: the scripts starts a defined
number of worker agents on a certain resource. However, UWs will be constrained
to the specific number of cores managed by a worker agent. A flexible allocation
of resource chunks as with BJ-SAGA is not possible. By default a UW is mapped to
a US; application can however implement smarter allocation schemes, e.\,g.\ the
clustering of multiple UWs into a US.

%Scheduling
DIANE includes a simple capability matcher and FIFO-based task scheduler.
Plugins for other workloads, e.\,g.\ DAGs or for data-intensive
application, exists or are under development. The framework is extensible:
applications can implement a custom application-level scheduler.


%Other impl. related issues: FT and security
DIANE is as BJ-SAGA a single-user PJ, i.\,e.\ each PJ is executed with the
privileges of the respective user. Also, only UWs of this respective user can be
executed by DIANE. For communication, DIANE currently relies CORBA/TCP. The 
implementation of GSI is on TCP-level is possible, but currently not yet 
implemented. Further, DIANE supports fault tolerance: basic error detection and propagation mechanisms are in place. Further, an automatic re-execution of UWs is possible.


\subsection{Condor Glide-In\upp\upp}

Condor GlideIn pioneered the Pilot-Job concept~\cite{condor-g}. The pilot-job is
actually represented by a complete Condor pool that is started using the Globus
GRAM service of a resource (this feature is referred to as Condor Glide-In);
subsequently jobs can be submitted to this Glide-In pool using the standard
Condor tools and APIs. Condor utilizes a master/worker coordination model. The
PJ manager is referred to as the Condor Central Manager. The functionality of
the Central Manager is provided by several daemons: the conder\_master that is
generally responsible for managing all daemons on a machine, the
conder\_collector which collects resource information, the conder\_negotiator
that does the matchmaking and the condor\_schedd that is responsible for
managing the binding and scheduling process. Condor generally does not
differentiate between workload, i.\,e.\ UW, and schedulable entity, i.\,e.\ US.
Both entities are referred to as job. However, it supports late binding,
i.\,e.\ resources a job is submitted to must generally not be available at
submission time. The scheduler matches the capabilities required by a UW to the
available resources. This process is referred to as matchmaking. Further, a
priority-based scheduler is used. For communication between the identified
elements Condor utilizes point-to-point messaging using a binary protocol on top
of TCP.

Different fault tolerance mechanisms, such as automatic retries, are supported.
Further, Condor supports different security schemes: for authentication it
integrates both with local account management systems (such as Kerberos) as well
as grid authentication systems such as GIS. Communication traffic can be
encrypted.



\subsection{SWIFT and Coaster\upp\upp}

SWIFT~\cite{Wilde2011} relies on its scripting language to describe abstract
workflows and computations. The language provides among many things capabilities
for executing external application as well as the implicit management of data
flows between application tasks. For this purpose, SWIFT formalizes the way that
applications can define data-dependencies. Using so called mappers dependencies
can be easily extended to files or groups of files. The runtime environment
handles the allocation of resources and the spawning of the compute tasks. Both
data- and execution management capabilities are provided via abstract
interfaces. SWIFT supports e.\,g.\ Globus, Condor and PBS resources. The pool of
resources that is used for an application is statically defined in a
configuration file. While this configuration file can refer to highly dynamic
resources (such as OSG resources), there is no possibility to manage this
resource pool programmatically. By default a 1:1 mapping for UW and jobs is
used. However, SWIFT supports the clustering of UWs into a large US as
well as PJs for which the term Coaster~\cite{coasters} is used. Coaster
relies on a master/worker coordination model; communication is implemented using
GSI-secured TCP sockets.



\jhanote{It should probably be Coasters -- which is their notion of a pilot-job.
Just to keep life interesting, they call it head-job and not pilot-job!
\url{http://www.ci.uchicago.edu/swift/guides/release-0.92/userguide/coasters.php
}}


\alnote{SWIFT eval: no standard resource abstraction (SAGA), proprietary 
language (not Python), TODO: check how coasters work! 1 coaster == 1 Condor-G 
job?}


%\subsubsection*{Other Pilot-Jobs and Conclusion}

\jhanote{Can we add some structure to these *other* PJ.. this will be
  ambitious and time-consuming, but if we can, that'll be (i) a great
  service to the community, (ii) a strong intellectual addition to the
  paper by virtue of validation of the P*-model} \alnote{which are the minimal P* elements and characteristics we should discuss here?}



\begin{table*}[t]
\centering
\begin{tabular}{|l|p{2.5cm}|p{2.5cm}|p{2.5cm}|p{2.5cm}|}
	\hline
	\textbf{P* Characteristic}
	&\textbf{SAGA BigJob} &\textbf{DIANE} &\textbf{Condor Glide-In} &   
	\textbf{SWIFT Coaster} \\ \hline
End User Environment &API &API and Master/Worker Framework &CLI Tools &Swift script\\ \hline

Coordination &Master/Worker (push) &Master/Worker (pull/push) &Master/Worker &Master/Worker \\ \hline
	
Communication &Advert Service &CORBA &TCP &GSI-enabled TCP \\ \hline

UW Binding &Early/Late &Late &Late &Late\\
% \hline
% MPI/Multinode Applications &yes &no (yes with custom implementation of ApplicationWorker)\\
\hline
UW Scheduling &FIFO, custom &FIFO, custom &Matchmaking, priority-based scheduler 
&Load-aware scheduler, UW clustering\\
\hline

Security &Middleware dependent (GSI, Advert DB Login) &Middleware dependent (GSI) &Multiple (GSI, 
Kerberos) &GSI\\ \hline

Resource Abstraction &SAGA &Ganga/SAGA &Globus &Resource Provider API/Globus CoG 
Kit \\ 
\hline
Agent Submission &API &Ganga Submission Script &Condor CLI 
&Resource Provider API\\
% \hline
% Application Interfaces &Big-Job/Sub-job Management &Big-Job/Sub-job 
% Management\linebreak[4] Master/Worker API (\texttt{ITaskScheduler}, 
% \texttt{IApplicationManager}, \texttt{IApplicationWorker}) &&\\
\hline
Fault Tolerance &Error propagation &Error propagation, Retries &Error propagation, Retries &Error propagation, retries, replication\\
\hline
	
\end{tabular}
\caption{P* Characteristics and Pilot-Job Implementations}\label{table:pilot-job-comparison}
\end{table*}

\section{Implementations and Results\upp\upp}

% In this section we discuss the TROY API. Further, (i) we show that the TROY API
% can be used to marshal DIANE stand-alone as well as (ii) that the TROY API can
% be used stand alone with both BigJob and DIANE concurrently. 

\jhanote{The experiment section needs tighter and more focussed
  writing in general}

In these experiments we execute BFAST~\cite{bfast2009} - a genome
sequencing application. To manage the workflow, we use the
DARE~\cite{dare-tg11} framework, which in turn uses TROY.  We utilize
the LONI\jhanote{reference} resource Oliver and a total of 16 cores
distributed across four nodes.

In these experiments, we use two BigJob implementations BJ-SAGA and
BJ-DIANE. We define eight UWs that run BFAST. The UWs are run on three
different backends: (i) BJ-SAGA only, (ii) BJ-DIANE only and (iii)
both BJ-SAGA and BJ-DIANE. In case (iii) on each backend four UWs are
executed.

For each scenarios respectively two PJ descriptions are
prepared. Depending on the specified backend the respective BJ
implementation is used for dispatching the actual pilots. While
BJ-SAGA utilizes one BJ agent on the resource, BJ-DIANE requires the
spawning of one worker agent per UW that must be executed in
parallel. \jhanote{What is executed in parallel to what?} The TROY API
marshals these differences: for each backend the right number of
agents is started and the requested number of resources is allocated.
\jhanote{What are we trying to say here?  Is it that although the API
  is similar, semantics of implementation and execution remain
  different, and that TROY backend handles them?}

Each BFAST instance requires two cores; a total of eight UWs
allocating two cores each is specified and assigned to TROY. Each UW
is associated with a set of input files. After assignment of the UW to
the PJ manager, the TROY runtime takes care of binding and scheduling
the UWs to the pilots. The application can monitor the state of the
PJs and UWs using the TROY API.


% To perform the above experiments we built an application \smnote{this
% application is DARE. do we need mention it here?} using TROY API to submit Bfast
% UW's with different input files to different backend implementations. The UW's
% are completely handled and submitted to different backends by TROY and the
% co-ordination of UW's was made simple for the application. Further, applications
% can also continuously monitor status of remote agents and UOWs.

% Further, it shows that both PJ implementations can be used
% concurrently. This is particularly useful if the time-to-completion can be
% reduced in cases where resources on another infrastructure are available.

\begin{figure}[t]
	\centering
		\includegraphics[width=0.4\textwidth]{perf/perf-bfast-bj.pdf}
	\caption{\textbf{Performance TROY with BJ-SAGA and BJ-DIANE:} Running BFAST 
	on a 4 node cluster with a total of 16 cores. The time-to-completion for 
	BJ-DIANE is higher than for BJ-SAGA mainly due to the higher startup 
	time required and some light runtime overhead caused by the additional 
	agents required on the resource.\upp \upp}
	\label{fig:perf_perf-bfast-bj}
\end{figure}



Figure~\ref{fig:perf_perf-bfast-bj} shows the result of the
experiment. The time-to-completion for BJ-DIANE is about 207\,sec
longer than for BJ-SAGA. The main contributor for this increased
runtime is the deployment time required for DIANE. In a multi-node
setup multiple work agents must be used (8 in this case).  For each
worker agent DIANE must be downloaded, installed and started
separately. In total this requires about 178\,sec. Further, each DIANE
worker agent is queued as a separate job at the local resource manager
-- this contributes the the higher deviation in the measured
runtimes. While BJ-SAGA requires a pre-run installation on each site,
it only shows a startup time of 28\,sec. Additionally, we also
observed a runtime overhead of about 17\,sec for the BJ-DIANE
scenario. This overhead is likely caused by the additional agents
required.

% % In the case of BJ-DIANE backend Multinode submitter was not used/not % yet. As
% a result it submits job request's for each node separately and there % also some
% delay between actual launch and startup time of UOW. Further, DIANE % was
% installed separately for every agent/node requested.

Finally scenario iii) demonstrates that two BJ implementations
(BJ-SAGA and BJ-DIANE) can be utilized concurrently using the TROY
API. The performance in this scenario is slightly better than in the
DIANE only case, mainly due to the fact that only four DIANE worker
agents need to be started. Also, only half of the UWs are executed on
a DIANE node and thus, show a longer runtime. While there a some
limitations in the current BJ-DIANE implementation, the aim of this
experiment is to emphasizes the possibilities that the TROY API
provides to dynamic applications. TROY enables applications to utilize
a dynamic resource pool consisting of resources of different
infrastructures, e.\,g.\ EGI and TG/XD resources. Dynamic applications
can utilize the elasticity of the TROY resource pool e.\,g.\ to
improve the time-to-completion and/or to scale the accuracy of their
computations.


% Further, TROY hides many of the complications with
% various pilot jobs like multiple node backend agents, how UW's distribution to
% various backends. Thus, this enables users of this API to concentrate more on
% designing their applications instead worrying about the various configurations
% of the pilot jobs. 

\upp


\section{Conclusion and Future Work\upp\upp}

The P* model provides a common framework for describing and
characterizing PJ. We validate the P* model by
demonstrating % The P* model was used to demonstrate
that the most widely used PJ implementations, viz., DIANE, Condor
Glide-In and SWIFT can be compared, contrasted and analyzed using this
framework, i.\,e.\ both their architecture and their communication and
coordination schemes.  Furthermore, we established PJ and PD as
abstraction for supporting dynamic execution by decoupling workload
and resource assignment/scheduling.  TROY is an implementation of the
P* model that captures the commonalities between the different PJ
implementations via a common API.

% they share different important properties with respect to the commonly
% used communication and coordination schemes.

In the future, we will align the TROY API with the emerging SAGA
Resource Management API~\cite{saga_rm}. We will further refine the
TROY API, e.\,g.\ by generically supporting different security models
via a context API (similar to the SAGA Context API). Further, we will
investigate autonomic resource management strategies, e.\,g.\ by
deploying more decentral decision logic into the agents of our
framework.

\note{do something with BigData, make sense, improve the world}

\section*{Acknowledgements\upp\upp}
\footnotesize{This work is funded by Cybertools project
  (http://cybertools .loni.org; PI Jha) NSF/LEQSF
  (2007-10)-CyberRII-01, HPCOPS NSF-OCI 0710874 award, and NIH Grant
  Number P20RR016456 from the NIH National Center For Research
  Resources. Important funding for SAGA has been provided by the UK
  EPSRC grant number GR/D0766171/1 (via OMII-UK).  MS is sponsored by
  the program of BiG Grid, the Dutch e-Science Grid, which is
  financially supported by the Netherlands Organisation for Scientific
  Research, NWO. SJ acknowledges the e-Science Institute, Edinburgh
  for supporting the research theme. ``Distributed Programming
  Abstractions'' \& 3DPAS. We thank J Kim (CCT) for assistance with
  the DNA models.  SJ acknowledges useful related discussions with Jon
  Weissman (Minnesota) and Dan Katz (Chicago). This work has also been
  made possible thanks to computer resources provided by TeraGrid TRAC
  award TG-MCB090174 (Jha). We thank Ole Weidner for useful feedback.}
\bibliographystyle{plain}
\bibliography{pilotjob,saga.bib,saga-related}
\end{document}


\note{Facilities provided include the creation of a PJ, insertion of
  tasks, and attachment to a CPU resource pool for late-binding task
  execution.}  \note{Tasks are ultimately loaded onto specific
  resources using the pilot-job and late-binding. In other words}
\note{PJ provides a mechanism to decouple “task coordination” from
  “resource mapping”.}
