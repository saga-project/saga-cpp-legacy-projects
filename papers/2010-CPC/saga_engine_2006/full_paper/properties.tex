% $Header: /projects/VU-SAGA/Papers/saga_engine_2006/properties.tex,v 1.4 2006/08/13 18:01:00 hkaiser Exp $

The paper so far motivated the design objectives of the SAGA C++
Reference implementation, and described several implementation
techniques used to meet these design objectives.  This section will
summarize the resulting properties of the SAGA implementation from an
end user perspective, and will motivate further developments and
extensions.


\subsection{Uniformity over Programming Languages}
\label{ssec:lang}

 The SAGA API specification is language independent -- however, it is
 a declared goal to define language bindings which both, if possible,
 provide a language-native look\,\&\,feel to the user of the API, and
 strive for syntactic and semantic similarity over all SAGA language
 bindings.  One of the consequences of that goal is that the API
 specification does not use templates, as that was thought too
 difficult to express uniformly over many languages.  Also, the
 specification tries to be concise about object state management, and
 hence also expresses semantics for shallow and deep copies.

 Our implementation follows the specification closely, naturally.  Furthermore,
 it is designed to accommodate wrappers in other languages, so as to
 provide the same semantics, and similar look\,\&\,feel to other
 language bindings.  In fact, a Python wrapper for our library is in
 alpha stadium, and we consider similar thin wrappers to provide
 bindings to C, Fortran, Perl, and possibly others.

 From a different point of view, we find it extremely convenient to be
 able to implement \I{adaptors} in different languages as well.  The
 Grid Application Toolkit (GAT, \cite{gat}), a C-based API predecessor
 of SAGA,  already allows adaptors in different languages, and we
 consider to implement similar mechanisms to allow Python or C based
 adaptors for our implementation as well.  In particular Python based
 adaptors have shown to be extremely useful for rapid prototyping of
 middleware bindings for GAT.

\newpage
\subsection{Genericity in respect to Middleware, and
            Adaptability to Dynamic Environments}

 The dynamicity of grid middleware was mentioned a number of times
 already, as it \I{is} a very central and dominating property of grid
 environments.  The described adaptor mechanism used in our
 implementation to bind to diverse middleware binding addresses that
 point.  Additionally, late binding, fall back mechanisms, and
 flexible adaptor selection allows for additional resilience against
 an dynamic and evolving run time environment.  It must be noted,
 however, that adaptors need to deploy mechanism like resource
 discovery, and need to implement fully asynchronous operations, if
 the complete software stack is to be able to cope with dynamic grids
 -- our SAGA implementation usability will be severely impacted if the
 quality of adaptors undermines the libraries mechanisms.


\subsection{Modularity makes the Implementation \\Extensible}

 Section~\ref{ssec:macros} has shown in some detail that our SAGA
 implementation is well able to cope with the expected evolution and
 extension of the SAGA API.  On the other hand, the adaptor mechanism
 allows for easy extensions of the library, to provide additional
 middleware bindings.  In fact, the major future work for our SAGA
 implementation will be to provide multiple sets of stable adaptors
 for the major grid environments.  We expect, however, that that task
 requires massively more effort than the implementation of the
 presented library, and we very much hope for the support of grid
 middleware vendors to provide, and (above all) maintain these
 adaptors.  Ideally we would actually wish that middleware vendors
 \I{implement} adaptors to our SAGA implementation, and deliver them
 as part of their client side software stack.  If that scenario will
 ever come to pass remains to be seen -- it would certainly help to
 get the SAGA API a very wide user community.


\subsection{Portability and Scalability}

 Heterogeneous distributed systems naturally require por\-table code
 bases.  We think that our library implementation is in fact very
 portable, as we strictly adhere to the C++ standard and portable 
 libraries.  In fact,
 we currently develop the library on Windows and Linux concurrently,
 so we are confident that we are able to cover the two
 major target platforms without any problems -- but we don't expect
 (and currently don't encounter) any problems on other platforms.  It
 must be noted, however, that the portability of our SAGA implementation
 depends on the portability of the adaptors, and hence on the portability 
 of the grid middleware client interfaces, being the much greater problem 
 if compared to the library code itself.

 Distributed applications are quite often sensitive to sca\-lability
 issues, in particular in respect to remote communications.  As SAGA
 introduces a number of communication mechanisms, scalability concerns
 are naturally also raised in respect to SAGA implementations.
 Firstly it must be said, that the SAGA API is not targeting
 high performance communication schemes, but rather tries to stick to
 simple communication paradigms -- in no sense does SAGA intent to
 replace MPI or other distributed communication libraries.  Having
 said that, our design allows for zero-copy implementations of the
 SAGA communication APIs, and also allows for fast asynchronous
 notification on events -- both are deemed critical for implementing
 scalable distributed applications.  
 
 %Again, our library depends on
 %hi-quality adaptor implementations to utilize these features.


\subsection{Simplicity for the End User}

 SAGA is \I{designed} to be simple to use.  However,
 simplicity of use of an API is not only determined by its API
 specification, but also by its implementation: simple deployment 
 and configuration, resilience against lower level failures,
 adaptability to diverse environments, stability, correctness, and
 peaceful coexistence with programming paradigms, tools and libraries
 are some of the characteristics which need attention while
 implementing the SAGA API.  
 
 It is a challenging task to keep a library implementation as this one
 simple in itself, meaning to have readable code, with a limited
 amount of magic.  Again, a very modular approach helps here very
 much, in out opinion.  For example, it is very simple to hide the
 generic call routing, or the adaptor selection, in the engine module,
 as these features are not usually exposed to the user or adaptor
 programmer.  However, we think that modeling these central properties
 as modules increases the readability and maintainability of the code
 significantly.  

 The SAGA API implicitly introduces a concurrent programming model,
 due to its notion of asynchronous operations, or tasks.  The C++
 language binding of the API, and our implementation, allows to combine
 that model with arbitrary mechanisms for managing concurrent program
 elements (i.e. to ensure object state consistency in all
 circumstances, to ensure thread safety, and to allow for application
 level semaphores and mutexes).


% \subsection{Meeting the Design Objectives}
% 
% Our most important objective was to design a state-of-the-art grid application 
% framework satisfying the majority of user-needs while staying as flexible 
% as possible, both towards further API extensions (due to extensions of the SAGA API), 
% towards emerging grid middleware frameworks, services, and standards, as well as 
% towards its usability in different usage scenarios as synchronous, asynchronous,
% task based etc.
% 
% %Simplicity: easy to use, install, administer and maintain 
% \subsubsection{Simplicity for the End User}
% A major design goal of our implementation was \emph{simplicity} -- not
% for the implementation itself, but for the users of this
% implementation. In particular, this SAGA implementation should,
% be easy to install, use, administer and maintain, especially and even by 
% non-computer scientists. In our experience, this kind of simplicity is 
% a major precondition for a broad acceptance amongst grid users. 
% 
% The interfaces exposed towards the user of our library are defined by the
% SAGA strawman API~\cite{saga_spec} and hence by definition very simple, compared
% to complicated grid middleware interfaces, hidden by our implementation.
% 
% The SAGA C++ reference implementation potentially provides access to every
% possible grid middleware, by abstracting from middleware specific details.
% Only relevant details are defined in the SAGA API specification and our implementation 
% links the abstract and middleware independent specification to lower level
% middleware frameworks. Our implementation meets this requirement by extensively using
% the adaptor pattern~\F{cite Gamma et al.}. Middleware frameworks make their functionality
% available through adaptor libraries used by the SAGA C++ core on the fly
% by using a plug in mechanism.
% 
% During compile time it is normally not known which grid middleware frameworks will be 
% available at runtime. Furthermore, because of the highly dynamic nature of 
% distributed systems we cannot rely on the existence of required middleware services. 
% The only way to deal with this kind of dynamics is to implement trial and error 
% mechanisms ensuring a maximum of coverage of potential functionality: i.e. the only way 
% to ensure a certain middleware service is available is to try a required operation and
% verify it actually succeeded.
% 
% A dynamic middleware recognition and selection was required. Our implementation
% analyses available grid middleware frameworks and their adaptors and loads them dynamically
% at the moment of function execution. The most suitable
% adaptor is chosen to execute a given operation. As grid environments are dynamically
% changing, execution failures may occur frequently. If a given grid middleware fails to
% execute a given operation, the next most suitable adaptor is selected and executes the
% operation of interest. 
% 
% This whole mechanism is absolutely transparent for the library user, only the failure
% of all available frameworks to execute a given operation is reported to the invoking
% application.
% 
% The same dynamic plug-in mechanism provides also the means for simplicity in terms
% of maintenance and administration. As new middleware framework can be added dynamically
% -- even during runtime of an application -- no recompilation is required. The same holds
% for SAGA scope extensions. In both cases additional shared libraries are copied into
% the current installation and recognized and used by the SAGA core on the fly.
% 
% 
% % should this go into this paper? Is more a SAGA feature, than a feature of this impl? --STH
% %\subsubsection{Uniformity over Programming Languages}
% %As every scientist should potentially be able to use our
% %implementation, the target user group originates from a large variety
% %of research domains, each with their own programming habits and
% %language preferences. Hence it is highly desirable to provide the
% %widest possible range of \emph{uniformly} supported programming
% %language bindings. So far, our implementation provides bindings for
% %C++ and Python.  Language bindings for FORTRAN, Perl and others are
% %planned.  A separate Java reference implementation is currently
% %developed by the Open Middleware Infrastructure Institute (OMII-UK).
% 
% 
% %Uniformity: provides support for different application programming languages 
% \subsubsection{Uniformity over Programming Languages}
% \F{Explain constraints resulting from this SAGA requirement: no templates in the API,
% shallow copy semantics of API objects, ...anything else? -- HK}
% 
% %Genericity: adds support for different grid middleware, even
% %concurrent ones
% \subsubsection{Genericity in respect to grid Middleware}
% As different user groups use different grid middleware frameworks, another objective was 
% to enable the usage of these different middleware frameworks of choice through the means 
% of SAGA, while benefiting from its simplified and easy to use API in a \emph{generic}
% way.  Our implementation allows to use multiple grid middleware frameworks in parallel, 
% and even to chose the most appropriate one on the fly, to execute any required operation.
% 
% As already outlined in the above paragraph, our implementation is able to deal with
% every kind of middleware, if an appropriate adaptor (as shared library) exists, which
% is then used dynamically.
% 
% The adaptor loading and selection is of generic nature, both in terms of high-level
% functionality (file transfer as well as job submission etc.) as well as in terms of
% middleware support. This leads hence to a mapping of $n$ sets of API functionality
% to $m$ different middleware frameworks.
% 
% %Transparency: contains mechanisms for the same application (source)
% %code to run on a variety of systems ranging from laptops to HPC
% %resources Adaptability: orientation towards dynamic and adaptive
% %grid-aware applications Modularity: provides a framework easily
% %extendable 
% %\newpage
% 
% %Adaptability: orientation towards dynamic and adaptive grid-aware applications 
% \subsubsection{Adaptability to Dynamic Environments}
% 
% grid computing is defined as distributed computing with a focus on
% highly dynamic environments~\cite{CS_Foster01a}: any application running in grids
% must be aware of the volatile and dynamic nature of the environment.  Our
% implementation hides that behavior from the application layer as far as
% possible, and provides dynamically loadable extensions to cope with the changing
% middleware representation. That plug in mechanism is present on different layers
% of the implementations architecture:  new API packages can be added at compile
% time, new middleware bindings can be added at runtime, and the implementation
% can switch these middleware bindings between any two calls, for flexibility and
% fail safety.
% 
% %Modularity: provides a framework easily extendable 
% \subsubsection{Modularity makes the implementation Extensible}
% These plug in mechanisms have well defined interfaces, which allow middleware
% vendors and research projects to easily extend the reference implementation in 
% a \emph{modular} way, with both domain specific API packages, and with additional 
% middleware bindings.
% 
% \subsubsection{Portability and Scalability}
% The overall implementation is highly \emph{portable and platform independent}.  All
% operations are inherently asynchronous, even if this is not supported by the
% underlying middleware.  Bulk operations are automatically optimized, but need
% middleware support. The overall overhead introduced by the library is
% neglect able to any overhead introduced by the inherent latencies of grid
% environments.
% 
% In order to keep the implementation as portable and platform independent as possible, 
% we mainly rely on the strict usage of Standard C++ language features only, the usage 
% of Boost libraries wherever possible as well as the C++ Standard Library.
% 
% 
% 
