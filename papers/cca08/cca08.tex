\documentclass{article}

\usepackage{graphicx}
\usepackage{epsfig}
\usepackage{subfigure}
\usepackage[hypertex]{hyperref}
\usepackage{subfigure}  
\usepackage{color}
\usepackage{srcltx}

\usepackage[small,it]{caption}

\usepackage{ifpdf}

\newcommand{\I}[1]{\textit{#1}}
\newcommand{\B}[1]{\textbf{#1}}
\newcommand{\BI}[1]{\textbf{\textit{#1}}}
\newcommand{\T}[1]{\texttt{#1}}
\newcommand{\NL}{\newline}

\newif\ifdraft
%\drafttrue
\ifdraft
 \newcommand{\jhanote}[1]{  {\textcolor{red}    { ***Shantenu: #1 }}}
 \newcommand{\katznote}[1]{ {\textcolor{cyan}   { ***Dan:      #1 }}}
 \newcommand{\amnote}[1]{   {\textcolor{magenta}{ ***Andre:    #1 }}}
 \newcommand{\hknote}[1]{   {\textcolor{blue}   { ***Hartmut:  #1 }}}
\else
 \newcommand{\jhanote}[1]{}
 \newcommand{\katznote}[1]{}
 \newcommand{\amnote}[1]{}
 \newcommand{\hknote}[1]{}
\fi


\ifpdf
 \DeclareGraphicsExtensions{.pdf, .jpg}
\else
 \DeclareGraphicsExtensions{.eps, .ps}
\fi

% CFP:
%
%
% October 22 and 23, 2008
% 
% Dramatic growth in data and equally rapid decline in the cost of
% highly integrated clusters has spurred the emergence of the data
% center as the platform of choice for a growing class of
% data-intensive applications. To encourage conversations between
% those developing applications, algorithms, software, and hardware
% for such "cloud" platforms, we are convening the first workshop on
% Cloud Computing and its Applications (CCA'08).
% 
% This workshop will include a mixture of invited and contributed
% talks on cloud computing, data intensive scalable computing, and
% related topics.
% 
% Topics of interest include:
% 
%   - compute and storage cloud architectures and implementations
% * - map-reduce and its generalizations
% * - programming models and tools
%   - novel data-intensive computing applications
%   - data intensive scalable computing
%   - distributed data intensive computing
%   - content distribution systems for large data
%   - data management within and across data centers
% 
% If you would like to give a contributed talk, please submit a five
% page extended abstract by August 15, 2008. These submissions will
% also be used to select papers for a poster session. Extended
% abstracts should be submitted to:
% https://cmt.research.microsoft.com/CCA2008.


\begin{document}

\title{\large Programming Paradigms in Cloud Computing}

\author{Shantenu Jha$^{12}$,
        Andre Merzky$^1$\\[1em]
        %
        $^1$\small
          Center for Computation and Technology, 
          Louisiana State University\\
        $^2$ \small
          Department of Computer Science, 
          Louisiana State University
       }

\maketitle

\begin{abstract}

  True, Clouds seem like 'Grids Done Right', including scalability,
  transparency, and ease of management.  And the dominant application
  environment for compute clouds (which are virtual machines) seems to
  make a discussion about application programming void.  However, the
  most successful cloud applications show clearly that explicitely
  distributed programming paradigms, in the like of MapReduce,
  AllPairs, and BigTable, are required by a large set of applications,
  to make Cloud infrastructures a viable compute environment for a
  large class or problems.  The existence of multiple implementations
  of these programming paradigms also make clear, that application
  portability is, even in clouds, and emerging problem which needs
  addressing beyond the level of system virtualization.

  This paper discusses these and other challenges a cloud application
  programmer has to face, and demonstrate potential solutions by
  several example applications.  We show that efficient and scalable
  implementations of Cloud typical application frameworks, such as
  MapReduce and Hadoop, are possible on a system independent level.
  We further discuss that lessons learned from the programming of Grid
  applications apply, to some extent, also in Cloud environments.

\end{abstract}



\newpage

\begin{verbatim}

-- intro

  - dominant compute intensive cloud applications are, at the moment,
    pleasingly distributed or very loosely coupled.  
  - dominant data intensive cloud applications use novel programming
    paradigms, e.g. MapReduce, BigTable, etc.
  - what about other application classes?

-- clouds

  - Clouds seem to be designed to support the first two categories, by
    their affinity
  - explain affinity etc etc etc
  - discuss how affinity allows high level programming abstractions,
    such as map reduce (would be tough to implement on a not-data
    affine grid, as performant/redundant GFS would be missing)

-- programming models and affinity

  - discuss what programming models are required for other apps
    classes in table
  - pick one or two, and details, and derive required affinities
    - loosely coupled heterogeneous:
      - long living applications
      - fault tolerant
      - compute intensive, CPU colocation within components
    - tightly coupled, homogeneous
      - if small tasks:
        - need to avoid queueing delay (slide in, pools, ...)
      - if large tasks
        - need fault tolerance
      - all need co-location, fast channels, reservation?

-- conclusions 

  - clouds are young, and one should anticipate clouds with other
    affinities, and target application classes
  - for academia, the reverse procedure may prove useful: define the
    application classes, derive programming models, derive affinities,
    design clouds supporting those.  implement on Grids for a good
    measure ;)

\end{verbatim}


\section{Introduction}

\begin{table}[h]
 \begin{center}
  \footnotesize
  \begin{tabular}{|p{.25\textwidth}|p{.33\textwidth}|p{.33\textwidth}|}
    \hline

    \B{Application Class}                              &
    \B{Data    Driven}                                 &
    \B{Compute Driven}                                 \\\hline

    \B{Pleasingly\NL Distributed}                      &
       SETI$@$home                                     &
       Monte Carlo Simulations of\newline
       Viral Propagation                               \\\hline

    \B{Loosely Coupled,\NL Homogenous}                 &
       Image Analysis                                  &
       Replica Exchange Molecular\newline
       Dynamics of Proteins                            \\\hline

    \B{Tightly Coupled,\NL Homogenous}                 &
       Semantic Video Analysis                         &
       Heme Lattice-Boltzmann\NL Fluid dynamics        \\\hline

    \B{Loosely Coupled,\NL Heterogeneous}              &
       Multi-Domain\NL Climate Predictions             &
       Kalman-Filter Fluid Dynamics                    \\\hline

    \B{Dynamic Event\NL Driven}                        &
       Desaster support                                &
       Visualization                                   \\\hline

    \B{First Principle,\NL Distributed}                &
       GridSAT                                         &
       MapReduce-Based Motif\NL Distributed search     \\\hline

  \end{tabular}
  \caption{\small Classes and specific examples, 
           covering most primary categories of
           Distributed Applications~\cite{dpa-paper}.
          }
  \label{fig:classes}
 \end{center}
\end{table}


\section{Conclusions}

\section{Acknowledgements}

 \bibliographystyle{plain}
 \bibliography{cca08}

\end{document}

