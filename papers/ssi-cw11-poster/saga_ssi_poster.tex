% We switch to portrait mode. This works as advertised.
\documentclass[a0b,portrait]{a0poster}
% You might find the 'draft' option to a0 poster useful if you have
% lots of graphics, because they can take some time to process and
% display. (\documentclass[a0,draft]{a0poster})
\usepackage{graphicx}
% Switch off page numbers on a poster, obviously, and section numbers too.
\pagestyle{empty}
\setcounter{secnumdepth}{0}

% The textpos package is necessary to position textblocks at arbitary 
% places on the page.
\usepackage[absolute]{textpos}

% Graphics to include graphics. Times is nice on posters, but you
% might want to switch it off and go for CMR fonts.
\usepackage{graphicx,wrapfig,times}
\graphicspath{{images/}{../images}}

% These colours are tried and tested for titles and headers. Don't
% over use color!
\usepackage{xspace}
\usepackage{color}
\definecolor{DarkBlue}{rgb}{0.1,0.1,0.5}
\definecolor{Black}{rgb}{0.0,0.0,0.0}
\definecolor{Red}{rgb}{0.6,0.2,0} %{0.694,0.305,0.09}
\definecolor{LightGray}{rgb}{0.3,0.3,0.3}

\definecolor{ItalColor}{rgb}{0,0,0} %{0.2,0.6,0.2}
\definecolor{BoldColor}{rgb}{0,0,0} %{0.8,0.3,0.1}
\definecolor{TextColor}{rgb}{0,0,0} %{0.1,0.1,0.5}
\definecolor{CodeColor}{rgb}{0,0,0} %{0.1,0.1,0.1}
\definecolor{HeadColor}{rgb}{0.1,0.1,0.5}

% see documentation for a0poster class for the size options here
\let\Textsize\normalsize
\def\Head#1{\noindent\hbox to \hsize{\hfil{\huge\color{HeadColor} #1}}\bigskip}
\def\AHead#1{\bigskip\bigskip\noindent{\Large\color{HeadColor} #1}\smallskip}
\def\LHead#1{\bigskip\bigskip\noindent{\huge\color{HeadColor} #1}\smallskip}
\def\Subhead#1{\noindent{\huge\color{HeadColor} #1}}
\def\Title#1{\noindent{\VeryHuge\color{DarkBlue} #1}}

\TPGrid[40mm,40mm]{15}{25}  % 3 - 1 - 7 - 1 - 3 Columns

% Mess with these as you like
\parindent=0pt
%\parindent=1cm
\parskip=0.5\baselineskip

% abbreviations
\newcommand{\ddd}{\,\mathrm{d}}
\newcommand{\up}{\vspace*{-1em}}
\newcommand{\down}{\vspace*{1em}}
\newcommand{\I}[1]{{\color{ItalColor}\textit{#1}}\color{TextColor}}
\newcommand{\B}[1]{{\color{BoldColor}\textbf{#1}}\color{TextColor}}
\newcommand{\T}{\texttt}

\newcommand{\F}[1]{\B{FIXME: #1}}

\usepackage{srcltx}
\usepackage{fancyvrb}
\usepackage{ifpdf}

\DefineVerbatimEnvironment{mycode}{Verbatim}
{
  label=Code Example,
  fontsize=\normalsize,
  frame=single,
  framerule=1pt,
  framesep=1em,
% numbers=left,
  gobble=2
}

\begin{document}

\begin{textblock}{11.2}(0,0)
\baselineskip=3\baselineskip 
\Title{SAGA: The Simple API for Grid Applications\\
\Huge~Simple, Sustainable and Scalable Application Development}
\end{textblock}

\begin{textblock}{11.2}(0,1.35)
\AHead{Ole Weidner, Shantenu Jha\\[0.2em]
\large~Center for Computation \& Technology, Louisiana State University, Baton Rouge, U.S.A.
}
\end{textblock}

\begin{textblock}{3.8}(11.9,-0.2)
  \includegraphics[width=150mm]{cctfullcolortower}
\end{textblock}

\begin{textblock}{15}(0,2.6)
\color{LightGray}\rule{\linewidth}{2pt}
\end{textblock}

\begin{textblock}{15}(0,3.2)
  \includegraphics[width=788mm]{bigpicture}
\end{textblock}

\begin{textblock}{7}(0,8.5)

%%%%%%%%%%%%%%%%%%%%%%%%%%%%%%%%%%%%%%%%%%%%%%%%%%%%%%%%%%%%%%%%%%%%%%%%%%%%%%%%
\LHead{What is SAGA ?}
\large

\textbf{\color{DarkBlue} OGF Standard } The Simple API for Grid Applications
(SAGA) is an official OGF standard (GFD-R-P.90) that describes a high level,
object-oriented API that can be used to develop distributed applications for
\textit{clouds}, \textit{grids} and other distributed infrastructures.  SAGA
provides abstractions for the three pilars of large-scale distributed
computing: job handling, data handling and communication. It is the only
community standard that tries to provide a comprehensive set of interfaces for
all three pillars with the same \textit{look and feel} (common syntactic and
semantic conventions), which tremendously simplifies application development.

To cope with the constantly changing landscape in distributed computing, SAGA
also describes a well-defined API extension process. This allows for the
addition of new interface groups (\textit{packages}) without compromising the
integrity of the API. The following API packages (and extensions) are
currently defined in SAGA:

\begin{itemize} 
\item{Job definition, submission and management }
\item{File transfer, (remote) read/write } 
\item{Replica catalog }
\item{Advert catalog } 
\item{Streaming client and server } 
\item{Remote procedure calls } 
\end{itemize}

\textbf{\color{DarkBlue} SAGA Implementations } There are currently several
independent SAGA implementations under development. The three most prominent
ones are CC-IN2P3's jSAGA (Java), Vrije University's JavaSAGA (Java, Python)
and Louisiana State University's SAGA implementation in C++, Python. All three
SAGA implementations consists of three components: (1) the API layer, (2) a
thin runtime layer and (3) \textit{adaptors} that translate the API calls to a
specific distributed middleware service (see figure). This plug-in
architecture allows vertical extensibility, i.e. support for several
concurrent middleware services at the same time. Together with the horizontal
extensibility of the API itself, this approach allows SAGA to grow and scale
in two dimensions. Any distributed application, framework or tool that uses
SAGA can directly benefit from this feature. The SAGA implementations 
currently support different subsets of the following middleware services:

\begin{itemize} 
\item{Job API: Condor, GRAM, gLite, Amazon EC2, Eucalyptus, PBS, LSF, Torque
OGSA-BES (Unicore, Arc), SSH, Fork}
\item{File API: Curl, GridFTP, HDFS, Local FS, SSH} 
\item{Replica API: Globus RLS, SQLite3, PostgreSQL}
\item{Advert API: SQLite3, PostgreSQL} 
\item{Streaming API: TCP Sockets} 
\item{RPC API: XMLRPC} 
\end{itemize}

\textbf{\color{DarkBlue} SAGA Deployment } Over the past two years, SAGA is
becoming slowly but steadily part of many software stacks across large-scale
distributed computing infrastructure around the world. Finished and ongoing
deployment efforts include: LONI, TeraGrid, FutureGrid. Planned deployments
and deployments under negotiation include TeraGrid-XD (US) and EGI (EU) - two
of the largest upcoming distributed infrastructure deployments in the world.



\end{textblock}

\begin{textblock}{7}(8,8.5)

%%%%%%%%%%%%%%%%%%%%%%%%%%%%%%%%%%%%%%%%%%%%%%%%%%%%%%%%%%%%%%%%%%%%%%%%%%%%%%%%
\LHead{Why Use SAGA ?}
\large 

\textbf{\color{DarkBlue} Shorter Development Time:} SAGA provides a very
simple, POSIX-inspired programming model with a consistent look and feel
across different API packages. It is very easy to use and
it saves the developer a lot of time by superseding the need to learn and
understand different system-specific syntactic and semantic conventions.

\textbf{\color{DarkBlue} Portability:} Distributed applications that are
written against the SAGA interfaces are portable across different grid and
cloud infrastructure because the do not have to interface with platform- or
service-specific interfaces anymore. This means that a distributed
application, tool or framework can be taken from a local computer to a
cluster, to grid or cloud infrastructure without having to re-write its
distributed capabilities from scratch. 

\textbf{\color{DarkBlue} Scalability:}

\large


%%%%%%%%%%%%%%%%%%%%%%%%%%%%%%%%%%%%%%%%%%%%%%%%%%%%%%%%%%%%%%%%%%%%%%%%%%%%%%%%
\LHead{How SAGA is Used}
\large

There are many scientific computing projects that are successfully using SAGA
or SAGA-based software on national and international (or both!) grid and cloud
infrastructure to cary out production-level science and simulations. Examples 
include:

\textbf{\color{DarkBlue} High-Throughput Bio...?}

\textbf{\color{DarkBlue} RENAISSANCE} is a science gateway project for
facilitating community-wide ncRNA research and genome-wide analysis of
high-throughput sequencing data. It uses SAGA to bridge the gap between
browser frontend, scientific workflow generation and high-performance
computing backends. SAGA allows the project to explore advanced pipelining and
workflow strategies for ncRNA research on grids and clouds without being
locked-in to a specific infrastructure provider or technology. 

\textbf{\color{DarkBlue} Cactus??}

\textbf{\color{DarkBlue} Ganga / Diane Framework } This project is a good
example of how SAGA can be used to extend existing frameworks and tools: Ganga
and Diane are tools frequently used by the HEP community at CERN to run
master-worker style applications on  gLite-based European grid infrastructure.
Adding SAGA-based job submission and file transfer capabilities to these tools
enabled researchers to run their applications on non gLite-based (e.g.
Globus-based) grids without having to change any of their existing
applications.

\LHead{Road Ahead, \& Opportunities} 
\large



\end{textblock}

\begin{textblock}{15}(0,24.5)
\color{LightGray}
\rule{\linewidth}{2pt}
\end{textblock}

\end{document}

