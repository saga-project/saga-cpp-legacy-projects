%\documentclass[acmjacm,acmnow]{acmtrans2m}
\documentclass[10pt,letterpaper]{article}

\usepackage{graphicx}
\usepackage{url}
\usepackage{color}

\newcommand{\projectnamefull}{\textit{Programming Abstractions for
    Large-Scale Distributed Applications} }

\newcommand{\upup}{\vspace*{-0.5em}}
\newcommand{\up}{\vspace*{-0.25em}}

\newcommand{\I}[1]{\textit{#1}}
\newcommand{\B}[1]{\textbf{#1}}
\newcommand{\T}[1]{\texttt{#1}}

\newcommand{\BI}[1]{\B{\I{#1}}}

% \ifpdf
%   \DeclareGraphicsExtensions{.pdf, .png, .jpg}
% \else
%   \DeclareGraphicsExtensions{.ps, .eps}
% \fi

\long\def\comment#1{{\bf \textcolor{magenta}{\bf #1}}}
\long\def\ccomment#1{{\bf \textcolor{blue}{\bf #1}}}
\newcommand{\C}{\comment}
\newcommand{\CC}{\ccomment}

\newcommand{\yes}{$\bullet$}

\newif\ifdraft

\ifdraft
 \newcommand{\amnote}[1]{   {\textcolor{magenta} { ***Andre:    #1 }}}
 \newcommand{\hknote}[1]{ {\textcolor{blue}    { ***Hartmut:   #1 }}}
 \newcommand{\jhanote}[1]{  {\textcolor{red}     { ***Shantenu: #1 }}}
\else
 \newcommand{\amnote}[1]{}
 \newcommand{\jhanote}[1]{}
 \newcommand{\hknote}[1]{}
\fi

% \markboth{Shantenu Jha et al}{Abstractions for Large-Scale Distributed
%   Applications and Systems}

\begin{document}

\title{SAGA..} 
% \author{Shantenu Jha$^{1,2,3}$, Murray Cole$^{4}$, Daniel
%   S. Katz$^{1,5}$, \\ Manish Parashar$^{6,7}$, Omer Rana$^{8}$,
%   Jon Weissman$^{9}$ \\[1em]
%   \small $^1$Center for Computation \& Technology,
%   Louisiana State University\\[-0.3em]
%   \small $^2$Department of Computer Science,
%   Louisiana State University\\[-0.3em]
%   \small $^3$e-Science Institute,   University of Edinburgh\\[-0.3em]
%   \small $^4$School of Informatics, University of Edinburgh\\[-0.3em]
%   \small $^5$Department of Electrical and Computer Engineering,
%   Louisiana State University\\[-0.3em]
%   \small $^6$NSF Center for Autonomic Computing, Rutgers University\\[-0.3em]
%   \small $^7$Department of  Electrical and Computer Engineering, Rutgers University\\[-0.3em]
%   \small $^8$Department of Computer Science,
%   Cardiff University\\[-0.3em]
%   \small $^9$Department of Computer Science,
%   University of Minnesota\\[-0.3em]
% }

% \category{J.2}{Computer Applications}{Physical Sciences and Engineering}
% \terms{A.1 INTRODUCTORY AND SURVEY, C.2.4 DISTRIBUTED SYSTEMS, D.1.3 Concurrent Programming}
% \keywords{Distributed Applications, Distributed Systems,
%   CyberInfrastructure, Abstractions, Scientific Applications}
%\tableofcontents

\begin{abstract}
  And this is how we changed the world..

\end{abstract}

\maketitle

\begin{verbatim}

Scope of the CPC Paper

1.  Background: changing landscape of computational/digital
  infrastructure:
  - Traditional  Grids	+ Distributed Systems
  - Emerging Distributing Systems (Clouds, distributed data-store..)
  - High-end machines are becoming heterogenous, but at a different
        level of granularity
    Implication of the heterogenity of monolithic systems implies

  Develop scientific applications for all of the above -- in a
  "Standard"  uniform way. This is SAGA i.e, let us not just
  stay confined to traditional Grids. But we need to be very 
  clear that SAGA is applicable only for certain classes of 
  applications; SAGA != MPI; SAGA != a programming model per se
  and not an execution model (like MPI). SAGA supports
  different PM and ExecutionModels.

  Motivation: 
  Traditional Grid Systems:
    - heterogeneity (of what?)
    - .. 
  
   Emerging Distributed Systems:
    - heterogeneity (system interface, but also semantic difference)

   High-End Motivation:
     Why SAGA on these machines?

   SAGA provides the fundamental abstractions --- functionality,... ,
   that are required to develop applications in a system independent
   fashion
 
  Portability: syntactic, semantic and platform independence.

  SAGA Programming System:  (Engine + Packages) + Adaptors + Development tools for SAGA

  The API is exposed via the Engine (functional:context, session,
  non-functional:others) and Packages (functional and non-functional)


2. Some Distributed SCIENTIFIC Applications .. 
   possible use cases..

3. How these features/requirements are supported?
   Could have overlap with "requirements" 
   Develop "requirements"
   ....

4. All the Gory Details that have been developed to support the points
   in section 3.

   This is where OOPSLA paper comes in handy. Lots of the meat 
   is the same, but presented in a different perspective and style.

5. Applications that utilize these... and how.. for what.. (Broad spectrum)
   .. talk about N applications that we've developed
   .. show the specific features from Section 3 and how they are
   addressed/utilised/or contribute to the application usage
  
What is the contribution of this Paper: OOPSLA Paper is Implementation
paper + MCS Paper is Interface Paper, this is a mege of the two using
a consistent vocabulary along with applications.

\end{verbatim}

\section{Introduction}

The process of developing and deploying large-scale distributed
applications presents a critical and challenging agenda for
researchers and developers working at the intersection of computer
science, computational science and a diverse range of application
areas. 


\section{Applications}\label{application}

Introduce applications -- motivate SAGA

\section{An Overview of SAGA}

or {\it SAGA in a Nutshell}

\section{Design of the API}

\section{SAGA Programming System}
\subsection{SAGA Engine}
\subsection{SAGA Packages}
\subsection{Adaptors}

\section{Applications Redux}

\subsection{Closing the Loop}

Show how SAGA helps the problems outlined in Section\ref{application}

\subsection{Discuss Applications Further}

The Kind of Applications that can be developed..

\section*{Acknowledgements}

%\bibliographystyle{unsrt}
\bibliographystyle{phcpc}
\bibliography{saga_data_intensive}
\end{document}

