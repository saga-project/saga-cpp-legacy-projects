% $Header: /projects/VU-SAGA/Papers/saga_engine_2006/requirements.tex,v 1.9 2006/08/13 18:01:00 hkaiser Exp $

 As already indicated in the introduction, the SAGA C++ reference
 implementation has to cope with a number of very dynamic boundary
 conditions.  Additionally, it has to provide the simple and
 easy-to-use API the SAGA specification is intended to specify.  This
 section describes the resulting requirements in some detail, and
 motivates the SAGA implementation design which is described in the
 next section.

 \subsection{Dynamic Specification Landscape}
 \label{ssec:ogf}

   The Open Grid Forum (OGF)~\cite{ogf_website} is a international standardization
   body whose primary objective is to define a set of standards in the
   emerging field of grid computing.  OGF specifications are supposed
   to cover grid architectures, protocols, interfaces, and APIs.
   However, the whole field is young, and in its complexity not yet
   completely understood, neither in terms of academic research, nor
   in terms of industrial and commercial applicability and impact.
   That and the complexity of the problem itself causes the
   grid specification landscape to evolve rather slowly: it has
   several significant gaps, and it is widely expected that existing
   specifications will change.  The time required for grid standards
   to stabilize is expected to be in the order of 5 to 10 
   years~\cite{ogsa_roadmap}.

   Owed to the hyping of grid computing, and to the frustration of end
   users with distributed environments in general (scalability and
   interoperability is still, after many years, a very difficult
   problem on many layers), the expectations to grids to solve real
   world problems still are very high.  These observations imply the
   necessity of an interface abstraction for early adopters which
   shields the implementers of grid applications from the evolving
   grid standardization landscape, and allows for a migration path to
   later grid systems with assessable effort.  
   
   \IB{A SAGA implementation should therefore be able to cope with evolving
   grid standards, and changing grid environments.}


 \subsection{Evolving SAGA Specification}

   The SAGA specification itself is currently limited, and
   intended to expand, in scope over time.  In particular in
   respect to new emerging grid service standards it is expected that
   new SAGA extension will be required to provide the respective 
   programming paradigms to the application developers.  The general
   look and feel of the SAGA specification is, however, thought to be
   somewhat more stable, and there is hope that extensions are merely
   semantically (new objects, new method calls), but with limited or
   no syntactical additions (no new object model, or task model etc.).
   
   \IB{A SAGA implementation must be able to cope with future SAGA
   extensions easily, without breaking support and backward
   compatibility for early SAGA adop\-ters and applications.}


 \subsection{Evolving Grid Middleware}

   The evolution of grid standards as described in~\ref{ssec:ogf}
   implies that implementations of these standards are
   evolving as well, and very much so.  In fact, the major Grid
   middleware system used over the last 8 years or so,
   Globus~\cite{globus}, went from version 1.0 to 4.0, thereby
   undergoing significantly more interoperability breaking updates
   than the major version numbers suggest.  Evolutions of other grid
   middlewares does not differ in that respect significantly, unless it
   was developed for very specific environments and purposes.
   
   These systems are, on the other hand, large projects and well
   funded, and invest significant effort in training and support.
   Smaller systems, research developments, and standard reference
   implementations have, in general, the same problem, but much less
   resources to limit the impact of that development for the end user.
   Industrial/commercial implementations with the usually accompanying
   professional support and well defined migration paths are, in
   reality, to be counted on the fingers of one hand.

   \IB{Any high level grid API implementation, such as a SAGA
   implementation, must be able to shield the application programmer
   from the evolving middleware implementations, and in particular
   should allow various incarnations of grid middleware to co-exist.}


 \subsection{Dynamic Grid Environment}

   As grid middleware evolves, deployed grid environments face constant
   changes of middleware deployments (new versions and new services
   get enrolled frequently, often with unclear migration paths).
   Also, grid environments are dynamic by design, in respect to the
   availability of services and other resources.  Any application
   designed to run on grids must ideally be aware of that property of
   grids, and should implement fail safety mechanisms, and should not
   rely on the static availability or resources.  Very much of that
   flexibility however can (and should, in our opinion) be hidden from
   the application programmer.  For example, an upgrade in a services
   protocol version should be handled in the client libraries talking
   to the service, if possible, and not on application level.
   Resource discovery, fail safety on service failures and simple fall
   backs as the utilization of redundant service deployments are other
   examples of mechanisms which are vital for grid applications, but
   do not need any explicit reflection in the application code. 
   
   \IB{A SAGA implementation should therefore allow for and, where
   possible, actively support fail safety mechanisms, and should hide
   the dynamic nature of grid resource availability from the
   application.}


 \subsection{Heterogeneous Grid Environment}

   The dynamicity of grid environments is also reflected in their (at
   least potential) heterogeneous nature: although most deployed grids
   focus on Linux based clusters, grids are designed to cope with any
   OS (real or virtual), on any resource.  The predominance of Linux is
   rather a indication of the prematurity of grid middleware
   developments than an intentional design artifact. 

   \IB{A SAGA implementation must be portable and, both
   syntactically and semantically, platform independent.}


 \subsection{Distributed Grid Applications}

   With the use of distributed computing, and hence the
   use of remote communication within the application domain, the
   impact of the communications latency playes a major role in the
   design and applicability of distributed concepts.  Grid
   environments do not pose any exception.  
   
   A number of application domains have emerged though, which can, by
   loosely coupling distributed components, or by utilizing
   various latency hiding techniques, cope very well with latencies of
   distributed environments.  Latency hiding techniques (such as
   caches, bulk operations, interleave of computation and
   communication, and asynchronous communication) do often require
   application level information to be effective (e.g. concurrency
   information of operations).  
   
   \IB{A library designed for distributed
   applications must allow these and other latency hiding techniques
   to be implemented} -- otherwise its applicability to real world
   problems will be severely limited.


 \subsection{End User Requirements}

   The SAGA API is, by definition, designed to meet end user
   requirements.  In fact the current SAGA specification was developed
   based on the responses to a call for use cases to the grid 
   community~\cite{saga_req, saga_uc}.
   
   \IB{An API implementation must, however, meet other end user
   requirements which are outside the scope of the actual API
   specification, such as ease of deployment, ease of configuration,
   documentation, and support of multiple language bindings.}
   
   If any of these properties is missing in an implementation, its
   acceptance in the targeted user community will be severely limited.


% As the scope of the SAGA specification is to extent over time, 
% we use API packages allowing to extend the API scope horizontally, 
% at compile time.  Package creation is greatly simplified by 
% generic call routing, and by using Macros, which resemble the 
% IDL specification.
% \F{We need to define that term later, or use a different
% descrition -- AM}
%
% The API implementation is also extensible vertically, at
% runtime, as it can dynamically load adaptors binding teh SAGA API
% to the respective Grid middleware environment. 
% 
% The API implementation has a third orthogonal dimension of 
% extensibility: it provides a synchronous, an asynchronous and a task
% oriented version of every operation, it allows to transparently bundle
% several tasks into bulks, and in the future it will provide means
% of building graphs of tasks to build workflows.
%
% The dynamically loaded adaptors are selected at run time, on
% a call-by-call basis (late binding) -- the implementation of an 
% API object can hence be provided by multiple, partially complete 
% adaptors.
%
% The late binding requires to maintain object state on client
% side, and independent from the adaptors.  The implementation
% includes a generic class state repository ensuring thread safety 
% of the stored data items while maintaining easy data access.
%
% The core of the SAGA C++ implementation, the SAGA engine, provides
% common functionality for all API packages such as generic call routing,
% appropriate adaptor selection, common error recovery mechanisms etc.
% Additionally it implements common Grid oriented functionality as 
% security contexts, debug helper functions, generic monitoring etc.
% 
% As for any other distributed environment latencies are a
% major performance concern for Grid applications, too.  Our
% implementation allows any operation to be performed
% asynchroneously, for both - synchronous and asynchronous
% adaptors. Missing functionality is provided by the SAGA engine.
% Latency hiding schemes (such as combining operations into bulks) 
% can be applied generically and transparently with regard to the API.
% 
% Grid environments are, by definition, heterogeneous.
% Portability and platform independence, and multiple language
% bindings are thus crucial for user acceptance.  Extensive 
% utilization of Boost~\cite{boost_website}, strict adherence to standard C++ 
% features, and thin wrappers for other programming languages 
% ensure these properties for our implementation.  




