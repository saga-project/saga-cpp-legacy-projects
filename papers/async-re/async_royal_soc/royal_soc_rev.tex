\documentclass{rspublic}   

%------------------------------------------------------------------------- 
% take the % away on next line to produce the final camera-ready version 
%\pagestyle{empty}

%\usepackage[utf8]{inputenc}
\usepackage{graphicx}
\usepackage{url}
\usepackage{float}
\usepackage{times}    
\usepackage{multirow}    
\usepackage{listings}   
\usepackage{times}     
\usepackage{paralist}    
\usepackage{wrapfig}    
\usepackage[small,it]{caption}
\usepackage{multirow}
\usepackage{ifpdf}   
\usepackage{subfigure} 

                    
%Bibliography                     
\usepackage{natbib}   

\usepackage{listings}
\usepackage{keyval}  
\usepackage{color}
\definecolor{listinggray}{gray}{0.95}
\definecolor{darkgray}{gray}{0.7}
\definecolor{commentgreen}{rgb}{0, 0.4, 0}
\definecolor{darkblue}{rgb}{0, 0, 0.4}
\definecolor{middleblue}{rgb}{0, 0, 0.7}
\definecolor{darkred}{rgb}{0.4, 0, 0}
\definecolor{brown}{rgb}{0.5, 0.5, 0}

\title[Efficient Large-Scale Replica-Exchange Simulations on
Production Infrastructure]{Efficient Large-Scale Replica-Exchange
  Simulations on Production Infrastructure}

\author[Thota, Luckow, Jha]{
  Abhinav Thota$^{1,2}$, Andr\'e Luckow$^{1}$ and Shantenu Jha$^{1,2,3}$\\
  \small{\emph{$^{1}$Center for Computation \& Technology, Louisiana State University, Baton Rouge, LA 70803, USA}}\\
  \small{\emph{$^{2}$Department of Computer Science, Louisiana State
      University, Baton Rouge, LA 70803, USA}}\\
  \small{\emph{$^{3}$e-Science Institute, Edinburgh EH8 9AA, UK}}\\
}

%\date{}

\def\acknowledgementname{Acknowledgements}
\newenvironment{acknowledgement}%
{\section*{\acknowledgementname}%
\parindent=0pt%
}

\newif\ifdraft
\drafttrue
\ifdraft
\newcommand{\jhanote}[1]{ {\textcolor{red} { ***shantenu: #1 }}}
\newcommand{\alnote}[1]{ {\textcolor{blue} { ***andre: #1 }}}
\newcommand{\athotanote}[1]{ {\textcolor{green} { ***athota: #1 }}}
\else
\newcommand{\alnote}[1]{}
\newcommand{\athotanote}[1]{}
\newcommand{\jhanote}[1]{}
\fi

\newcommand{\I}[1]{\textit{#1}}
\newcommand{\B}[1]{\textbf{#1}}
\newcommand{\T}[1]{\texttt{#1}}

\newcommand{\glidein}[1]{Glide-In }  
\newcommand{\ReplicaAgent}[1]{Replica-Agent }         
\newcommand{\replicaagent}[1]{replica-agent }         
\newcommand{\remanager}[1]{RE-Manager }

\begin{document} 


\maketitle    

\begin{abstract}{Replica-Exchange, SAGA, Large-Scale, Production}  
%   The design and development of an application is often influenced and
%   constrained by the programming systems and the infrastructure it is
%   developed against.  It is important to break the coupling between
%   the development and the underlying infrastructure, in order to
%   enable applications to be flexible (across infrastructure),
%   extensible (to new methods of communication and coordination) and
%   scalable.
  % Developing applications that are able to orchestrate
%   heterogeneous resources across distributed resources is a complex
%   task yet is an important design objective of both logically and
%   physically distributed applications. 
\alnote{reads a little bumpy} 
  Replica-Exchange (RE) methods, which represent a class of algorithms
  that involve a large number of loosely-coupled ensembles, are
  used to understand physical phenomena -- ranging from protein
  folding dynamics to binding affinity calculations.  We develop a
  framework for RE that supports different replica pairing and
  coordination mechanisms, that can use a range of production
  cyberinfrastructure concurrently.  We use this framework to
  implement three different formulations of the RE algorithm.  We
  characterise the performance of the different RE algorithms at
  unprecedented scales on production distributed infrastructure.


%  flexible and robust implementation enables the efficient use of a
%   broad range of infrastructure.
%   (publish-subscribe centralised notification),

% Developing applications that are able to orchestrate heterogeneous
% resources across distributed resources is a complex task. Inevitably,
% the design and development of an application is influenced and
% constrained by the programming systems and the infrastructure it is
% developed against. Breaking this coupling between the development and
% the underlying infrastructure, to enable applications to be flexible
% (across infrastructure), extensible (to new methods of communication
% and coordination) and scalable is an important design objective of
% distributed applications both logically distributed and physically
% distributed.  In this work, we focus on the Replica-Exchange (RE)
% Methods which represent a class of algorithms that involve a large
% number of loosely-coupled ensembles. RE simulations are used to
% understand physical phenomena Ð ranging from protein folding dynamics
% to binding affinity calculations.  In this work, we develop a
% flexible, extensible and scalable implementation of RE that can
% utilise a range of infrastructure concurrently (and
% autonomically/adaptively), that supports different coordination
% mechanisms (publish/subscribe, centralized notification), different
% replica pairing mechanisms (synchronous versus asynchronous) and
% thereby different variants of the RE algorithm. We implement and
% demonstrate how a flexible and robust implementation enables the
% efficient use of a broad range of infrastructure.

\end{abstract}
% \alnote{We are on about 13.5 pages (without comments). 
% Do we need to shorten it to 12? space-saving? Or is it sufficient
% to do it with the camera-ready version?}


%\athotanote{addressed: corrected graph: from no. replicas to no. machines. 8. It seems that even in the decentralized case, the advert service is still centralized? Are there issues of latency when accessing the advert service or keeping it consistent? 9. When multiple replicas are concurrently searching for a partner to exchange with it is not clear how this is done safely. Even with reverification non-deterministic behavior could still arise if care is not taken. 7. In section 3b(ii) -- what is "MD"? - we explain this earlier as molecular dynamics in parenthesis.. should we do that again? - fixed 6. The advert service (in section 3(a)) needs further discussion. What is an advert in this context. Also how is a key/value pair generated. What does it mean in this context. Is this similar to a DHT in a peer-2-peer system? In particular, the authors should explain how a job is described -- and subsequently matched. 4. The time calculation in equation 2.2 needs further elaboration. Parameters such as T-ex, T-mgmt are not clear. For instance, the mention of an "advert-server" is surely an implementation detail? (corrected) In particular, the authors should comment on whether such a simple linear combination is an effective representation of reality -- given that many of these operations are being performed over a distributed environment. - addressed in section 2(a) 3. It would be useful for the authors to compare the replica exchange application, as disussed here, with the general class of "parameter sweep" application -- where some synchronisation may also be required. - addressed in section 2(b) 2. There needs to be an explanation of why the temperature needs to be exchanged between replicas and what the role of the Metropolis algorithm is. 5. What is the "Metropolis scheme" -- give reference or explain.- we reference metropolis in section 2.  1. The paper would be improved if the work were better motivated by an example application near the start of the paper (for example, the MD simulation that is used later in the paper) - in introduction. 1) Don't use symbols to refer to "Section" - write it out as text. 2) Try to avoid using the word "utilize" and derivatives thereof. The word "use" can almost always be used instead. 3) Introduction, line 1, change to "Replica-Exchange (RE)......methods represent a class of algorithms..." 4) Introduction, first para: Change "developed against" to "developed within"? 5) Introduction, near end of second para: remove "different coordination mechanisms". 6) Section 2(a), line 3: change to "total time-to-completion of an experiment". 7) Don't use ampersand - use "and" instead. 8) 3 lines after eqn. 2.2: this is the first mention of the advert server, so you either need to explain what it is, or make a forward reference, or drop the text in parentheses. Also "book-keeping" should be "bookkeeping". 9) Last line of page 2: change to "in the asynchronous-centralised case". 10) Section 2(b), second para, line 1: change to "For the synchronous RE formulation". 11) Section 2(b), third para: omit comma after "consequence". 12) Section 2(b), fourth para, line 1: omit "there are". 13) Section 3(a), first para, last line: change "the conduction of" to "conducting". 14) Section 3(a), second para, line 2: change to "enables the dynamical use of a range of" 15) Section 3(a), third para, change "store which is used for" to "store used for". 16) Section 3(a), third para, line 6: change to "multiple big-jobs". 17) Section 3(b), second para, line 5: change to "has a centralised". 18) Section 3(b)(ii), second para, line 6/7: change to "still in the done state". 19) Section 4, first para, line 3: change to "as well as a basic" 20) Section 4(b), first para, line 2: change to "is in the synchronisation" 21) Section 4(b), third para, line 9: change to "suggests there are between 2 and 4" 22) Section 4(b), third para, line 11: change to "increasing numbers of replicas". 23) Section 4(c), first para, last line: change to "in these cases" 24) Section 5(a), first para, last line: "the" is repeated" 25) Section 5(a), second para, line 2: change to "exchanges to replicas" 26) Section 5(a), last para, last line: "the" is repeated" 27) Section 5(b), first para, line 1: change to "of the asynchronous" 28) Section 6, first para, line 1: change to "Following theoretical underpinnings (Li and Parashar 2007, Gallicchio et al. 2008), in this paper..." 29) Section 6, second para, line 2: change "enable" to "enables" 30) Section 6, second para, line 5: change "out weight" to "outweigh".}


\section{Introduction}

Replica-Exchange (RE)~\citep{hansmann,Sugita:1999rm} methods
represent a class of algorithms that involve a large number of
loosely-coupled ensembles.  RE simulations are used to understand
physical phenomena -- ranging from protein folding dynamics to binding
affinity calculations.  The design and development of most RE
implementations~\citep{Woods:2005nx} is influenced and constrained by
the programming systems and the infrastructure it is developed
within.  Breaking this coupling between the development and the
underlying infrastructure, to enable applications to be flexible
(across infrastructure), extensible (to new methods of communication
and coordination) and scalable is an important design objective of
distributed applications -- both logically distributed and physically
distributed.

Application algorithms that are scalable while being flexible and
extensible are better suited to using the diverse range of traditional
and hybrid infrastructure (e.g., grid-cloud and heterogeneous
resources). Along with application formulations that facilitate the
flexible utilisation of a range of infrastructure, it is imperative to
have the correct runtime abstractions that support flexible
deployment of these applications.  In support of flexible and scalable
formulations of the RE class of algorithms, we develop a RE Framework
that supports multiple formulations, is extensible to a broad range of
infrastructure and as we shall show scales-up and scales-out.  Our RE
framework uses a flexible pilot-job implementation (SAGA BigJob)
to support the execution of the ensembles.  It supports scalable
implementation of RE that can use a range of infrastructure
concurrently that supports different
exchange coordination mechanisms (synchronous versus asynchronous),
and thereby different variants of the RE algorithm.  

\athotanote{we already mention the next sentence in section 4.}The physical
system that we use as benchmark is the Hepatitis-C Virus that was
examined in Ref.~\cite{Luckow:2008fp}. The problem and our motivation is similar as before. \athotanote{check if this ok}
\alnote{we could remove it from sec 4. The paragraph can in my opinion 
be merged with the one above}
% , if using multiple
% distributed resources, they require prior co-scheduling (Manos et
% al. 2008)

% \cite{Luckow:2008fp} take it to the next level and is an
% example of adaptive RE simulations on production-level grid resources,
% while \cite{parashar_arepex} is an example of \emph{asynchronous} RE
% simulations, which is based on
% CometG~\citep{Li:2005:CSC:1090948.1091381}, a decentralised
% computational infrastructure for Desktop Grid environments.
% and for a specific implementation of the asynchronous scaled-out to
% 4 machines.  \jhanote{Last sentence is unclear} We present results
% using which the reader can understand which model of RE is most
% suited for a particular set of resources (distributed, local etc.,)

The paper is organised as follows. Section~\ref{sec:repex-approach}
sketches out the three different RE algorithms that are investigated;
we also present an approximate mathematical model for the different
algorithms.  Section~\ref{repexfw} outlines the architecture of the RE
Framework -- the SAGA BigJob (how it supports the dynamic execution of
multiple replicas) and other important elements that make the
framework flexible and extensible.  In section~\ref{sec:re_impl}, we
present our implementation of the RE algorithms and understand the
primary determinants of performance and relate it to the mathematical
model of section~\ref{sec:repex-approach}.  In section~\ref{sec:performance}, we
describe the experiments performed to assess and understand
performance when scaling-up (on a single machine) as the number of
replicas increases.  We compare and analyze the performance of the
different RE formulations (synchronous and asynchronous) when
scaled-up to 256 replicas as well as when scaled-out to use more than
one machine. Section~\ref{sec:conclusion} concludes the paper and
discusses future work.

% We also investigate the scaling-out
% characteristics, namely performance as the number of replicas are
% increased while the (distributed) resources employed
% increases. % while keeping the number of replicas on local and
% distributed resources.  \jhanote{Currently we have only Section 6
%   and no 7}.
% We present the results and analysis, in Section VI and 


% \alnote{Should we add a section with some scientific background: HIV,
%   Hepatitis...?}  \jhanote{given the tightness of space, I think we
%   should try to avoid it. OK?}

\section{Replica-Exchange Algorithms}\label{sec:repex-approach}

The RE class of algorithms involves the concurrent execution of
\emph{replicas} - which are defined as instances of essentially
similar simulations but with minor differences, such as the defining
temperature of the replica. These replicas are loosely-coupled, in
that there is an infrequent exchange between pairs of 
% \jhanote{do we want to say paired?} \athotanote{fixed}
replicas; in addition to the frequency of communication between the
replicas being low (relative to within a single replica), the amount
of information/data exchanged between replicas is small (relative to
the operating data-set size). The size of the data exchanged between stages is in bytes, where as the operating data-set size is approximately 20 megabytes. \athotanote{should i add anything else?}
\alnote{I would merge the last two sentences}

% \alnote{We should give a number for $\eta$ for sync
%   RE}\athotanote{wouldn't that be implementation dependent?}
% \alnote{What is the value of $\eta$ for async RE?} \athotanote{as
%   mentioned earlier, i the value depends on the implementation;
%   centralised=1, decentralised= $N_R\over2$} \athotanote{do we need
%   the equation here again?}  \athotanote{I think
%   $\eta_{sync,async-cent}$ should be 1 as only one exchange is done at
%   a time(single master process) and $\eta_{async-decent}$ should be
%   $N_R \over 2$, as each pair is involved in negotiating an
%   exchange. since this depends on implementation, i am not including
%   the $\eta$ values here.} \jhanote{Andre: If the points in the above
%   exchange have been addressed, could you please comment out? Thanks.}

\subsection{Mathematical Model}
\label{sec:math-model}
In this subsection we develop a mathematical model that captures
the primary components that make up the total runtime of a RE
experiment. In an ideal scenario, the total time-to-completion of an
experiment would be equal to the concurrent runtime of the ensemble of
replicas and there would be no overhead associated with the
coordination of the replicas.  If the ensemble contains $N_R$
replicas, the total number of pairwise exchanges is defined to be $N_X$ and the
runtime of a replica to complete a defined number of time steps is defined to be
$T_{MD}$, the total time-to-completion of an experiment $T$ would
be:
\begin{eqnarray}
T = {1\over p} \times (T_{MD} \times  {N_X \over {N_R \over 2}}) 
\label{eq:totaltime}
\end{eqnarray}
where $p$ is defined as the probability of a successful exchange (the
probability of a successful exchange is not 1, the decision to accept
an exchange or not is made using the Metropolis
scheme~\citep{metropolis:1087}. After the exchange, the replicas are restarted with the new temperatures. \athotanote{reviewers asked to explain metropolis, its role, why temperature needs to be exchanged and/or provide reference. is the explanation we give here sufficient?}  However, any RE experiment will
entail an overhead of coordination, job-submission and termination
etc. Thus, the time $T$ to complete the RE experiment where N$_X$ is
the total number of exchanges is:
%\alnote{I would propose to write $\frac{N_X}{\eta}$ to have both terms consistent}
\begin{eqnarray}
  T = {1\over p} \times [(T_{MD} \times  {N_X \over {\eta}}) +
  {(T_{EX} + T_{W})} \times {N_X \over \eta}]
\label{eq:totaltime}
\end{eqnarray}
where, $T_{EX}$ is the time to perform a {\it pairwise} exchange and
includes the following components, (i) time to find a partner ($T_{find}$),
(ii) time to exchange/write/transfer files ($T_{file}$) as well as (iii)
managing state (e.g., in a central database) and bookkeeping
associated with replica pairing/exchanging ($T_{state}$) (thus $T_{EX} =
T_{find} + T_{file}+T_{state}$); $T_W$ is comprised of the synchronisation
time spent by a replica waiting for other replicas to complete running.
%($T_s$), the time spent waiting to be restarted after each exchange
%and other associated costs ($T_r$). For example, in the
%asynchronous-centralised case, $T_r$ arises due to serialisation at
%the exchanging agent.  
\alnote{Shouldn't we explain $T_{state}$ before $T_{W}$?}
$T_{state}$ arises due do different
reasons -- which may be related to
implementation % Not suprisingly, the
% management costs also differ for
of the different RE algorithms. Finally, $\eta$ is the number of
independent exchange events occurring concurrently; for $N_R$
replicas, this is typically $\frac{N_R}{2}$. %$N_R \over 2$.
In an distributed scenario, $T_{MD}$ might be different on different machines and therefore, $T_W$ could be effected. Also, depending on the physical location and other network related aspects, $T_{EX}$ could also be effected. \athotanote{is this ok or should i add anything else?} 
\alnote{I would discuss the distributed scenario in the experiments section with the resp. scenario. Also, the analysis must be improved! We defined $T_W$ as 0 for the async case e.g.}

% It is worth
% noting that depending upon the case in consideration, there might be
% different causes leading to a synchronisation barrier, e.g., for
% synchronous RE the synchronisation barrier is between replicas waiting
% to complete; 

\subsection{Synchronous Replica-Exchange}

Traditionally, RE algorithms have been implemented such that the
exchanges have been synchronous.  If the number of replicas is
${N_R}$, this leads to a {\it fixed} number of ${N_R \over 2}$ pairs
of replicas are created.  When \emph{all} the replicas in the ensemble
reach a pre-determined state (e.g. the Molecular Dynamics (MD)
simulation completes a pre-determined number of steps), an exchange of
temperatures between the paired replicas is attempted using the
Metropolis scheme.  If the exchange attempt is successful, parameters
such as the temperature are swapped.

% In Equation~\ref{eq:totaltime}, we introduced the various components
% that make up the total time to complete an RE experiment. Only 
% 1 para limitation on traditional replica exchange
%running concurrently

For the synchronous RE formulation, all replicas must reach a
pre-determined state (\texttt{done}), before exchanges are performed.
$T_W$ includes the time waiting for all the replicas in the ensemble
to reach this state, the time spent waiting before the replicas are
restarted after each exchange and any other miscellaneous costs. In contrast to parameter-sweep applications, RE is not scatter-gather in nature. There is synchronization involved in RE, but not in scatter-gather applications. \athotanote{please verify preceding statement} \alnote{Gathering involves synchronisation.}

A major limitation of this model is that replicas are paired in fixed
groups and thus exchanges take place between pre-determined pairs of
replicas.  As a consequence of pairs being determined before an
exchange, although $T_{find}$ is $0$, this limits the number of possible
exchange partners that are available for a given replica; this
inhibits exchanges between replicas with non-nearest temperatures, and
ultimately negates the possibility of crosswalks -- where a crosswalk
is said to occur when a replica originally with a low temperature
reaches the upper temperature range and then returns to the lower
temperature range.

In addition to limitations in modeling the physics, rigid
replica-pairing is efficient only in homogeneous environments; for
heterogeneous environments and systems, where resource availability and
performance fluctuates, the need for synchronisation leads to
slow-down and inefficiencies. We show how these limitations are
overcome in the asynchronous (exchange) formulations of RE.

\subsection{Asynchronous Replica-Exchange}

% - Introduce asynchronous Replica Exchange -- 1 para on case II and
% case III (algorithmically) To overcome these limitations and
% implement çit on distributed grid resources.

In asynchronous formulations of the RE
algorithms~\citep{parashar_arepex,DBLP:journals/jcc/GallicchioLP08}, a
replica does not have to wait for {\it all} other replicas to reach a
pre-determined state. An exchange occurs whenever a replica reaches a
pre-determined state. It then performs an exchange with another suitable
replica in the ensemble.  The reduction in {\it synchronisation}
(wait) times comes at the cost of increased {\it coordination} costs.
The specific values of the terms $T_{EX}$ and $T_W$, in
Equation~\ref{eq:totaltime} differ from the synchronous formulations.
Since each replica on completing a run has to find a {\it new}
partner, $T_{find} \neq 0$.  Additionally, $T_W$ is 0 because there is no synchronisation involved.% only includes the time
%spent waiting for the next replica to become available, any time spent
%waiting before the replicas are restarted after an exchange and any
%other miscellaneous costs.

% a pair of replicas are available, and where \jhanote{Is the
%   algorithm similar or identical? If it is not identical, how is it
%   different, or is it just the case that the implementation is
%   different?} \athotanote{is it ok now? i think we are taking the
%   async algorithm as is and implementing it on production
%   infrastructure. ?}
% Let us look at how the ŧō and waiting costs introduced in
% equation \ref{eq:totaltime2} behave. 

% \alnote{This is too general.  We should drop the redundancy of
%   explaining every component and rather analyze the differences.}
% \athotanote{you are right; i am adding a couple of lines here. does
%   it help?} \jhanote{As currently written, $T_W$ has not been
%   defined or ``different components'' discussed, i.e., this
%   paragraph should come after the modelling section}

% But also, an asynchronous RE algorithm has the potential to perform
% better than the traditional RE: (i) when we scale-up the number of
% replicas and (ii) when we scale-out across many machines.
%\jhanote{There is no basis for this sentence at the moment. REMOVE}

\section{Replica-Exchange Framework}\label{repexfw}

An important motivation for this work is to design and implement a
framework that provides the capability to implement and compare the
performance of the different RE algorithms formulations at
large-scales.  In addition, it is important that the framework be
independent of the underlying infrastructure.  It is useful to
highlight that we differ from other RE implementations (e.g.
\cite{parashar_arepex}) in that we use {\it production grade} national
and regional cyberinfrastructure, such as the US TeraGrid and
LONI~\citep{LONI_web}, using general purpose tooling and standard
capabilities available on these production infrastructure.
Additionally, our framework {\it natively} supports individual
replicas that are MPI jobs. In this section we outline the
architecture, implementation and the basic performance of the RE
framework when used to implement the different formulations.
The source-code is available here ~\cite{}. \athotanote{need to create citation or give url here?}

% In following subsections we
% introduce the different components in the RE Framework.

% is to present an infrastructure independent solution that makes it
% possible to implement a variety of RE algorithms, such as synchronous
% and asynchronous, that can
% asynchronous RE model we developed runs on production level grids such
% as the Teragrid, unlike specialized
% infrastructures.

\subsection{SAGA BigJob - A Pilot-job Framework}
\label{sec:BigJob}

% The Simple API for Grid Applications (SAGA)(~\citep{saga_gfd90}) is an
% API that provides the basic functionality required to build
% distributed applications, tools and frameworks so as to be independent
% of the details of the underlying infrastructure. SAGA is an API
% standardization effort within the Open Grid Forum
% (OGF)~\citep{ogf_web}, an international standards development body
% concerned primarily with standards for distributed computing. 

%%%%% FIGURE %%%%%
\begin{figure}[t]
      \centering
          \includegraphics[scale=0.45]{../figures/Bigjob_arch.pdf}
          \caption{\footnotesize RE-Manager and SAGA BigJob Manager framework: The BigJob-Agent is used
as place-holder job for all replica sub-jobs under a big-job. The RE-Manager controls
both the BigJob-Agent and the replica jobs using the SAGA BigJob Manager.
              }
      \label{fig:bigjob}
\end{figure}

We have demonstrated the usage of the SAGA-based Pilot-Job
framework~\citep{saga_bigjob_condor_cloud} -- called the BigJob, to
run RE simulations across multiple, heterogeneous, distributed grid
and cloud infrastructure~\citep{Luckow:2008fp}.
SAGA~\citep{saga-url} %\athotanote{2011?}
is an API to the basic functionality required to build distributed
applications, tools and frameworks so as to be independent of the
details of the underlying infrastructure.  The various tasks that are
carried out using the SAGA APIs include file staging, job spawning and
conducting the exchange attempts.

Here we use the BigJob framework to efficiently request and manage
computational resources for multiple replicas.  Specifically, it
enables the dynamical use of a range of infrastructures, i.e., it
supports a scheme that does not depend on a static set of resources
that are pre-defined at the time of workload submission.

Figure ~\ref{fig:bigjob} shows the architecture of SAGA BigJob.  It
consists of three components: (i) the BigJob-Manager, (ii) the
BigJob-Agent and (iii) the advert service which is a central key/value
store used for communication between the BigJob-Manager
and the BigJob-Agent. 
% Here each application can store the endpoint contact URL as an advert, and another
% application which wants to connect to the first one may obtain this information
% from the advert service. 
\athotanote{check sentence} \alnote{refined. please refer to the advert service as 
advert service not server (see also figure). Server refers in my opinion to a piece of hardware.}
For each new sub-job, an advert entry storing the description of the sub-job
is created by the BigJob-Manager. The BigJob-Agent periodically 
polls for new jobs. If a new job is found and resources are available,
the job is dispatched, otherwise it is queued. 
% The key-value pairs are initially generated by the BigJob-Manager and once the big-job becomes active, the BigJob-Agent continuously updates the values. The advert service is different from a distributed-hash table in that it is a centralised system. 
% The BigJob-Manager submits the big-jobs to the resource manager. 
% Sub-job descriptions are store in the advert service. Once a big-job 
% becomes active, the BigJob-Agent retrieves the job descriptions from the advert
% server, allocates the required number of nodes and launches the
% sub-jobs on each resource. 
Further, the agent continuously monitors the
running sub-jobs and updates the sub-job states at the advert
service. Once a sub-job finishes running, the nodes are freed and
marked as available. 
% The BigJob-Agent periodically polls the advert
% server for new jobs.

\subsection{Replica-Exchange Manager}\label{repexmanager} 

% \jhanote{Abhinav: This subsection should help the reader understand
%   the basic and common elements of the RE Framework -- job submission,
%   role of bigjob-agent, replica-agent, replica-exchange manager etc
%   etc} \athotanote{job submission, bigjob agent are explained in 3a. RE-Manager is explained below. replica-agent is touched upon but a more detailed explanation in section 3b(ii)}
  
%   \jhanote{ABHINAV: DEFINE THE ROLE OF RE-MANAGER HERE.  Possibly BIGJOB
%   AGENT here if not in previous subsection.}  \athotanote{please check if it's ok now}
  
The RE-Manager is the master process controlling the different
components of the framework and aspects of the exchanges. The 
RE-Manager uses the SAGA BigJob-Manager; the actual tasks that the RE-Manager
performs depend on which RE algorithm is being investigated.  A
replica-agent is a wrapper script that manages an individual replica
-- starts, facilitates the exchange of that replica, and restarts it.

% And at the least, it is the SAGA BigJob Manager.  At the most,
% launched in place of the replica. The replica-agent implementation
% where the RE-Manager does a lot of work (a centralized
% implementation). But in an implementation where there is a
% \emph{replica-agent} for every replica, the It should be noted that
% synchronous and asynchronous RE are different algorithms and the
% implementation of same algorithm could be centralized and
% decentralized.

The asynchronous RE algorithm can be implemented using a centralised
or decentralised coordination scheme.  In the former, the RE-Manager
manages all the replicas and performs the exchanges; in the
decentralised coordination implementation, multiple replica-agents
take on that responsibility.  It is worth noting that the synchronous
RE algorithm has a centralised coordination scheme.  Interestingly, for
decentralised coordination implementation, the RE-Manager is
functionally similar to the SAGA BigJob-Manager.

% Synchronous RE is implemented in a centralised fashion. Centralized
% implementation suits synchronous RE because there is a
% synchronisation step at the beginning of each exchange step. But we
% implement asynchronous RE in both centralised and decentralised
% fashions.

% The RE-Manager also creates directories, configuration files and
% stages them to the respective directories.  \alnote{We should focus
%   on the RE part here. The BigJob stuff move to subsec a). This
%   should also remove the redundancy with a)}

\begin{figure}%
\centering
\subfigure[Centralised]{\includegraphics[width=0.35\textwidth]{../figures/central_AL.pdf}}\qquad
\subfigure[Decentralised]{\includegraphics[width=0.35\textwidth]{../figures/decentral_AL.pdf}}\\
\caption{\textbf{Centralised vs. Decentralised Coordination:} Both
  central coordination style is used both by the synchronous RE  and asynchronous (centralised) RE.  In the asynchronous (decentralised) RE --
  the master is only required for initially setting up all
  replicas. The later coordination is done peer-to-peer via the Advert
  Service.}
\label{fig:coordination}
\end{figure}

% \jhanote{this is an implementation detail. Abhinav: IIUC we restart in
%   both cases -- successful and unsuccessful exchanges. Non? ALso don't
%   toggle between ``replicas'' and ``jobs''!}  

We first explain how the synchronous and asynchronous (centralised) RE
Manager works in the following section followed by an explanation of
how the asynchronous (decentralised) RE-Manager/replica-agent
combination works.

\subsubsection{Synchronous and Asynchronous (Centralised) RE-Manager}

% We first explain how the synchronous RE-Manager works and then
% explain asynchronous (centralized) RE-Manager. 
% centralized implementation, Figure~\ref{fig:coordination}a) can be
% used to describe both.

% The architecture of an asynchronous (centralized) RE-Manager is
% similar to that of synchronous RE-Manager.
% Figure~\ref{fig:coordination}a) shows the control flow.

The control flow of a centralised RE scheme is shown in
Figure~\ref{fig:coordination}(a), which can be used to understand both
the asynchronous (centralised) RE and synchronous RE implementations.
A replica can be in one of these three states: (i) \texttt{new} (submitted but not started), (ii) \texttt{running} and (iii) \texttt{done}.
Once the BigJob(s) is active and replicas are \texttt{running}, the RE-Manager constantly
queries the SAGA BigJob-Manager for the latest replica states.  When
the RE-Manager finds a replica that has finished running, it collects
the energy and temperature of that replica by reading the output
file. Once \emph{all} the replicas have finished running, the
RE-Manager performs the exchanges by swapping temperatures and writing
new configuration files. The new configuration files are staged to the
appropriate location. The RE-Manager then submits the replicas for
restarting, and the SAGA BigJob-Manager restarts them. The RE-Manager
keeps count of the successful exchanges, until the required number of
exchanges are done.

The implementation of the asynchronous (centralised) RE-Manager, is
different from the synchronous RE-Manager, in that, instead of waiting
for \emph{all} replicas to finish running before performing \emph{all}
exchanges, whenever the asynchronous (centralised) RE-Manager finds a
replica that has finished running, it tries to find a partner to make
an exchange. In order to find a partner, the RE-Manager goes over the
list of all the replicas in the ensemble. If it finds a replica
available it attempts the exchange. If a replica is not found
available, the RE-Manager queries the SAGA BigJob-Manager for the
latest replica states and updates its local list. It then loops over
the list to find a replica that has finished running and a partner to
exchange with that replica.  If successful, the replicas are submitted
to be restarted. % \alnote{see comment about BJ
%   impl. detail above}.
% The RE-Manager counts the successful
% exchanges. This process is repeated till all the exchanges are made.


% \alnote{Not sure what we should do with the following paragraph}
% \jhanote{Does not belong here} It should be noted that where as in the
% synchronous RE, the exchanges are conducted when all replicas finish
% running. And the replicas are restarted only after all the exchanges
% are completed. But in an asynchronous (centralized) RE, the exchanges
% are conducted whenever possible and the replicas are restarted.

\subsubsection{Asynchronous (Decentralised) RE-Manager and the Replica-Agent}

% Thus, after the replica-agents are launched, the RE-Manager and the
% SAGA BigJob Manager don't have any responsibilities.
 

Figure~\ref{fig:coordination}(b) shows the asynchronous
(decentralised) RE control-flow.  In the asyn\-chron\-ous (decentralised)
implementation, in order to conduct the exchanges the RE-Manager
launches multiple replica-agents (in lieu of replicas directly).
Replica-agents then take control of replica start/re-start and
exchange attempts.  The replica-agents upon launch, run the replicas;
%\jhanote{what is nodefile? I know what it is but use the
%  general description of what is happening here} 
% \alnote{refined} \jhanote{Thanks Andre. Should have been clear, it was
%   for Abhinav to understand. Sorry you had to do it}
a list of nodes that is used to carry out the MD (molecular dynamics) run is passed to the
replica-agent as an argument at startup.  The replica-agent constantly
monitors the replica, and when the replica finishes, it updates the
advert server with the current state of that replica.  It also reads
the temperature and energy from the output files, and posts the values
to the advert server.  The RE-Manager is primarily responsible for
keeping track and count of the number of exchanges performed; when the
desired number of exchanges are done, the RE-Manager ends the
experiment.

% \jhanote{ISNT THE NEXT PARAGRAPH ESSENTIALLY THE SAME AS THE PREVIOUS
%   PARAGRAHGP? WHY DONT YOU REFER TO THE FIGURE AND THE BASIC StePS
%   OUTLINED IN THE FIGURE?}
  
% \begin{figure}[t] \centering
%           \includegraphics[width=0.82\textwidth]{../figures/decent_state.pdf}
%           \caption{\footnotesize Asynchronous (decentralized) RE
% replica state diagram } \alnote{This chart is not correct yet. There
% can't be state changes on conditional fields} \jhanote{is this a four
% state model or a 3 state model?}
%       \label{fig:state}
% \end{figure}

% As shown in state diagram  (Figure \ref{fig:state}) 
% The replica-agent also goes through the list of replicas randomly to
% try to find an exchange partner.  

% \alnote{In the original state chart we had a ``free'' state. I think
%   this state is needed to ensure the atomicity property of the
%   exchange protocol (2pc style).}
% \jhanote{Would the \texttt{free}
%   state the same as the \texttt{complete} state?}
% \athotanote{i think a replica in \texttt{free} state implies that it
%   is available to make exchanges (meanwhile it also searches or a
%   partner). in the present explanation, that would be \texttt{done}
%   state} 


% \jhanote{Are the four states/description common to all RE cases? If
%   yes, why is it in this paragraph and not in a subsection that is in
%   scope for all three cases? If not, what are the states for the other
%   cases?} \athotanote{not common. added a sentence in the previous section mentioning the states.}

% \athotanote{after an
%   exchange is complete, the state has to be set as complete by the
%   exchange-initiator, so that the other replica-agent can restart its
%   replica} 

A replica can be in one of the these four states: (i)
\texttt{running}, (ii) \texttt{done}, (iii) \texttt{pending} (for an
exchange) and (iv) \texttt{complete} (exchange has been completed).
When a replica reaches a pre-determined state it (the replica-agent
acts as the proxy for the replica) transitions from \texttt{running}
to \texttt{done}. It initiates the search for a partner, and scans the
list of replicas randomly, so as to avoid contention if multiple
replicas have initiated a search for a partner.  If a replica finds
another replica available, it reverifies that their states as still in
\texttt{done} state; if both are still in \texttt{done} state, the exchange
can proceed and their states are set as \texttt{pending}.  When {\it
  reverifying} the state of the replicas, if the state of the
initiating replica has already been changed (to \texttt{pending}), the
current exchange attempt is aborted; if not, the exchange attempt
proceeds.

% , and waits for its state to
% be set to \texttt{complete} by the other replica (the initiating replica now).
% \jhanote{Is this ``it'' the initiating replica or the ``other
%   replica''?}  \jhanote{Please read the
%   previous sentence carefully: ``... it can restart its replica..''
%   something is wrong!} \athotanote{modified. any better?}

After the exchange is performed (the temperature of both replicas in
the advert server has been changed), their states are set as
\texttt{complete}. Here the state \texttt{complete} is a marker
which tells the replica-agent that the exchange has been made, the
configuration files are in place and to restart the replica.  
The associated replica-agents write new
configuration files with updated temperatures, and restart their
replicas. The replica-agent that initiated the exchange increments the
exchange count. The RE-Manager constantly queries the advert server
for the latest exchange count and when all exchanges have been made,
it stops the experiment.

\alnote{paragraph needs attention}
It should be noted that even in the decentralised implementation, the advert 
service is still centralised. Only the decision-making and replica-management 
is decentralised. We did not observe any increase in latency when accessing 
the advert service with respect to increasing number of connections. 
\alnote{we did observe a severe slowdown!!} But the 
communication times do vary depending on the physical distribution. Also, 
maintaining data consistency is not an issue because it is a centralised system. 
Two write actions cannot be carried out simultaneously on a single value. 
It should also be noted that the exchanges are non-deterministic in nature 
because the replicas available for exchange are found in a random order 
and then the Metropolis scheme~\citep{metropolis:1087} is used to match the replicas. 



% The search for an exchange partner is conducted in a random
% manner when the number of replicas in the ensemble is large, so as to
% avoid having all the replicas starting at the first replica, which
% causes contention and failed exchanges. But for smaller number of
% replicas, a linear search over all the replicas also works. 

% To ensure atomicity of the state transitions, after the initiating
% replica finds an exchange partner, it {\it reverifies} its state,
% i.e., it has not been chosen by another partner in the duration that
% it was finding a partner.  


% \athotanote{i was trying to say the following: replca 1 finds replica
%   2 in done state, but before it locks replica 2, it reverifies both
%   states. (replica 1, while searching could have been set to pending
%   by replica x) If states are still done, the replicas are locked and
%   exchange is made. I made some changes in the explanation. please see
%   if its easier to understand now.}

% \jhanote{I don't understand how its state is
%   set to complete and then to restart the replica} \athotanote{is it
%   better now?}. 
% \jhanote{Do you mean
%   replica-agent?} \athotanote{yes, fixed.}

% At the point where the first replica reverifies the states, if one of
% the replicas' states have changed: (i) if it is the second replica,
% the replica-agent tries to find a new replica; (ii) if it is the first
% replica itself, that would mean another replica has set the replica as
% \texttt{pending}. 

% \subsubsection*{This paragraph to be removed on discussion}

% , it re-verifies the
% states of both itself and the other replica and if both are still
% available, 
% exchanges their temperature values on the advert server.  

% \jhanote{I though the RE-Manager did the following} \athotanote{the
%   replica-agent increments the count, the RE-Manager gets the count
%   from the advert server. Please look at the above para and tell me it
%   makes sense}\jhanote{OK. thanks}

% \jhanote{I don't understand any of the stuff below. State changes have
%   to be atomic!?: If the state of any of the replicas negotiating the
%   exchange changes at the time of re-verifying the replica pair's
%   states, that would mean that one or both of the replicas' is now in
%   the process of making an exchange with a different replica. In that
%   case one or both of the replica-agents will wait till their state is
%   set as "Ready" and start their next run. If one of the replicas is
%   still available, it tries to find a new partner for exchange.}
% \jhanote{Lets talk about this. I
%   don't think another replica should be allowed to change the state
%   of the ``initiating'' replica whilst the ``initiating'' replica
%   is out looking for a partner}

% The search for an exchange partner is conducted in a random manner so
% as to avoid having all the replicas starting at the first replica,
% which causes contention and failed exchanges. If it finds another
% replica available, it re-verifies the states of both itself and the
% other replica and if it finds both still available, marks their states
% as "Pending". Let the replica initiating the exchange be $R_a$ and the
% other replica be $R_b$.  The replica $R_a$ which initiates the
% exchange is the replica in-charge of the exchange. $R_a$ then
% exchanges their temperature values on the advert server.  \alnote{We
%   should detail the exchange protocol. How do we ensure that the
%   exchange is atomic? figure?}  It then modifies its configuration
% files locally, sets its state as running and runs the replica. On the
% other hand, the $R_b$'s state is set as "Ready" by $R_a$ and that
% makes the $R_b$'s replica-agent to retrieve its own temperature
% (modified by the $R_a$) from the advert server, modify its
% configuration file locally and run the replica. \alnote{Why does it
%   need to modify its own config if it uses its own temperature?}
% \athotanote{fixed} This removes the need to stage the configuration
% files to different machines. This is possible because there is a
% replica-agent for each replica locally. Where as earlier, the
% RE-Manager and the replica might be located on different machines.  If
% the state of any of the replicas negotiating the exchange changes at
% the time of re-verifying the replica pair's states, that would mean
% that one or both of the replicas' is now in the process of making an
% exchange with a different replica. In that case one or both of the
% replica-agents will wait till their state is set as "Ready" and start
% their next run. If one of the replicas is still available, it tries to
% find a new partner for exchange.  \alnote{How often is the process
%   repeated until the replica is started with the old temperature?}
% \athotanote{fixed}The replica-agent that initiated the exchange
% increments the exchange count by one.  If an exchange fails due to
% contention, the replica-agent tries to find a new partner. If an
% exchange is deemed unsuccessful after comparison of energies or
% temperatures, the replicas are restarted with their old
% configurations.

% \jhanote{Do we have a pending state? Accoridng to Abhinav there is
%   only a two-state model -- run and done?!} 
% \athotanote{should i keep the state diagram? do you suggest any
%   corrections to the state diagram? i also think that instead of
%   having the current control flows, we should have figures showing how
%   the exchanges are made or instead hae state diagrams. the control
%   flows right now show a lot of stuff already shown in the bigjob
%   architecture figure. }
  

%\section{Implementation of Synchronous and Asynchronous RE}
\section{RE Framework: Implementation and Performance}
\label{sec:re_impl}
% \alnote{Do we want to present impl. details here?}  \athotanote{i too
%   think that we are explaining how we implemented in section 3(RE
%   Framework). Here we are analyzing our implementation?}
% \jhanote{``Basic characterization of the performance of our implementation''}

In section~\ref{repexfw}, we presented the basic components of the RE
framework and discussed the control flow for the three RE algorithm
formulations.  In this section we provide further details of the
working of the RE framework, as well as a basic characterization of the
performance of the RE framework for the three RE algorithm formulations.

% Further, we analyse the primary components
% that impact the performance of a RE simulation.

For the basic characterisation, the following experimental
configuration is used: (i) Infrastructure: Our experiments are
performed on LONI and Teragrid shared resource \emph{QueenBee (QB)}. A
highly scalable, parallel MD code -- NAMD~\citep{Phillips:2005gd}, is
used to perform the MD simulations for each replica (although, it
is important to mention that any other MD or Monte Carlo code could be
used just as simply and effectively with the RE framework).  (ii)
Replica-Exchange Configuration: The total number of replicas ($N_R$)
in the ensemble are 32 and the total number of pairwise exchanges
($N_X$) is 128. As the ensemble of replicas are run concurrently, 16
pairwise exchanges are possible after each concurrent run. Thus, each
replica on average is restarted 7 times.  Each replica is configured
to run 500 time-steps and is allocated 16 processors. One BigJob of
size 512 processors is requested. On average each 500 time-step run
takes $71$ s.  For all implementations, in the event of a
successful exchange, jobs are restarted~\citep{Luckow:2008fp} with new
temperature values.  In the case of an unsuccessful exchange, jobs are
restarted without exchanging the configuration.  (iii) The physical
system that we use as benchmark is the Hepatitis-C Virus that was
examined in Ref.~\cite{Luckow:2008fp}.


% \jhanote{Can the following be eliminated or moved to someplace later} \athotanote{yes, can be deleted}
% It should be noted that the replicas take longer (91 s) to
% complete their 500 time-steps from a fresh start. But when they are
% restarted, they finish the run slightly faster (68 s).  And the
% total time the ensemble of replicas spend running is $71 \times 8 =
% 568$ s.

The different RE algorithms were repeated multiple ($\approx$ 10)
times, during different load factors; therefore specific queue wait
times can be approximated to be similar.  Additionally, as we will
%eventually be primarily 
be interested in understanding the scale-up and scale-out properties
of synchronous and asynchronous RE, we do not consider queueing times.
As explained in section~\ref{sec:repex-approach} (\ref{sec:math-model}),
the relative performance of RE implementations is primarily determined
by the waiting time $T_W$ and the time for conducting the exchange
$T_{EX}$. In the following we analyse the average values for $T_{W}$ and
$T_{EX}$ for each replica pair.  Table~\ref{table:repex_perf} summarizes
the results. We will discuss each case in the following sections.



% Therefore, we start measuring time lapsed only when
% the BigJob agent starts running the replicas/replica-agents because, almost every
% time, the RE-Manager completes submitting them before the
% BigJob becomes active. 
% \jhanote{Is this not just specific to the Synch
%   case? i.e.  for async decentralised case, the replica-agent submits
%   the replicas and not the RE-Manager?!} \athotanote{fixed.}

\begin{table}
    \centering
	\begin{tabular}{|l|r|r|r|}
	\hline
	                        &\textbf{synchronous}  &\textbf{asynchronous (decentralised)} 
	                        &\textbf{asynchronous (centralised)}\\
	% \hline
	%  $T_{MD\_init}$  &23\,sec &23\,sec &23\,sec\\
	%  \hline
	%  $T_{MD\_run}$   &68\,sec &68\,sec &68\,sec\\
	\hline
	\hline
	$T_{MD}$       &71\,s &71\,s &71\,s\\
	\hline
	\hline
	$T_{W}$        &2.8\,s &0\,s &0\,s\\
   % \hline
    %\hspace{2mm}$T_{s}$ &2.8\,s &0\,s &0\,s\\ 
    %\hline
%    \hspace{2mm}$T_{r}$ &0 s&0.7\,s &1.1\,s\\
	\hline\hline
	$T_{EX}$        &0.6\,s &7.9\,s &1.9\,s\\
	\hline
	\hspace{2mm}$T_{find}$        &0\,s   &7.1\,s &1.3\,s\\
	\hline
	\hspace{2mm}$T_{file}$       &0.4\,s &0.6\,s &0.4\,s\\
	\hline
    \hspace{2mm}$T_{state}$    &0.2\,s &0.2\,s  &0.2\,s\\
	\hline
	\hline
	$\mathbf{T}$        &\textbf{1003\,s} &\textbf{631\,s}    &\textbf{811\,s}\\
	\hline
    \end{tabular}
    \caption{Average values of terms in Eqn~\ref{eq:totaltime} for the three RE algorithms. $T_{MD}$=time replica takes to complete 500 timesteps; $T_W$=synchronisation time; $T_{EX}$=time to make a pairwise exchange; $T_{find}$=time to find/lock a partner; $T_{file}$=time to write/transfer files; $T_{state}$=time to update states after exchange; $T$=total time-to-completion.} 
	\label{table:repex_perf}
\end{table}

\subsection{Synchronous RE}
\label{sec:impl_sync_re}
% The replicas are run by the BigJob agent after the BigJob becomes
% active. The average time the Bigjob agent takes to start one replica
% is 0.3 s.  \alnote{Is this the time agent only? Or does it
%   include the submission time on the master?}  \athotanote{fixed in
%   last lines of the previous section. the time the master takes to
%   submit the replicas is being included in $T_EX$. The time spent at
%   BigJob agent is included in $T_W$. }  \alnote{Don't see any
%   reference to $T_{EX}$ in last section. Later you refer to the
%   resubmission time as $T_{mgmt}$. Also, I don't see this component
%   in the $T_W$ formula below. The reader will not know what to do with
%   the 0.6 sec!}  For a pair of replicas, it is 0.6 s.

In the synchronous RE implementation replicas are started
sequentially, i.\,e.\ there is a delay between the startup of the
first and the last replica. Also, the post-processing of each replica
run, i.\,e.\ updating the state, marking nodes free, the stage-out of
the output file, is done sequentially. The longer of the two
determines the overall time spent waiting ($T_W$) for other replicas.
In this case the post-processing time is the larger, and on average,
it takes $1.4\,s$
%for the % BigJob agent
to process a replica that has completed running. Thus, $T_{W}$, which
is defined for a pair of replicas, is $2.8\,s$. For an ensemble of 32
replicas (with 16 pairs) the delay between the first and last replica
transitioning to \texttt{done} state adds up to $44.8\,s$.  
%The waiting time at the BigJob-Agent is subsumed by the synchronisation time, thus the BigJob-Agent is effectively always ready to start replicas, thus, $T_r$ is 0 for synchronous RE.

% \alnote{which is also not really the case since sub-jobs
% are all submitted at once and as we discuss in sec b, the bj-agent 
% processes these sequentially.} \athotanote{this is mentioned above: ``In the synchronous RE implementation replicas are started
% sequentially, i.\,e.\ there is a delay between the startup of the
% first and the last replica. Also, the post-processing of each replica
% run, i.\,e.\ updating the state, marking nodes free, the stage-out of
% the output file, is done sequentially. The longer of the two
% determines the overall time spent waiting ($T_s$) for other replicas.
% In this case the post-processing time is the larger, and on average,
% it takes $1.4\,s$"} 

% As the BigJob agent starts working only after the RE Manager completes
% all exchanges.  



%Thus, $T_W$ is 2.8 s per replica pair.

% That is the time it takes to update the state and
% mark the nodes as free.  Since the whole ensemble of replicas needs to
% complete running, it is sufficient to consider the longer of the time
% to start and the time to mark the end of a replica run. Therefore,
% $T_r$ is included in $T_w$. Since there are 16 pairs in the ensemble,
% for a pair of replicas, $T_w$ is 29.44/16=1.84 s.  

% \alnote{We need to be careful here. If I understand this correctly: stageout of
% file is $T_r$ and stagein $T_{ex}$? Shouldn't we rather consider the stageout
% time above in $T_w$ (since it is part of the delay between the termination of the first
% and last replica)?} \athotanote{i am thinking like this: stageout is done before the exchange and it delays all the replicas becoming available. stagein is done after the exchange. only after the stagein the replicas are submitted to restarting.}

% \alnote{\textbf{included stage-out time into $T_W$ above - just so
%     that you guys can follow my steps (comment and paragraph below can
%     be deleted):} The RE-Manager marks the replicas which finish
%   running as done and retrieves the energies and temperatures by
%   reading the output files. That costs $0.4\,s$ per replica and
%   $0.8\,s$ for a pair. Thus, $T_W = 1.84 + 0.8=2.64$\,s.}


% \alnote{What is the component that is actually spent waiting for the
%   other replicas to complete. The times you mention above are times
%   for actually doing something, but not waiting times!} \athotanote{is
%   it better now?}  
%   \jhanote{in the previous paragraph you write
%   ``seconds''; in the following paragraphs you write {\it sec}. Need
%   consistency!} \athotanote{fixed} \jhanote{Previous paragraph has to be simplified.  Too
%   difficult to understand. REMEMBER: Short simple (ie non compound)
%   sentences}. \athotanote{i think its slightly better now..}

$T_{EX}$ comprises of three sub-components: $T_{find}$, $T_{file}$ and
$T_{state}$. In this scenario $T_{find}=0$\,s due to the fact that there
is fixed pairing. The updating and stage-out of the configuration
files, is observed to be approximately $0.2\,s$ per replica and thus,
$T_{file} = 0.4\,s$ per replica pair (the transfer of the input files is
done sequentially). 
$T_{state}$ is the time required by the RE-Manager to post job
description to the advert server. On average $T_{state}$ amounts to
$0.1$\,s per replica, i.\,e.\ $0.2$\,s per replica pair.
Thus, $T_{EX}$ is: $0+0.4+0.2=0.6$\,s. 

%\alnote{moved this from sec b to here (since it also applies heire)}
Although there are $\frac{N_R}{2}$ concurrent pairs, the exchange 
at the RE-Manager is carried out sequentially; thus the effective 
number of concurrently exchanging pairs is 1. Substituting the above values in
equation~\ref{eq:totaltime}, we get:
% \alnote{TODO: validate numbers with just one decimal position!}
\begin{eqnarray}
  T=  {1 \over p} \times {[ {(71\times {128\over 16}})+ (0.6 + 2.8)\times 128]} = {1 \over p} \times 1003\,s.
  \label{eq:sync}
\end{eqnarray}


% \alnote{Original text: ``$T_{ex}$ includes updating configuration files and transferring them to their
% respective directories; it takes 0.2 s to write and copy a file
% locally.`'' At least in the synchronous version I used there was 
% a remote copy involved. Do we have a number for that?} \athotanote{but this is single machine. local copy right?}
% \jhanote{Abhinav - is the previous sentence correct?
%   ``resubmitting a pair of replicas for restarting'' was too vague!}
% \athotanote{fixed. it may be the case, but we actually explain the
%   actual process after every exchange - that the replicas job
%   descriptions are posted to the advert server, the bigjob agent
%   restarts}
%restart the replicas using the BigJob framework.


% For a pair of replicas involved in the exchange, $T_{ex} =
% 0.2 \times 2=0.4$ s, as file writing and copy is done
% sequentially,
% \jhanote{Should it be $T_{mgmt}$ or $T_{MGMT}$ ? Consistency}
% \athotanote{it should be $T_{mgmt}$, fixed} 
% \alnote{We should be
%   consistent with all subscripts: Why $T_F$, but $T_{ex}$?}
% \athotanote{changed to $T_f$.} \jhanote{Table still needs attention}

% \jhanote{Why is seconds within math mode sometime, and outside math
%   mode at others?}  \athotanote{should we mention somewhere that $p$
%   is one in our experiments and that we dont use metropolis scheme? or
%   can we just leave $p$ as is in the equation?}  \jhanote{For this
%   particular configuration, is this consistent with Data in Section
%   5?}  \athotanote{within error bars, yes.}  \alnote{What about case
%   III: 641+/-2.4 in comparison to 603? But, I guess this doesn't
%   invalidate the model completely}

% \athotanote{yes. i saw this. in the other two cases, substituting
%   $N_X$ in the equation satisfies the all the experiments in section
%   5. but in decentralized, $T_f$ changes with $N_R$. as to why its
%   603 and not closer to 640: i think its something to do the time to
%   find a replica ($T_f$).}

\subsection{Asynchronous (decentralised) RE}

%\athotanote{we can remove the sentence about 0.3/10 s, what do you think?}
%\alnote{yes, please remove. Also, I don't understand how 0.3 s can add up to 10 s}
%When the BigJob becomes active, the BigJob agent starts the replica
%agents and the replica-agents in turn start the replicas. It takes 0.3
%s to start a replica-agent (but this is only a one time event and does not influence the overall time to completion by more than 10 s). 

% \alnote{Is it not clear to me: what is $T_{w}$ and what is $T_{r}$? Shouldn't these times
% be lower than in the async-cent. case since we don't have a central coordinator?}


%As in the asynchronous-centralised case the, asynchronous-decentralised implementation does not involve a replica waiting for other replicas to reach a specific state before initiating an exchange. There are other types of holding that contribute to the overall $T$, e.\,g.\ for managing simulation runs, for updating the advert service, etc. Also, queries and updates to the advert service take longer in the decentralised than in the other both case -- mainly due to the higher number of connections and requests. In total, there are three advert updates required: one for updating the state, two for posting the simulation results. Each update requires about $0.1\,s$, i.\,e.\ in sum about $0.5\,s$. Further, about $0.5\,s$ are required for monitoring, restarting replicas as well as processing of output file. Thus, in total $T_{W}$ has a value of $1\,s$ approximately.

A fundamental difference between synchronous and asynchronous
formulations of RE is in the synchronisation barrier for the replicas
before exchanges, i.e., the former has a barrier, the latter does not.
So although, $T_W = 0$ for asynchronous formulations, this comes at
the cost of a higher $T_{find}$, the lack of a synchronisation barrier
leads to a more involved implementation to dynamically pair replicas.

In the asynchronous (decentralised) RE, the replicas are managed
individually by replica-agents. The replica-agent constantly monitors
its replica. If it finds the replica has completed its run, it updates
its state in the advert server, which takes $0.1\,s$.  The
replica-agent retrieves the energy and temperature from the output
files, each of which takes $0.2\,s$. It then posts both these values
to the advert server.  The waiting times to retrieve these values is
thus $0.1+0.2\times2+0.1\times2=0.7\,s$, so part of $T_{find}$ is $0.7\,s$.  Note that
here $T_{find}$ involves pre-exchange management at the replica-agent and not the 
BigJob-Agent (like the centralised case).  %Therefore the value of $T_W$ $0.7\,s$.

% \alnote{\textbf{Old Paragraph - please check whether I put the important points into
% the paragraph above. Also check my comments in original paragraph.} \athotanote{can be removed}
% \alnote{Initial costs are not really part of our model? Do we 
% need to consider all of them?: 
% The replica-agent initially creates the output and error files, reads the configuration, 
% opens a connection to the advert server, sets the state to running and starts the
% replica. The replica is restarted after every exchange. Therefore, $T_r$, which is 
% recurring is 0.3 seconds.} \alnote{Abhinav: not sure whether I understood everything correct: $T_r$ does not
% include any advert update, or does it?} \alnote{Why does it take longer than in case I/II? 
% \textbf{TODO Abhinav:} separate $T_{mgmt}$ out of $T_r$} \athotanote{done} \alnote{We took
% about 1\,s out of $T_{w/r}$, but only add 0.1\,sec to $T_X$.}
% \athotanote{we are now considering some actions as happening only at
% the start}

% \alnote{How do you get 1.3 sec? The starting of the replica-agent
%   must not be repeated after every exchange, correct?}
% \athotanote{fixed. is it ok now?}

% \alnote{A advert modification took in the previous
% cases 0.01\,sec? What is different here?}.\athotanote{yes, the number of 
%connections to the advert server effects the performance}

% \alnote{Yesterday, it scanned through 15 replicas; in the
%   decent case it scans 16 replica; now it scans 64 replicas. Why is
%   that?} \athotanote{discussed in call.} \alnote{ ok, we still need to
%   explain why we need 4x more attempts than in the async-cent case.}
% \athotanote{done}
% \alnote{Why 15?
%   Previously, it has been 16.} \athotanote{random queries}
% \alnote{Even if it is random, why should the average \# be different
%   in this case compared to the other case?} \athotanote{hmm.. i am not
%   sure. 15 is the average i got after looking at the outputs} 


% This is because, every time the replica-agent finds an
% exchange partner in $N_R/2$ steps, we observe that this needs to be
% repeated 2-4 times as the exchange does not proceed due to conflict
% with another replica.
% Each query requires $0.1\,s$,


$T_{EX}$ is primarily determined by $T_{find}$, i.\,e.\ the time necessary
to lock an exchange partner.  When searching for a replica randomly,
%\alnote{does randomly also apply to case II?}
on average it takes $N_R \over 2$ attempts to find a replica that is
in the \texttt{done} state.  However, finding a replica in the
\texttt{done} state is not enough to ensure an exchange attempt.
Given that there are several ``active exchanges'' being attempted,
often the reverify step leads to an aborted exchange attempt; a
reverify step must occur to ensure there has been no change in the
states of either of the two replicas involved.  The exact number is a
random variable, determined by the number of replicas, the
distribution of states and whether the attempt to find a replica is
random or sequential. Empirical observation suggests there are between 2 and 4
find and reverify attempts before an exchange is attempted.  But in
general, replicas contend with each other to lock a partner for
exchange; this gets worse with increasing numbers of replicas, i.e.,
$T_{find}$ increases with increasing $N_R$.  Specifically, for the random
access case, we find $T_{find}$ is $2 \times N_R \times 0.1$ (where $0.1$
is the typical time to set/get a value to/from the advert server), that is $6.4+0.7=7.1\,s.$
% \alnote{Should we argue with $\frac{N_R}{2}$*4 (in worst case) as
%   previously described or is this too complex?}
% On average the replica-agent must scan
% the list of replicas 2-4 times before it finds a potential partner for
% an exchange (due to conflicts with other attempting replicas).  As a
% consequence of the data-structure and pairing approach used, we find
% that $T_f$ is $0.1\times64=6.4\,s$.
The process of exchanging the
states and temperatures via the advert server and writing a new
configuration file takes, $T_{file} = 0.1\times 4 +0.2 = 0.6\,s$. % (And
% even in the other scenario, where the replica in question is not the
% initiator of the exchange, it has to wait for the other replica to set
% its state as complete, so that it can restart).
$T_{state}$ is the time it takes to update the state in the advert
server, which is $0.2\,s$.  Thus, $T_{EX}$ is $7.1+0.6 +0.2=
7.9\,s$.  It should be noted that $T_{EX}$ is highly dependent on
the actual implementation. While the current implementation is kept
simple on purpose, this value can be improved in a more sophisticated 
implementation.

% \alnote{In the other cases the state must be updated in the
%   advert server as well. Why don't we consider this in the other cases
%   as well?}\athotanote{in the other two cases, the bigjob agent makes
%   the state changes} \alnote{No matter who does the state changes, the
%   time needed should be about the same. Either we put it into the
%   model or we don't. If so, then consistent} \athotanote{we do include
%   the time to update the states. but that is included in $T_W$} 
%   \alnote{Should we just state that there is no central 
%   coordination overhead since the sub-job must not be restarted? And the 
%   time needed to spawn a local process is negligible.} \athotanote{fixed}

% \alnote{The following paragraph can be removed.}
% We also implement a random search of replicas by each replica-agent when the total 
% number of replicas if the ensemble is 128 or more, which reduces contention. 
% It should be noted here that in the centralised implementation of
% asynchronous RE, each query took only 0.01 s. But in
% the decentralised version, each query is shown as 0.1 s. This is due to the fact 
% that each replica-agent stores its replica's data (state, energy, temperature) in 
% a unique directory and changing the directory after each query 
% causes the action to take longer.\alnote{Would it be possible to improve the
% data structure used?}\athotanote{it might be possible..}



% \alnote{Is this the reason why the query takes
%   0.11\,sec instead of 0.01\,sec now? These would be a severe
%   performance degradation.}

% \athotanote{ do we need this para?} In the decentralised implementation, there are many pairs of
% replica-agents negotiating the exchanges concurrently. While this
% causes the time taken per unit exchange to go up, more exchanges occur
% in unit time.

As there are $N_R \over 2$ concurrent pairs, substituting the above
values in equation~\ref{eq:totaltime}, we get:
\begin{eqnarray}
T=  {1 \over p} \times {{(71\times {128\over 16}}) + {(7.2+0.7)\times 128\over 16}} = {1 \over p} \times 631~s.
\label{eq:decent}
\end{eqnarray}

% \subsection{Summary}

%\alnote {one could ask why is the config file staged in case I and II then?} 

% \begin{figure}
% \centering
% %\subfigure[Control Flow: Decentralized Replica Exchange]{
% \includegraphics[width=0.9\textwidth]{asyncre.pdf}
% %\label{fig:async:b}
% \caption{\small Decentralized control flow: In the decentralized asynchronous RE, for  each replica there is a replica-agent which individually manages the replica.}
% \label{fig:decent}
% %\vspace{-1em}
% \end{figure}

%\alnote{we should write Case consistently with small or capital letter}
% We have to bear in mind that while Case II and Case III both implement the same asynchronous RE algorithm, they do it differently.
% At first glance it appears to be a question of philosophy, whether to
% let the replicas be managed by a master or to let each replica be
% managed individually.
%There could be implications effecting the performance of the
%algorithm. Where as in Case II, the master has to manage all the
%replicas and since it can only manage one replica at a time, although negligible, it is a cause for concern with large number of replicas. %The effect could be negligible and might now effect the overall performance.
%But the decentralized version (Case III) has no
%such issues as each replica is managed individually. % \jhanote{The distinction between Case 3 and 2 needs to
%  be made more clear. The following is ``implementation detail''. What
%  is the conceptual difference between Case 3 and Case 2?}

\subsection{Asynchronous (centralised) RE}

%As before, the average time the BigJob agent takes to
%start one replica is 0.3 s. For 32 replicas, it is 9.6
%s. Thus, the BigJob agent finishes starting the last replica before the a replica completes its run. 


% ($\eta$ is 1). % \alnote{Not sure
% whether this view helps us here. I am aware of the sequential part
% at the central master, but couldn't that be considered
% quasi-parallel?  Also, the same limitation would apply to the
% synchronous case. An option would be to introduce a $\eta_{MD}$ and
% $\eta_{mgmt}$ in equation~\ref{eq:totaltime}: one for the MD part
% and one for the coordination part.}

% \alnote{I 
%   think this conflicts with the ```deterministic'' approach we
%   discussed yesterday}

% \alnote{``It should be noted that we also implemented a
% special case for an ensemble containing a very large number of
% replicas, where the RE-Manager tries to find a partner randomly. 
% We observed that this only improves the performance when the ensemble
% contains more than 128 replicas.'' Either we should
% state how much this effects $T_f$ or we should not mention it in my opinion. 
% to be discussed} \athotanote{can be deleted}


% If the RE-Manager tries to find a partner for exchange in a linear fashion, 
% it would have to go through 0 to 32 replicas. On average, it would go 
% through 16 replicas to find a partner. $T_f$ is $16\times0.01=0.16\,s$. 
% % \alnote{i.e. per average 16 queries are necessary in
% %   order to find a partner? Why 16?} \athotanote{explained} 
% $T_{ex}$ is the time it takes to
% update the configuration files and copy them locally. That would be
% $({0.2})\times 2=0.4\,s$. $T_{mgmt}$ is the time it takes to
% resubmit the pair of replicas to the advert server, which is
% $0.11\times 2 = 0.22$ s. Therefore, $T_{X}$ is
% $0.16+0.4+0.22=0.78$ s. 

As in the decentralised case, the asynchronous centralised RE
algorithm does not require synchronisation between {\it all} replicas
in order to transition a replica-pair from \texttt{run\-ning} to
\texttt{done} state and $T_W = 0$ by definition. Using a centralised
implementation the time to find an exchange partner ($T_{find}$) can be
reduced; however, this comes at a tradeoff that there is some
contention at the master. Also, we observed some delays at the BigJob
agents during the startup of the replicas sub-jobs mainly due to the
fact that the BigJob-Agent is single-threaded and thus, is busy
processing other replicas after their termination.  Specifically, the
time-to-submit a replica-pair to the BigJob-Manager, 
i.\,e.\ two replica sub-jobs, is in the asynchronous (centralised) case 
in average 1.1\,s. This is 0.5\,s longer than in cases
without contentions at the BigJob-Agent -- in these cases the submission
of two sub-jobs requires only 0.6\,s.
% post-processing of a sub-job (replica) requires
% 1.1\,s. % \jhanote{Abhinav: Either we multiply by 2 to get pair-wise
% %   value for $T_r$ or we elaborate/explain better} \athotanote{explained} 
% In average every second replica sub-job has a contention with another
% sub-job processed by the BigJob-Agent. Thus, the average time per replica
% pair to restart a replica sub-job is: $T_r = 1.1 / 2 \times 2 = 1.1\,s$.
% For 32 replicas, that would be 35.2 seconds. Therefore, each replica
% pair waits $35.2/16=2.2$ s at BigJob-Agent before being
% restarted. But, we observed that - as the experiment progresses - the
% BigJob-Agent would be processing smaller groups of replicas and
% starting smaller groups of replicas. 
%\jhanote{Abhinav: Andre and I agreed to use 2.2 and not 1.1. Table and values in the Equation need  to be updated to reflect this} \athotanote{but this is the worst  case possible and as the experiment progresses, the bigjob agent  frequently moves between starting and marking jobs as done - instead  of doing 32 in a row. if we make it 2.2 instead of 1.1 there would  not be a difference between synchronous and centralised. and the  equation/model and the data dont support each other.} \alnote{OK,  my original argument fall through after thinking about it. Assuming  a average of waiting for 16 replicas increase $T_r$ even further.   I tried an alternative explanation. Please check!} \alnote{OK,  we settled for a more generell argument talking about sub-job  submission times without speculating how many sub-jobs might  block the processing of another sub-job.}

% On average replica pairs after an exchange actually wait 2.2/2=1.1 s
% to be restarted by the BigJob agent.  \alnote{Shouldn't we use the
%   average number of sub-jobs that a new job has to wait for as an
%   argument?}  Therefore, the value of $T_W$ is the same as $T_r$.
% \alnote{the last sentence can be removed in my opinion.}


% there is some amount of
% waiting involved, which is mainly caused by the single-threaded BigJob
% implementation, which leads to some delays at the agent.
% In particular, a delay arises when new sub-jobs in cases where the
% BigJob agent was busy processing other sub-jobs after their
% termination. In this scenario the post-processing of a sub-job
% requires 1.1\,s about 0.2\,s more than in the synchronous
% case. 
% \alnote{Abhinav please check: in my opinion this is the worst
%   case not the average case. The second replica eg has to wait less.}
% \athotanote{well...}

% In the worst case, the first replica pair that is restarted has to
% wait for the processing of the other remaining 30 replicas: $1.1\,s
% \times 30 = 33\,s$. This corresponds to $T_{r}$ of $1.1\,s$ per
% replica. $T_s$ is 0 in asynchronous RE. Thus, $T_W$ is 1.1 s.\alnote{I
%   still have some issues with that formula i) we don't model anything
%   related to the central master ii) I am not sure whether the per
%   replica view is correct. What is a reasonable $\eta$
%   (resp. $\eta_{MD}$ and $\eta_{mgmt}$) for async RE?}

In contrast to the synchronous case, $T_{EX}$ for the asynchronous
centralised case has a $T_{find}$ component since replica pairs are
dynamically determined and not fixed. $T_{find}$ depends on the overall
number of replicas ($N_R$), which determines the number of records the
RE-Manager has to scan in order to find an available replica. On
average the RE-Manager must search through $\frac{N_R}{2}$ replicas before it
finds a partner. Although the search for a replica is random like the
decentralised implementation, there is no need for reverifying, as due
to centralised control there is no contention in replica pairing,
i.e., exchanges are made by the RE-Manager and only one exchange takes
place at a time.  An advert query for a replica state takes
0.01\,s. Note this is a factor of 10 less than the decentralised
implementation where there was a connection to the advert server for
every replica-agent; here there is a single connection. For 32
replicas, the RE-Manager requests on average 16 other replica states
before it finds a partner; thus, $T_{find}$ is in total
$0.01\times16+1.1=1.3\,s$. Both $T_{file}$ and $T_{state}$ are same as in the
synchronous case, i.\,e.\ $T_{EX}$ is thus: $1.3+0.4+0.2=1.9\,s$.

% \alnote{\textbf{Abhinav: Could you improve the description of this paragraph, please? In particular
% I don't really get the last sentence. Is this the worst or average case? Ok,
% this is the worst case. How do we get the average case?} \athotanote{to do..}
% As previously described, the BigJob agent has an average delay of 0.92\,s 
% for discovering that a sub-job has terminated and for book-keeping, i.\,e. for
% updating the list of free/busy nodes. But in this
% implementation, we also make the BigJob agent retrieve the energy
% and temperature from the output files, which adds 0.2 s per
% replica. In summary it takes 1.12 $\times$ 32 = 35.84 s for 32
% replicas. 
% % \alnote{This somehow assumes that all replica terminate at
% % the same time, right?} \athotanote{as this is the first iteration, 
% % all replicas will end within those 35.84 s of the 1st and last replica. on a homogeneous resource} 
% As soon as the first pair of replicas are marked as done by the
% BigJob agent, the RE-Manager makes the exchange between that pair of replicas 
% - and that pair of replicas would wait 32.8 s ($35.84-1.12\times2-T_X$) 
% before the BigJob agent is able to restart them.} It can be seen from this 
% equation that later pairs of replicas wait a smaller amount of time before 
% they are restarted by the BigJob agent (the BigJob agent alternates between 
% starting all replicas that are new and ending all replicas that are done). 
% On average, a replica waits $T_r$ = 32.8/32=1.02 s to be restarted. 
%   \alnote{I don't understand 
% the notion of pair in this context. Is 32 sec == $T_W$?}\athotanote{is it better now?}\alnote{yes. I think
% we can shorten it and directly break the 32 sec down to $T_W$ for 1 replica: 32.8/32=1 sec.} \athotanote{done}

% \alnote{\textbf{I propose to take the rest of this paragraph out and
%     not go into the details of $T_r$ and $T_w$:} It should be noted
%   that the timelines of the RE-Manager and the BigJob agent run
%   concurrently. Also, the time RE-Manager waits for the next replica
%   to become available, $T_w$ is included in $T_r$. This is because we
%   already included the time each replica waits at the BigJob agent
%   before it is restarted ($T_r$). This would be the same time a
%   replica waits for the next replica to become available. Thus, $T_W$
%   is $1.1\,s$. } \alnote{That is somehow imprecise now! We should only
%   include it in one component.} {\athotanote{this $T_w$ is not
%     $T_W$. it is only part of $T_W$(time waiting for the next replica
%     to become available.  therefore, $T_W$ is still left with
%     $T_r$(time waiting to be restarted at the bigjob agent }
%   \alnote{My point is that the last sentence confuses the reader:
%     either it is $T_f$ or $T_w$.} \athotanote{corrections made.}
%   \athotanote{can be deleted}



% The point we are trying to make is that the BigJob agent might not
% be free while the RE-Manager submits the replicas to be
% restarted. But as the experiment progresses, instead of all the
% replicas in the ensemble running in synchronisation, pairs of
% replicas would start and end together.  By the time the BigJob agent
% finishes processing the last of the replicas that finished running,
% the RE-Manager would have re-submitted 15 pairs of replicas to the
% advert server for restarting by the BigJob agent. In the next couple
% of s the rest of the replicas would have been exchanged and
% submitted to the advert server.

% \alnote{TODO Abhinav: roundup numbers to one decimal place}
% \athotanote{done}

% Although there are $N_R \over 2$ concurrent pairs, the exchange at the
% RE-Manager is carried out sequentially; thus the effective number of
% concurrently exchanging pairs is 1.  

As in the synchronous case the effective number of concurrently
exchanging pairs is 1 due to the fact that exchanges are sequentially
carried out by the RE-Manager.  Substituting the above values in
equation ~\ref{eq:totaltime}, we get:
\begin{eqnarray}
T=  {1 \over p} \times {[ {(71\times {128\over 16}}) + (0 + 1.9)\times 128]} = {1 \over p} \times 811~s.
\label{eq:cent}
\end{eqnarray}

\section{Scale-Up and Scale-Out: Experiments and
  Results}\label{sec:performance}

To evaluate the scaling properties of the different RE algorithms and
implementations, we conducted several experiments on TG and LONI
resources. We initially increased the number of replicas while keeping
the number of machines constant (``scale-up"); then for the
asynchronous-centralised case, for a given number of replicas, we
varied the number of distributed machines used (``scale-out''). In
this section we outline the experiments conducted. Results of these
experiments establish the advantages of asynchronous formulations for
both scaling-up and scaling-out.

% \alnote{Should we reference the model used? Not sure which model it is
%   Hepatitis or HIV?}

%
%%%%% FIGURE %%%%%
\begin{figure}
\centering
\includegraphics[width=0.8\textwidth]{../data/scale_up.pdf}
\caption{\small \textbf{Scale-Up Performance for 8 to 256 Replicas:}
  The graph shows the runtimes for the different RE implementations.
  The ratio between the number of exchanges and number of replicas is kept constant. Each replica is assigned 16 processors and run 500 timesteps. 
  The asynchronous decentralised RE implementation shows the best
  scaling behaviour. Both centralised RE versions scale less well
  mainly due to the limitations of the single master, which becomes a
  bottleneck.}%\athotanote{there is a graph with log scale in svn, if needed.} 
\label{fig:scaleup}
\vspace{-1em}
\end{figure}

% \alnote{The notation of the x- and y-label is kind of confusing. On
%   the y-axis the unit is in parentheses. On the x-axis
%   \#exchanges. Since we keep the ratio between \#replica and
%   \#attempted exchanges constant, do we need both values on the
%   x-axis?}

\subsection{Scale-Up}


{\it Experiments: } To understand the scaling behaviour of the three
cases, we compute the $T$ for 4, 8, 16, 32, 64, 128 and 256 replicas,
for 16, 32, 64, 128, 256, 512 and 1024 exchanges respectively. This
fixes the ratio of the number of exchanges to the number of replicas
($N_X \over N_R$) to 4.  Each replica is configured to run 500 time-steps
before an exchange is attempted, and is allocated 16
processors. Experiments up to 64 replicas are performed on {\it
  QueenBee}, while experiments with 128 and 256 replicas are done on
\emph{Ranger}; this is because \emph{QueenBee} only allocates a
maximum of 2048 processors per job request and getting a 2048 processor allocation involved extremely large waiting times. We have normalised the data to factor in the difference in performance of {\it Ranger} and {\it QueenBee}.

The reasoning behind this design of experiments to understand the behavior 
of the different implementations as we increase $N_R$. We believe that 
the scale-up properties of  the centralised implementations are easy to 
understand and predictable. But the reason we did more experiments with 
even larger $N_R$ is to ascertain the behavior of the decentralised 
implementation. Now that we have done the seen the results for 256 replicas, 
we are confident that we can predict the performance of different 
implementations for even larger $N_R$. \athotanote{check preceding sentence}
\alnote{bold claim}


%\subsubsection{Scale up performance: Results and Analysis}
\alnote{please double-check that the new abbreviations are used! $T_{EX}$}
{\it Results:} Figure~\ref{fig:scaleup} shows the results obtained by
running the scale-up experiments mentioned.  % The ratio of the number
% of exchanges ($N_X$) to the number replicas ($N_R$) is kept constant,
% i.e., the number of generations is a constant for different replica
% counts. 
As a consequence of the ratio of the number of attempted exchanges to
of replicas being a constant, the $T_{MD}$ term across the different
values of $N_R$ is essentially the same; hence comparison between
different cases will reveal differences in the coordination cost ($T_W
~and~T_{EX}$).  Thus, as $N_R$ increases, the variation in $T$ is due to
the coordination component -- terms $T_{EX}$ and $T_W$ in
Equation~\ref{eq:totaltime}.  The increase, however, is not uniform
across the three implementations: it is largest for
synchronous RE, and the least for asynchronous (decentralised) RE.  We
analysed $T$ and the values of its components for 32 replicas in
section~4.1; we use that analysis as the basis to understand the scale-up
behaviour of the three cases.
 
In the synchronous RE algorithm, there is an explicit synchronisation
of all replicas; thus early replicas wait for other replicas to
finish. As can be seen from Table~\ref{table:repex_perf}, $T_W$ is a
major component of the $T$.  As $N_R$ increases, the number of
exchanges at a given exchange step increases; consequently the total
coordination time at each exchange step increases.  For a value of
$N_R$ of 64 and $N_X$ of 256, using Equation~\ref{eq:totaltime}, the
coordination time is $(0.6+2.8) \times 256 = 870.4$.  The difference
in coordination times ($435.2$ s) accounts for the difference
in $T$ for 32 and 64 replicas in the
Figure~\ref{fig:scaleup}. % The actual difference observed in $T$ from 32 to 64 replicas is 366 s. \jhanote{Abhinav: Please provide value of
%   the difference in T here please} \athotanote{mentioned}
A similar analysis was performed for different replica and exchange
counts and the values obtained are in agreement with the empirical
data in Figure~\ref{fig:scaleup}.

% Since replicas are currently started sequentially by the master, a
% delay between the start and thus the termination of the first and last
% replica exists.
% From table1\ref{}, it can be seen that in synchronous RE, the largest
% component causing slow down is $T_W$. 
% (number of pairwise
% exchanges) increases, while
% Thus by substituting the values from the table in the
% eqn2.2, and changing $N_X$, we can understand how the values in graph1
% are possible.

% In the asynchronous (decentralized) RE, as can be seen from the
% table1\ref{}, the individual values for each of the components that
% make up the time to make a pairwise exchange are not smaller. In fact,
% $T_X$ is very large when compared to other cases. 

% In contrast to the other cases, the exchanges are performed by the
% replica-agents instead of having the RE-Manager make the exchanges
% centrally (\jhanote{or serially}).  Thus finding a partner -- and thus
% $T_f$, has two components: a search and reverify stage, which must
% both be successful before an exchange is carried out.
% %a successful search and reverify stage.

% When searching for a replica randomly, on average it takes $N_R \over
% 2$ attempts to find a replica that is in the \texttt{done} state.
% However, finding a replica in the \texttt{done} state is not enough to
% ensure an exchange attempt.  Given that there are several ``active
% exchanges'' being attempted, often the reverify step leads to an
% aborted exchange attempt; a reverify step must occur to ensure there
% has been no change in the states of either of the two replicas
% involved.  The exact number is a random variable, determined by the
% number of replicas, the distribution of states and whether the attempt
% to find a replica is random or sequential. Empirical observation
% suggests that between 2-5 find and reverify attempts before an
% exchange is attempted.  But in general, replicas contend with each
% other to lock a partner for exchange; this gets worse with increasing
% number of replicas, i.e., $T_f$ increases with increasing $N_R$.
% Specifically, for the random access case, we find $T_f$ is $2 \times
% N_R \times 0.1$ (where $0.1$ is the typical time to set/get a value
% to/from the advert server)

% and for the sequential access case, we find
% $T_f$ is $2.5 \times N_R \times 0.11$. 
% We implemented a serial search
% for upto 32 replicas, and a random search for $N_R$ value greater than
% 32.
% and from 64 to 256 replicas we implemented a random search. 
%\alnote{in the centralised as well, doesn't it?} 
% For the random access case,\alnote{I think in sec 4 we only talk about
%   the random case now} 

For the decentralised implementation, $T_{find}$ has a strong $N_R$
dependence. We find $T_{find}$ is $2 \times N_R \times 0.1$ (where $0.1$ is
the typical time to set/get a value to/from the advert server).  For
example, in Equation~\ref{eq:decent} we see that for 64 replicas,
$T_{find}$ is approximately $2 \times 64 \times 0.1 = 12.8$ s.  Thus the
difference in $T_{EX}$ when $N_R$ is 64 versus 32 is 51.2s, which
accounts for the bulk of the increase in overall $T$ when $N_R$
changes from 32 to 64. % The actual difference observed in $T$ from 32 to 64 replicas is 18 s.\jhanote{Abhinav: Please provide value of the
%   difference in T here please again}.\athotanote{done}
The variation in $T$ for other values of $N_R$ can be
accounted for by changing values of $T_{find}$.

% For 64 replicas, $T_f$ is approximately $1.5 \times 64 \times 0.11 =
% 10.56$ s.  Only $T_f$ changes with the number of replicas.

%  \alnote{Next two sentences: How much does this really impact the
%    decentralised case?  Also, we defined $T_s$ as 0}

In contrast to the decentralised implementation, $T_{find}$ in the
asynchronous centralised case is only weakly dependent on $N_R$; however,
since the RE-Manager makes the exchanges serially, we still see an
increase in the time-to-completion with increasing $N_R$. % Using
% Equation~\ref{eq:cent} for 32 replicas as a basis:
For $N_R =$ 64 and $N_X$ 256, using Equation~\ref{eq:cent} leads to a
new coordination ($T_W + T_{EX}$) time of $(0.8+1.1) \times 256 =
486.4\,s$, up from $243.2$ for $N_R = 32$. This change accounts for a
large component of the difference in $T$ as $N_R$ goes from 32
replicas to 64 replicas, as shown in Figure~\ref{fig:scaleup}. The
actual difference observed in $T$ from 32 to 64 replicas is 293
s. % \jhanote{Abhinav
%   provide specific values of T difference}\athotanote{done}.
We have verified this for different number of replicas and found it to
be consistent.

In addition to the above explicable changes arising from different
coordination costs, as $N_R$ increases different nodes of a machine
tend to perform differently, thus influencing the time-to-completion
of different replicas, but not always monotonically.  % Consequently,
As $N_R$ increases, the time to start and synchronise replicas
typically increases.

% When searching serially, we observed that the replica-agent
% finds a partner after looping over the all the replicas between 2 and
% 2.5 times. 
%makes between 1 and 1.5 times $N_R$ calls to find a partner.

\subsection{Scale out}
\alnote{Please reference figures in text.}
{\it Experiments:} To evaluate the performance of the different algorithms 
when scaling out across many machines, we
used the following TeraGrid and/or LONI resources: \emph{QueenBee,
  Eric, Louie} and \emph{Oliver}. The experiments conducted are 8, 16
and 32 replicas making 32, 64 and 128 exchanges. These exchanges are
repeated on 1, 2 and 4 machines while distributing the number of
replicas equally on each machine. It is important to note that all
experiments are conducted using four BigJobs, irrespective of the
number of machines used; varying the number of BigJobs as well as the
ratio of replicas per BigJob effects overall performance.  Another
important point to note is that only the experimental runs where all
the four big-jobs \alnote{when referring to the pilot job, BigJob should be consistently
spelled as big-job} become active within $30\,s$ of each other on
submission to the resource manager are considered and included in the
results. This is to remove the queue wait time from the equation and
focus on the runtime. Each experiment is repeated 5 times.

%
%%%%% FIGURE %%%%%
\begin{figure}%
\centering
\includegraphics[scale=0.35]{../data/sync_scaleout.pdf}\qquad
\caption{\textbf{Scale-out performance for 8, 16 and 32 replicas, synchronous:} The experiments were done on LONI resources and repeated at least 5 times. The error bars denote standard error. As the number of machines increases, the time-to-completion increases in general mainly  due to higher exchange costs ($T_{EX}$) caused by e.\,g.\  remote file copies and additional synchronisation costs.}
\label{fig:scaleout_sync}
\end{figure}

\subsubsection{Scale out - Synchronous}
\alnote{Please use our model from sec 2}
\alnote{Also, the text does not describe the behaviour with different no. of replicas. Maybe we should merge fig 4-6 into one figure using e.g. 32 replicas.}
{\it Results:} The results are shown in the graph~\ref{fig:scaleout_sync}. 
We see that as we go from 1 to 2 and 4 machines, the time to completion 
increases. The difference in performance that we see as we go from 1 to 2 
and 4 machines could be caused by: (a) increased synchronisation costs, 
(b) additional time taken for remote file staging, and (c) higher communication 
costs to contact the advert server. But the machines we used are similar 
in nature and the NAMD runtimes are almost close to each other. But still, 
the NAMD runtimes are not constant and vary slightly from one to another. 
We do not observe any difference in the 
communication times with the advert server. Therefore, the factors
effecting the $T$ is the additional time to stage the configuration 
files and the slight overhead added to the synchronisation cost.

\alnote{Didn't we decide to abbrev. seconds with s?}
To copy the configuration file locally it takes 0.009 seconds and to a remote machine it takes 0.379 seconds, approximately. The difference in time is 0.37 seconds. In the graph~\ref{fig:scaleout_sync}, going in the 8 replica case, going from 1 machine to 2 machines, the $T$ changes from 608 to 623 seconds. In the 2 machine case, the RE Manager would have to stage 4 configuration files remotely every exchange step. That would be $4 \times 8 = 32$ times in the whole simulation. That is an additional $32 \times 0.37 = 12$ seconds. And, 608+12= 620 seconds. This holds for 16 and 32 replica cases. 

\begin{figure}%
\centering
\includegraphics[scale=0.35]{../data/decent_scaleout.pdf}\qquad
\caption{\textbf{Scale-out performance for 8, 16 and 32 replicas, asynchronous (decentralised):}  The experiments were done on LONI resources and repeated at least 5 times. The error bars denote standard error. As the number of machines increases, the time-to-completion remains constant as there are no remote file copies or synchronisation costs.} \alnote{Please insert proper caption}
\label{fig:scaleout_dec}
\end{figure}


\subsubsection{Scale out - Asynchronous (decentralised)}
\alnote{This is a very short analysis. I think the decentralised algorithm and thus, less synchronisation
should also play an important role?}
{\it Results:} The results are shown in the graph~\ref{fig:scaleout_dec}. We see in the graph that as we go from 1 to 2 and 4 machines, we do not observe a perceptible change in the time to completion.
The time to completion almost stays constant from 1 machine to 2 and 4 machines. We believe that this is due to the fact that there is no remote file staging involved in this implementation. The fact that this is the asynchronous RE algorithm and a decentralised implementation also play a role. There is no synchronisation involved in this implementation. 

\begin{figure}%
\centering
%\subfigure[8 replicas]{
\includegraphics[scale=0.35]{../data/cent_scaleout.pdf}\qquad
%\subfigure[16 replicas]{\includegraphics[width=0.47\textwidth]{../data/scaleout_16.pdf}}\\
%\subfigure[32 replicas]{\includegraphics[width=0.47\textwidth]{../data/scaleout_32.pdf}}
\caption{\textbf{Scale-out performance for 8, 16 and 32 replicas, asynchronous (centralised):} 
   The experiments were done on LONI resources and repeated at least 5 times. The error bars denote standard error. As the number of machines increases, the time-to-completion increases in general mainly
  due to higher exchange costs ($T_{EX}$) caused by e.\,g.\  remote file copies.}
%   \alnote{Can we increase the linewidth and order the entries in the legend in the same order
%   as the lines from top to down?} \athotanote{reordered the legend. could not increase linewidth.. tried a few things but did not work}
\label{fig:scaleout_cent}

% \alnote{16 or 32 replicas? Why is the centralised async performing
%   better for 8 replicas and a small number of machines? We should
%   consider showing only the 32 replica scenario. bar chart instead of
%   lines? opinions?}  \alnote{Could you check the figures again. They
%   look very pixelated. Further I would propose to just have numbers on
%   the x-axis und not 1Machine, 2Machines,.... If otherwise, please use
%   a space between 1 and machine.} \jhanote{the labels have to be made
%   larger. Currently can't be read}

\end{figure}


\subsubsection{Scale out - Asynchronous (centralised)}
\alnote{Why is there a sharp decrease from 1-2 and an light increase from 2-4 machines?}

% \alnote{Is this correct? When do we
%   start to assign the replicas? If the queue time is included, we had
%   at least an explanation for the high stddev-} \alnote{How many
%   repeats?}

% This is necessitated by the fact that the number of BigJob
% agents effects the overall performance.
% If there are more BigJob
% agents per number of replicas, the number of replicas under each
% BigJob agent would be smaller. And each replica would face smaller
% waiting times at the BigJob agent.



%In the synchronous RE, the total time to completion consistently increases with increasing number of machines. But we do not see a consistent increase of decrease in the other two implementations. The changes that we do see are within error bars. 
% In the synchronous RE, with increasing number of machines, the synchronisation cost only increases. Which is even more true in a heterogeneous environment. Also, with more machines, the RE-Manager has to stage-in more remote files to retrieve the energy and temperature of a replica and  stage-out more configuration files to remote machines. A remote file transfer is much more expensive than a local file copy. These two factors make the total time to completion consistently increase with increasing number of machines. 

% Also, in this case the
% BigJob agent retrieves the temperature and energy of a replica and
% posts the value in the advert server, which is much more efficient
% than staging-in the output files to the local machine to retrieve the
% values.
% But the RE Manager still has to stage-out the modified
% configuration files after each exchange. From 1 machines to 2
% machines, we see a change in $T$ within error bars and with 4 machines
% we notice an increase outside the error bars. That shows the cost of
% staging-out files to 3 remote machines.

% \alnote{The scale-out analysis needs to be improved to get on par with the
% scale-up analysis.}\jhanote{Please check and refine as you deem fit}
% \alnote{Sounds good for the data we have.}
{\it Results:} The results obtained are shown in
Figure~\ref{fig:scaleout_cent}.  The time-to-completion $T$ increases
moderately or remains constant with higher number of machines, which
indicates that the scale-out behaviour is quite good.  As discussed in
section~\ref{sec:re_impl}, the asynchronous (centralised) RE algorithm
does not involve any synchronisation cost ($T_W$). 
Thus, the loose coordination between replicas at the exchange stage helps to 
provide the flexibility to use multiple resources, which would not be possible if
there was tighter coordination between replicas at the exchange stage.
The main contributor of the small increase is $T_{EX}$ are the higher
costs for remote file transfers; fluctuations in remote operations also account for the somewhat higher
variations in the $T$. We observe in the graph that the increase in $T$ from 1 to 2 and 2 to 4 machines is not always consistent. But the inconsistencies are in order of seconds and almost always within the error bars. 

% When $N_R$ is between 8 and 32 and number of BigJobs ($N_{BJ}$ is 4.

%\alnote{Not sure whether this fits in here. 1 BJ means we only used one resource?}
% Comparing with the scale-up experiments where only 1 BigJob was used;
% it can be seen
% % the figure~\ref{fig:scaleup}, we can see
% that $T$ for 8, 16 and 32 replicas goes up when $N_{BJ}$ is
% reduced. It should also be noted that for $N_R = 64$ and $N_{BJ} =4$,
% $T$ tends to increase, but it would still be less than what it would
% be for 1 BigJob.


% Therefore, we can say that if ${N_R \over N_{BJ}
% }\approx$ constant then $T \approx$ constant.

%In the asynchronous (decentralised) RE, compared to the centralised implementation, there is no need to stage-in or stage-out the configuration files at any time. The energy and temperature values are exchanged via the advert server and the configurations files are modified locally by the replica-agent. Thus, even with increasing number of machines, we don't actually see a difference in $T$ outside the error bars. 
%\alnote{To-be-answered: Why is case II faster than III in the 8rep/[1,2] machine scenario?}
%We see in figure4(a)\ref{} that the asynchronous (centralised) is faster than the decentralised implementation. This is because with a small number of replicas (8 in this case), it more efficient to conduct the exchanges in a centralised manner.  This way we avoid the contention for exchange partner that is seen in the decentralised implementation. 


% \alnote{Why is that? fewer replicas should also mean less load on the
%   advert service? How is this reflected in our math. model?}
% \alnote{TODO: Local vs. distributed coordination, BigJob startup
%   times}


% Overall, there is not much difference between the local and distributed 
% runs in any case. The reason being that there is very
% little interaction between the replicas. A very small configuration
% file is staged to the remote machines after an exchange, which will
% not add more than a couple of seconds per exchange. The rest of the
% co-ordination is done via the advert service, which is usually located
% on a remote machine in any case. Each query to the advert server is in
% the order of milli seconds and does not produce a noticeable effect on
% the performance.

% \subsubsection{Scale out performance: Results and Analysis}


%  In Figure~\ref{fig:24machines}, we see the performance
% of all three cases when run in a distributed manner across two
% machines. As the data that is exchanged between replicas is very
% small, the cases I, II and III behave in a similar manner to the way
% they behave on a single machine. \alnote{we should add some numbers
%   and maybe a graph for an example scenario: x: machines y:
%   time-to-solution} Again, asynchronous RE is better suited for
% distributed runs and the decentralised implementation scales best. The
% reason is, since each replica has its own replica-agent, there is no
% need to transfer any files between machines. The required
% configuration files are created locally by the replica-agent.


%In Case I, the pair-wise replica exchange can occur only between replicas of the same generation. Therefore, each exchange step is attempted only after all the replicas have finished running. After the exchange, all the replicas are restarted sequentially. This inserts a delay between the start time of the first replica, the last replica and the replicas in between. %As more resources become available at different times, the replicas already running or done are forced to wait for the newly running replicas to finish before moving on to the next exchange step. %Each exchange step is counted as an exchange.
%In Case II, the pair-wise replica exchange can take place between any two replicas in the ensemble irrespective of generation. As more resources become available, the new replicas join the ensemble immediately and the replicas already running are not restrained from attempting exchanges or restarting. This gives the asynchronous or synchronous RE a slight advantage. But with a large number of replicas we could easily see large difference.
%\athotanote{Further, we show performance gains by running across more than one machine. By running across more than one machine, we demonstrate the ability to divide the jobs into smaller sub-jobs and then distribute them across a number of machines, thereby reducing the risk of long queue wait times on an over-crowded resource. In Figure~\ref{fig:graph}, it can be seen that the asynchronous RE time to completion improves almost by a factor of 3 when moving from one machine to four machines. This was caused due to the fact that when the experiment was done on one machine, by the time the experiment ended, only 64 cores were allocated by the resource manager. But on the other hand, when the experiment was launched across four machines, it received an allocation of 64 cores on each of the four resources. The improvement that is seen in the case of synchronous RE from one to four machines is also due to a similar reason.} %The asynchronous RE appears faster by a couple of minutes due to the fact that when the BigJobs become available randomly, the synchronous RE has to wait for the newly running replicas to finish.

%slightly over 2 machines, but again increases over 4 machines. This is due to the fact that the experiments have been run only a handful of times but, over time, it can be assumed that it will result in reduced queue wait times.

\subsection{Comparison of Scale-Up and Scale-Out Results}
In the preceding sections we have presented the scale-up and scale-out results of the different implementations. We would now like to compare the scale-up and scale-out results. Let us first consider the synchronous RE case with 32 replicas. The reader might have noticed that while it takes 1008 seconds to complete the experiment with 1 big-job in the scale-up case, it takes 931 seconds to complete the experiment on 1 machines with 4 big-jobs. The reason for this is the difference in the ratio of replicas per BigJob-Agent in the two cases. In the first case, there is one BigJob-Agent for 32 replicas, where as in the second case there are four BigJob-Agents for 32 replicas. Since the BigJob-Agent plays a major role in the management of the replicas, the number of BigJob-Agents effects the performance. This is true for  the centralised implementations more than the decentralised implementation. We do see an improvement in the decentralised implementation too, as the BigJob-Agents are used to start the replica-agents at the beginning of the experiment. For example, in the asynchronous (centralised) case with 32 replicas, it takes 816 seconds to complete the experiment with 1 BigJob-Agent and 638 seconds to complete the experiment with 4 BigJob-Agents on 1 machine. 
The improvement in performance is the most pronounced in the asynchronous (centralised) RE. There is a performance improvement in the other two cases as well, but since the BigJob-Agent is not used after the start-up of replica-agents in the decentralised implementation and  due to increased synchronisation costs in the synchronous RE, the improvement in performance in subdued. 

\section{Conclusion}
\label{sec:conclusion}

Following theoretical underpinnings
~\citep{parashar_arepex,DBLP:journals/jcc/GallicchioLP08}, in this
paper we investigate {\it traditional} and {\it advanced}
replica-exchange algorithms at unprecedented scales.  An important
motivation for this work has been to implement a framework for the RE
class of algorithms that can use general purpose and standard
capabilities available on production infrastructure, such as, the
Teragrid and LONI.  Additionally, our framework uses a flexible
pilot-job implementation, which enables effective resource allocation
for multiple replicas.
%that do not enforce a static model of resource usage or availability.
% \alnote{should we be here slightly more concrete and how BJ can help
% here?}

Results shown in figures~\ref{fig:scaleup} and~\ref{fig:scaleout_cent}
indicate that using algorithmic formulations that impose less tight
coordination constraints enables both good scale-up and scale-out
behaviour.  Algorithms based on asynchronous coordination are
typically more difficult to implement than synchronous ones; however,
we find that even with a simple, non-optimised prototype of the
RE-framework, the advantages of asynchronous formulations soon outweigh the synchronous formulations, i.e., as $N_R$ increases the
performance of asynchronous RE beats that of the synchronous RE.  

Our analysis shows that a fundamental trade-off is between the lower
cost of replica synchronisation at the exchange stage that
asynchronous formulations provide, versus the higher cost of
permitting dynamic replica pairing.  In an attempt to investigate an
optimal trade-off between these factors, and to demonstrate the
advantages of asynchronous RE, we implemented a centralised version of
the asynchronous RE with a lower cost of dynamical pairing than in
the decentralised implementation. Our initial results show promising
scale-out behaviour, but more work is required to separate and
understand fundamental algorithmic advantages from implementation
specific issues.



% and scale-out properties on different
% production level infrastructure, 
% Further, we compare the synchronous and asynchronous RE algorithms and
% implementations by modeling and repeating the experiments a reasonable
% number of times, so as to accurately quantify the scientific and
% performance benefits.

%\athotanote{is this right? }
% We are also going to have a wider group of replicas to look at for
% each replica as we are not pairing the replicas.

% Also, we have the usual advantages of using a pilot-job,
% such as reduced queue wait times by not having to submit to the queue
% at every step.  We also provide major advantages when compared to
% Parashar et al.

%  to run the asynchronous RE simulations,
% including the ability to run MPI
% jobs.
% ??We need to evaluate the performance of our models and compare with other models for conducting replica exchange simulations.


%%%%% FIGURE %%%%%
%\begin{figure}
%\centering
%\subfigure[Time to complete 64 exchanges on QB with two 64 core BigJobs and on both QB/Louie jointly with a 64 core BigJob on each machine.]{
%\includegraphics[width=0.40\textwidth]{figures/graph1.pdf}
%\label{fig:subfig3}
%}
%\hspace{0.5cm}
%\subfigure[Time to complete different number of exchanges on QB/Louie with a 64 core BigJob on each machine.]{
%\includegraphics[width=0.40\textwidth]{figures/graph2.pdf}
%\label{fig:subfig4}
%}
%\caption{\small In Figure 2(a), we can see the improvement in performance when run on more than one machine. It is due to the fact that usually the first queued job becomes active before the second on a machine and running jobs on more than one machine solves this problem. In Figure 2(b), we can see consistent performance over prolonged runs, making 32, 64 and 128 exchanges.}
%\label{fig:graphs}
%\vspace{-1em}
%\end{figure}
%%%%% FIGURE %%%%%

%We also propose to measure the frequency with which crosswalks occur with increasing number of replicas and measure the advantages due to a decentralized implementation in the full paper.

%With this asynchronous replica exchange mechanism we can improve the
%number of exchanges per unit time, a key parameter in judging the
%performance of a replica-exchange mechanism. \athotanote{is this
 % right? }  We are also going to have a wider group of replicas to
%look at for each replica as we are not pairing the replicas. Also, we
%have the usual advantages of using a pilot-job, such as reduced queue
%wait times by not having to submit to the queue.  Unfortunately we
%dont have results \jhanote{What results can we present -- any? some?},
%so we will say, (i) we establish the ability to scale-out (distributed
%and exa-scale) across different infrastructure (ii) compare the Async
%versus sync formulation at unprecedented scales \jhanote{At least
%  outline what infrastructure we / you are planning to use?} (iii)
%compare different implementations of the Async version
 

\begin{acknowledgement}
  This work is part of the Cybertools (http://cybertools .loni.org)
  project and primarily funded by NSF/LEQSF (2007-10)-CyberRII-01.
  Important funding for SAGA has been provided by the UK EPSRC grant
  number GR/D0766171/1 (via OMII-UK) and HPCOPS NSF-OCI 0710874. This
  work has also been made possible thanks to computer resources
  provided by TeraGrid TRAC TG-MCB090174 and LONI resources.
\end{acknowledgement}

\bibliographystyle{kluwer}
\bibliography{saga,literature}    
\end{document}


% and $T_W=T_w+T_r$, and where
%($T_w$), 

% \begin{eqnarray}
% T = {1\over p} \times [(T_{MD} \times  {N_X \over {N_R \over 2}}) + (T_{X} + T_{W}) \times N_X]
% \label{eq:totaltime}
% \end{eqnarray}

% \alnote{Using ex and EX as subscripts is very
%   confusing!} \jhanote{Why not use T$_X$ in lieu of $T_{EX}$? }

% Equation \ref{eq:totaltime} gives the total time to complete an RE
% experiment. The next equation aims to calculate the time to complete
% particular number of exchanges ($N_z$), where $0 \geq N_z \geq N_X$.

% \begin{eqnarray}
% T_z = {1\over p} \times [(T_{MD} \times  \lceil{N_z \over {N_R \over 2}}\rceil) + (T_{EX} + T_{W}) \times N_z]
% \label{eq:parttime}
% \end{eqnarray}

% Here $\lceil$ $\rceil$ denotes the ceiling function. The reason we put the term ${N_z \over {N_R \over 2}}$ under a ceiling function is because the ensemble of replicas run concurrently and for each concurrent run, $N_R \over 2$ exchanges are possible. Therefore, only the coordination and waiting costs effect the total time. 

% \athotanote{please suggest alternative terms if you find anything
%   confusing}

%It should be noted that only the coordination and waiting costs are divided by $\eta$. The ensemble of replicas are already running concurrently and 


% The RE algorithm involves the concurrent execution of multiple similar
% simulations, the \emph{replicas}.  There is a loose-coupling between
% the replicas in form of periodic exchange attempts between paired
% replicas. The traditional approach to RE is the synchronous model,
% which works well in an ideal scenario with a well defined model of
% resource availability. But with heterogenous systems and fluctuating
% resource availability, the asynchronous RE model could be more
% effectively used to conduct simulations. 
% >>>>>>> .r3441


%Previously, we demonstrated the usage of the SAGA Pilot-Job
%framework~\citep{saga_bigjob_condor_cloud} -- called the BigJob, to run
%RE simulations across multiple, heterogeneous distributed Grid and
%Cloud infrastructures~\citep{Luckow:2008fp}.
%\alnote{maybe we should also intro SAGA at some point} \jhanote{Yes} The Simple API for Grid Applications (SAGA)~\citep{saga_gfd90} is an API standardization effort within the Open Grid Forum (OGF)~\citep{ogf_web}, an international standards development body concerned primarily with standards for distributed computing. The various tasks that are carried out using the SAGA APIs include file staging, job spawning and the conduction of the exchange attempts.
%Further, we introduced several adaptivity modes, e.\,g.\ adaptive
%sampling that are able to react to dynamic changes in resource
%availabilities.

%\alnote{Not sure how many technical we need to provide...}  

%Traditionally, depending
%on the number of processes \texttt{N}, the manager creates \texttt{N/2} pairs
%of replicas.  Before launching a job, the manager ensures that all
%required input files are transferred to the respective resource. For
%this purpose, the SAGA File API and the GridFTP adaptor are used. The
%replica jobs are then submitted to the resource using the SAGA CPR
%API and the MIGOL/GRAM middleware.

%\jhanote{Mention that these are SAGA-based implementations. Something  else would be implemented differently}
%\alnote{Proposed structure: a) math. model b) sync c) async. We should give the reader some orientation in this section.}
% \jhanote{Abhinav: I still note the inconsistent and variable use of
%   algorithms, methods(?) and cases/approaches. I think we should refer
%   to Replica-Exchange as a ``class of algorithms'' or ``algorithm'',
%   and different implementations -- sync versus async} 

% \athotanote{RE class of algorithms - synchronous and asynchronous
%   models within RE class of algorithms; later sync, centralized-async,
%   decent-async implementation. right? or - RE class of algorithms -
%   synchronous and asynchronous implementation of RE class of
%   algorithms; later sync, centralized-async, decent-async
%   implementations. please suggest.}  \jhanote{I would say (i) RE Class
%   of Algorithms, (ii) Sync/Async algorithms too, and then centralized
%   or decentralized implementations of these sync versus async RE
%   algorithm}
%\subsection{Mathematical Model}
%\alnote{I think we should discuss the mathematical model before going
%  into the specifics of sync/async RE.}
%Here we first provide a basic mathematical model for different RE
% models which explains the terms involved. In this section, we aim to
% develop an equation for total time to run any RE simulation. If $t$
% is the average time between successful exchanges, and $p$ is the
% probability of a successful exchange,
%\begin{eqnarray}
%t=  {1 \over p} \times {[T_{MD} + T_{X} + T_{W}]} 
%\label{eq:timebtw}
%\end{eqnarray}
%where $T_{X}$ is the time to carry-out a pairwise exchange, which is
%comprised of (i) finding a partner, (ii) exchanging states including
%file transfer, (iii) book-keeping, and (iv) (re)starting the replica;
%$T_{W}$ is the time spent waiting for all replicas to complete running.
%Therefore, ${T_{X}} = {T_F + T_{ex} + T_{mgmt}}$ 
%and for ${T_{X}}^{async}, T_W = 0$
%The time ($T$) for N$_{X}$ exchanges is therefore $N_{X} \times t$
%If there are $\eta$ independent exchange events occurring
%concurrently, then the time $T$ for N$_x$ exchanges is $T \over \eta$.
