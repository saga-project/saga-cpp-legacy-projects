% cpedoc.tex V2.0, 13 May 2010

\documentclass[times]{cpeauth}

\usepackage{moreverb}

\usepackage[
%dvips,
colorlinks,bookmarksopen,bookmarksnumbered,citecolor=red,urlcolor=red]{hyperref}

\newcommand\BibTeX{{\rmfamily B\kern-.05em \textsc{i\kern-.025em b}\kern-.08em
T\kern-.1667em\lower.7ex\hbox{E}\kern-.125emX}}

\def\volumeyear{2012}

\begin{document}

\runningheads{A.~N.~Other}{Scalable, Extensible, Interoperable Pilot-Abstractions for Iterative MapReduce Applications on Clouds, Grids}

\title{An Extensible, Scalable, Interoperable Pilot-Abstractions for Iterative MapReduce Applications on Clouds and Grids}

\author{Melissa Romanus, Andre Luckow, Pradeep Mantha, Shantenu Jha\corrauth}

\address{Radical Research Group, Rutgers University}

\corraddr{Journals Production Department, John Wiley \& Sons, Ltd,
The Atrium, Southern Gate, Chichester, West Sussex, PO19~8SQ, UK.}

\begin{abstract}

The data generated by scientific applications is experiencing an exponential growth. The efficient use of distributed resources will be the key to making meaningful sense of all of the data produced. The recent extension of MapReduce to work more effectively on distributed data across different resources (Pilot-MapReduce) has shown one such example of handling large datasets. Pilot-MapReduce is based on Pilot-abstractions for compute (Pilot-Jobs) and data (Pilot-Data). Pilot-Jobs are used to couple the  map phase computation to the nearby
 source data, and Pilot-Data are used to move intermediate data using parallel data transfers to the reduce
 computation phase.
 This work can be extended from grid resources to work in cloud environments. In leveraging cloud architectures, Pilot-MapReduce can be furthered from hierarchical MapReduce to a more distributed, iterative MapReduce. The performance of both hierarchical and iterative MapReduce is then compared on grids versus clouds, as well as the trade-offs in each such infrastructure (i.e. the overhead times in spawning virtual machines, the geographic distribution of grids, etc). Finally, the combined use of both grids and clouds with Pilot-MapReduce is explored.

\end{abstract}

\keywords{MapReduce; Grid Computing; Cloud Computing; K-Means; Data-Intensive; Compute-Intensive}

\maketitle

\footnotetext[2]{Please ensure that you use the most up to date
class file,
available from the CPE Home Page at\\
\href{http://www3.interscience.wiley.com/journal/117946197/grouphome/home.html}{\texttt{http://www3.interscience.wiley.com/journal/117946197/grouphome/home.html}}}

\vspace{-6pt}

\section{Introduction}
\vspace{-2pt}
Motivation of problem in scaling data intensive applications- Interoperability, 
Scalability, Extensibility/Flexibility/usability.
   - Why is this a problem? Any real application requires this problem to be solved?

   - CMS, Atlas generates PBs of data/ day.

Why Iterative Mapreduce?

What Application?  ( k-means?)

Why k-means?

  - twister mapreduce used k-means?

  - k-means implemented using windows azure

What Infrastructure?

  - FutureGrid/ XSEDE ( Sierra, Kraken )

  - OSG ( Need Bliss Condor adaptor - Not yet developed )

  - Eucalytpus Cloud ( Ashley's contributions )
    - Get widely distributed instances. 

  - OpenStack Cloud ( Melissa contributions )
    - Get widely distributed instances. 



Experiments -

  - Scale data 1GB, 10GB, 100GB, 1000 GB
  - Scale Resources 1000 cores, 10000 cores, 50,000 cores


Some Research Questions?
 
Can we say something when to use cloud or When Grid ? 
Does minimizing queue wait time, distributed nature of Pilot-abstractions motivate Domain Scientists to use freely available Grid resources? 
Waiting time and cost increases as Number of instances required increase? 
HOw is it beneficial than Grids? 

Why Domain scientists are moving to cloud? Hype? Due to non-availability of necessary simple abstractions to scale applications on Grid?

\pagebreak

\section{Getting Started}


\begin{figure}
\setlength{\fboxsep}{0pt}%
\setlength{\fboxrule}{0pt}%
\begin{center}
\begin{boxedverbatim}
\documentclass[times]{cpeauth}
%\documentclass[times,doublespace]{cpeauth}%For paper submission

\begin{document}

\runningheads{<Initials and Surnames>}{<Short title>}

\title{<Initial cap, lower case>}

\author{<An Author\affil{1},
Someone Else\affil{2}\corrauth\ and Perhaps Another\affil{1}>}

\address{<\affilnum{1}First author's address
(in this example it is the same as the third author)\break
\affilnum{2}Second author's address>}

\corraddr{<Corresponding author's address (the second author in
this example)>. E-mail: <corresponding author's email address>}

%\cgs{<Contract/grant sponsor name (no number)>}
%\cgsn{<Contract/grant sponsor name>}{<number>}

\begin{abstract}
<Text>
\end{abstract}

\keywords{<List keywords>}

\maketitle

\section{Introduction}
.
.
.
\end{boxedverbatim}
\end{center}
%\vspace{-12pt}
\caption{Example header text.\label{F1}}
\end{figure}


\section{The Article Header Information}
The heading for any file using \textsf{\journalclass} is shown in
Figure~\ref{F1}.

\subsection{Remarks}
\begin{enumerate}
\item[(i)] In \verb"\runningheads" use `\emph{et~al.}' if there
are three or more authors.

\item[(ii)] Note the use of \verb"\affil" and \verb"\affilnum" to
link names and addresses. The author for correspondence is marked
by \verb"\corrauth" and \verb"\corraddr" is used to give that
author's address, which will be printed as a footnote, prefaced by
`Correspondence to:'.

\item[(iii)] For submitting a double-spaced manuscript, add
\verb"doublespace" as an option to the documentclass line.

\item[(iv)] Use \verb"\cgs" for giving details of financial
sponsors; alternatively use \verb"\cgsn" if the grant number is
also to be included. These details will be printed as a footnote,
with `Contract/grant sponsor:' and `contract/grant number:'
inserted in the appropriate places.

\item[(v)] The abstract should be capable of standing by itself,
in the absence of the body of the article and of the bibliography.
Therefore, it must not contain any reference citations.

\item[(vi)] Keywords are separated by semicolons.
\end{enumerate}

\begin{figure}
\setlength{\fboxsep}{0pt}%
\setlength{\fboxrule}{0pt}%
\begin{center}
\begin{boxedverbatim}
\begin{table}
\caption{<Table caption>}
\centering
\tabsize
\begin{tabular}{<table alignment>}
\toprule
<column headings>\\
\midrule
<table entries
(separated by & as usual)>\\
<table entries>\\
.
.
.\\
\bottomrule
\end{tabular}
\end{table}
\end{boxedverbatim}
\end{center}
\vspace{-6pt}
\caption{Example table layout.\label{F2}}
\vspace{-6pt}
\end{figure}

\section{The Body of the Article}

\subsection{Mathematics} \textsf{\journalclass} makes the full
functionality of \AmS\/\TeX\ available. We encourage the use of
the \verb"align", \verb"gather" and \verb"multline" environments
for displayed mathematics. \textsf{amsthm} is used for setting
theorem-like and proof environments. The usual \verb"\newtheorem"
command needs to be used to set up the environments for your
particular document.

\subsection{Figures and Tables} \textsf{\journalclass} includes the
\textsf{graphicx} package for handling figures.

Figures are called in as follows:
\begin{verbatim}
\begin{figure}
\centering
\includegraphics{<figure name>}
\caption{<Figure caption>}
\end{figure}
\end{verbatim}

For further details on how to size figures, etc., with the
\textsf{graphicx} package see, for example, \cite{R1}
or \cite{R3}. If figures are available in an
acceptable format (for example, .eps, .ps) they will be used but a
printed version should always be provided. \medbreak

The standard coding for a table is shown in Figure~\ref{F2}.

\subsection{Cross-referencing}
The use of the \LaTeX\ cross-reference system
for figures, tables, equations, etc., is encouraged
(using \verb"\ref{<name>}" and \verb"\label{<name>}").

\subsection{Acknowledgements} An Acknowledgements section is started with \verb"\ack" or
\verb"\acks" for \textit{Acknowledgement} or
\textit{Acknowledgements}, respectively. It must be placed just
before the References.

\subsection{Bibliography}
The normal commands for producing the reference list are:
\begin{verbatim}
\begin{thebibliography}{99}
\bibitem{<x-ref label>}
         <Reference details>
.
.
.
\end{thebibliography}
\end{verbatim}
where \verb"\bibitem{x-ref label}"
corresponds to \verb"\cite{x-ref label}" in the body of the article
and \verb"{99}" is the widest such number expected and determines
the width of the number column in the reference list.

Please note that the file \textsf{wileyj.bst} is available from
the same download page for those authors using \BibTeX.

\subsection{Double Spacing}
If you need to double space your document for submission please
use the \verb+doublespace+ option as shown in the sample layout in
Figure~\ref{F1}.

\section{Support for \textsf{\journalclass}}
We offer on-line support to participating authors. Please contact
us via e-mail at\\
\href{mailto:cpeauth-cls@wiley.co.uk}{\texttt{cpeauth-cls@wiley.co.uk}}.

We would welcome any feedback, positive or otherwise, on your
experiences of using \textsf{\journalclass}.

\section{Copyright Statement}
Please  be  aware that the use of  this \LaTeXe\ class file is
governed by the following conditions.

\subsection{Copyright}
Copyright \copyright\ \volumeyear\ John Wiley \& Sons, Ltd, The
Atrium, Southern Gate, Chichester, West Sussex, PO19~8SQ, UK. All
rights reserved.

\subsection{Rules of Use}
This class file is made available for use by authors who wish to
prepare an article for publication in \emph{\journalnamelc}
published by John Wiley \& Sons, Ltd. The user may not exploit any
part of the class file commercially.

This class file is provided on an \emph{as is}  basis, without
warranties of any kind, either express or implied, including but
not limited to warranties of title, or implied  warranties of
merchantablility or fitness for a particular purpose. There will
be no duty on the author[s] of the software or  John Wiley \&
Sons, Ltd to correct any errors or defects in the software. Any
statutory  rights you may have remain unaffected by your
acceptance of these rules of use.

\ack This class file was developed by Sunrise Setting Ltd,
Torquay, Devon, UK. Website:\\
\href{http://www.sunrise-setting.co.uk}{\texttt{www.sunrise-setting.co.uk}}

\begin{thebibliography}{9}

\bibitem{R1} Kopka~H, Daly~PW. 2003. \emph{A Guide to \LaTeX} (4th~edn).
Addison-Wesley.

\bibitem{R2} Lamport~L. 1994. \emph{\LaTeX: a Document Preparation System} (2nd~edn).
Addison-Wesley.

\bibitem{R3} Mittelbach~F, Goossens~M. 2004. \emph{The \LaTeX\ Companion}
(2nd~edn). Addison-Wesley.
\end{thebibliography}
\end{document}
