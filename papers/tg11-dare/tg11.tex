%\documentclass[3p,twocolumn]{article}
\documentclass[12pt]{article}

\usepackage{graphicx}
\usepackage{color}
\usepackage{url}
\usepackage{ifpdf}
\usepackage{hyperref}
\usepackage{xspace}
%\usepackage[draft]{pdfdraftcopy}

\setlength\parskip{-0.015em}
\setlength\parsep{-0.15em}

\newenvironment{shortlist}{
	\vspace*{-0.85em}
  \begin{itemize}
 \setlength{\itemsep}{-0.3em}
}{
  \end{itemize}
	\vspace*{-0.6em}
}

\usepackage{fullpage}
%\usepackage[top=tlength, bottom=blength, left=llength, right=rlength]{geometry} %http://en.wikibooks.org/wiki/LaTeX/Page_Layout
%\usepackage[margin=1in, paperwidth=5.5in, paperheight=8.5in]{geometry}

\usepackage{fancyhdr}
\setlength{\headheight}{16.0pt}
\pagestyle{fancy}
\headheight = 0pt
\headsep    = 25pt
\fancyhf{}
%\fancyhead[OC]{\bf {\it \footnotesize{Jha et al: A Case for SAGA as an Access Layer for DCI}}}

\newif\ifdraft
\drafttrue
\ifdraft
 \newcommand{\smnote}[1]{  {\textcolor{magenta} {***SM: #1}}}
 \newcommand{\jhanote}[1]{ {\textcolor{red}     {***SJ: #1}}}
 \newcommand{\olenote}[1]{ {\textcolor{blue}    {***OW: #1}}}
\else
 \newcommand{\smnote}[1]{}
 \newcommand{\jhanote}[1]{}
 \newcommand{\olenote}[1]{}
\fi

\newcommand{\dn}{\vspace*{0.33em}}
\newcommand{\dnn}{\vspace*{0.66em}}
\newcommand{\dnnn}{\vspace*{1em}}
\newcommand{\uppp}{\vspace*{-1em}}
\newcommand{\upp}{\vspace*{-0.66em}}
\newcommand{\up}{\vspace*{-0.33em}}
\newcommand{\shift}{\hspace*{1.00em}}

\newcommand{\T}[1]{\texttt{#1}}
\newcommand{\I}[1]{\textit{#1}}
\newcommand{\B}[1]{\textbf{#1}}
\newcommand{\BI}[1]{\B{\I{#1}}}
\newcommand{\F}[1]{\B{[FIXME: #1]}}
\newcommand{\TODO}[1]{\textcolor{red}{\B{TODO: #1}}}

\begin{document}

% \title{The Distributed Adaptive Runtime Environment (DARE) Framework : Enhancing Life Science Applications with Distributed Scalable HPC grids with a Lightweight, Versatile, Extensible Science Gateway Development}

\title{Building Gateways for Life-Science Applications using the
  Distributed Adaptive Runtime Environment (DARE) Framework}

\author{Joohyun Kim$^{1}$, Sharath Maddineni$^{1}$, Shantenu Jha$^{1,2}$, \\
  \small{\emph{$^{1}$Center for Computation \& Technology, Louisiana State University, USA}}\\
  \small{\emph{$^{2}$Department of Computer Science, Louisiana State University, USA}}\\
  \small{\emph{$^{*}$Contact Author \texttt{sjha@cct.lsu.edu}}} }

\maketitle

\section*{Abstract}
We present the Distributed Adaptive Runtime Environment (DARE)
framework, introducing four bioinformatics gateways, DARE-RFOLD,
DARE-DOCK, DARE-HTHP and DARE-NGS, which have been developed for RNA
secondary structure prediction, virtual screening using a docking
method, large-scale molecular dynamics simulations and alignment of high-throughput sequencing data,
respectively.  The DARE framework provides an efficient means for developing a production level gateway
that comprises a user-friendly access layer such as web UI as well as middleware, built upon SAGA/BigJob abstraction, running a target
scientific application whose capacity is significantly enhanced by a variety of
execution patterns on distributed scalable HPC resources such as Teragrid.  With the DARE framework, a lightweight, extensible, versatile, full-fledge gateway that seamlessly utilizes scalable infrastructure can be built for life science applications. 


\section{Introduction}

% \bibliographystyle{plain}
% \bibliography{egi-white-paper}

%\begin{figure}
% \centering
%\includegraphics[scale=0.45]{figures/workflow.png} 
%
%\caption{\small Overall workflow for a mapping procedure using BFAST.  In this work, we focus on the step for finding Candidate Alignment Locations (CALs).  }
%  \label{fig:workflow-bfast} 
% \end{figure}
%
%
%\begin{table}
%\begin{tabular}{|c|c|c|c|} 
%  \hline 
% BFAST command & Description & Features for \\ 
%  &  &     Parallelism \\ \hline \hline
%\texttt{bfast fasta2brg} & creation of a ref. genome  &    multiple independent contigs \\ \hline 
%\texttt{solid2fastq}  &  preparation of short reads files &     multiple sequence reads files \\ \hline
%
%\texttt{bfast index} & creation of reference genome indexes& multi-threading and  \\
% &   & low memory option  \\  \hline
%\texttt{bfast match} & finding candidate alignment locations  &  multi-threading and  \\
%& &  parallel execution \\ \hline
%\texttt{bfast localalign} & alignment of each CAL  &   parallel execution \\  \hline
%\texttt{bfast postprocess} & prioritization of alignments  &  parallel execution \\ \hline
%
%
%\hline
%\end{tabular} \caption{Description of BFAST commands and features for parallel and multi-threading execution}
% \label{table:bfast-summary} 
%\end{table}
%

\subsection{TeraGrid Usage by the life science community}

The single largest community is the life-sciences community --
including MD (25\%)...
Get a break-down of the total usage of the TG by discipline and
application type.

\subsection{Challenges in large scale life science applications}

For example, Next-Generation DNA Sequencing (hereafter, NGS) technologies challenge computational biology with unprecedented
amounts of data produced by their high-throughput capability.  Required data analytics for processing such sequenced data along with dealing with genome data sets available in public and private databases is overwhelmed by the pace of growing data volume.  This poses the question on how the challenge of large volume data-management as well as the requirement of analyzing large volumes of data are effectively handled.  

Interestingly, the cyberinfrastructure considerations requried to
support a broad-range of analytical approaches and at the scales
required, has received less attention that the data-management problem
and algorithmic advances.  Thus not surprisingly, traditional
production cyberinfrastructure, such as the TeraGrid, have not been
used for such data-intensive analytics. There are multiple reasons,
but a couple of contributing factors are: (i) insufficient runtime
enviroments (and abstractions) to support concurrent computational
capabilties with large-data sets to support data-analytics (beyond
visualization) in an easy, scalable and extensible fashion, (ii)
insufficient support for user-customizable data-intensive "workflows"
that effectively hide the challenges of data-movement and efficient
data-management whilst managing concurrent distributed (computational)
resources.



\subsection{Challenges in developing a gateway supporting heterogeneous HPC environments}



\section{Four Life Science Applications}

\begin{table}
\small
\begin{tabular}{|c|c|c|c|} 
  \hline Science Domain & Description & Target Application(s) 
  \\ \hline \hline 
  
  Molecular Dynamics &  &  \texttt{NAMD}  \\ \hline
  RNA Folding Prediction &  & \texttt{SFOLD, RNAFold} \\ \hline
  NGS data analytics &    &  \texttt{BFAST} \\ \hline
  Docking  &   & \texttt{Autodock} \\ \hline

\hline
\end{tabular} \caption{Four life science applications. The DARE framework was used for developing gateways for these applications}
 \label{table:four-applications} 
\end{table}



\subsection{Derive what the computational requirements and challenges
  for these applications are}


\jhanote{At the end we propose DARE based Gateways as a solution}


\section{Distributed Adaptive Runtime Environment}

\subsection{SAGA and BigJob abstraction}

To execute a scientific application using heterogeneous distributed computing resources, we develop the Distributed Adaptive Runtime Environment (DARE) framework\cite{dareurl}.  The framework is compose of an open source Web application framework, Pylons
and middleware of the application management system built upon SAGA an BigJob abstraction\cite{saga-ccgrid10,saga-royalsoc,saga-web,jha2009developing,ecmls10}.  This combination of the open source technology and the application management system enables us to develop a lightweight, extensible, full-fledged distributed computing science gateway quickly and effectively\cite{pylonsurl}. 

\subsection{Architecture}

\includegraphics[scale=0.70]{figures/DAREOutline.pdf}

\subsection{The three components -- Access Layer, Services Layer and
  Resource/Provisioning Layer} 

Level 2: (i) Computational aspect, (ii) Data managment and movement, 

Level 1: (iii)providing user interface

\section{Four DARE-based Gateways}
\subsection{DARE-NGS}
Our DARE-NGS gateway (http://cyder.cct.lsu.edu/dare-ngs) supports Genome-wide analysis on Teragrid and provides currently the mapping process using BFAST that aligns the large number of short reads sequenced from NGS machines onto a reference genome sequence, which is the first step in scientific discovery utilizing NGS sequencing-based protocols such as the whole genome resequencing, RNA-seq, and ChIP-seq.  De novo assembly without reference genome information is still in early stages.  Note that due to its design strategy, including other analysis tool such as assembly or extending to a pipeline with other tools successively applied after the mapping with DARE-NGS are straightforward and underway at this moment.

It is worth mentioning that the computational complexity
of the analysis (e.g. mapping) depends, upon other things, the size
and complexity of the reference genome and the data-size of short reads.
Given that these can vary significantly, the computational
requirements of NGS-analytics also varies (even between data-sets of
similar size).  Thus an efficient, scalable and extensible analytical
approaches must be supported by any framework supporting
NGS-analytics.

Recently, results of using DARE-NGS on the TeraGrid and
FutureGrid, using BFAST for mapping as the representative example of
the typical analysis that is required for NGS data were presented as a ECMLS11 (HPDC11) workshop paper.  

\subsection{DARE-Rfold}
DARE-Rfold gateway has been developed for a pipeline service for RNA structure prediction and non-coding RNA gene annotation with secondary structure formation.  In recent years, there has been an explosion of reports on newly
identified structured RNAs including many non-coding (nc) RNAs. These
RNAs are found to play significant roles in various gene regulation
mechanisms\cite{joyce1999,serganov2007,cruz2009,encode2007,amaral2008}.
The functions of these RNAs are often associated with complicated
structure formation, for example through the folding process occurring
in response to binding a metabolite or complex formation with other
proteins or nucleotides. These findings underscore the importance of
multiple folding pathways exhibited by an ensemble of structures
\cite{baek2008,blouin2009,dambach2009,hyeon2008,herschlag-nature-2010,ma2006}.

To support nc-RNA research and broadly for RNA structure prediction, with the science gateway, DARE-Rfold, a user is able to predict the Minimum Free Energy (MFE) secondary structure or an ensemble of structure sampled with a Boltzmann-weighted sampling scheme.  

Notably, many challenges in the computational investigation of RNA
folding dynamics exist.  For example, the support of high-throughput of highly-parallel tasks on heterogeneous distributed resources is critical.
We demonstrated the use of DARE-Rfold for the exploration of RNA folding energy landscape and structural characterization of SAM-I riboswitch sequences, which was greatly facilitated by the flexibility of our DARE framework and the capacity for many task computing to deal with 1000 structures from each sequence among 2910 sequences of SAM-I RAN family. 

\subsection{DARE-Dock}



\subsection{DARE-HTHP}






\subsection{How does this meet the requirements}
\smnote{ 1) Lets say we have "n" read files and with DARE it takes around time "t" time for matching step if we run it serially it would take n*t time. It probably exceeds wall time limit. Therefore speed up in match step depends how many number of read files we generate and process concurrently.
2) Yes we were able to process the complete run with entire Human Genome on QB and Ranger separately. (**I am currently working this to utilizing QB and Ranger together.) 
4) it should clearly provide the advantage with multiple resources. If we want to use the cloud resources from India to complete Human Genome run it is not practically possible because of the current limited disk size access provided by the FG Eucalyptus resources. Because whole human genome index files are of size 129 GB for Bfast matching step as opposed to HG 18 Chromosome 21 with size of  2 GB index files. On the other hand it also requires the temporary files disk space.  Thus it is important to utilize large capacity resources like QB and Ranger divide the work load across machines.}


\bibliographystyle{abbrv}
\bibliography{tg11}




\end{document}

